\documentclass{report}
\usepackage[margin=1in]{geometry} 
\usepackage{amsmath,amsthm,amssymb,amsfonts}
\usepackage{tabto}
\usepackage[yyyymmdd]{datetime}
\renewcommand{\dateseparator}{--}
\newcommand{\N}{\mathbb{N}}
\newcommand{\Z}{\mathbb{Z}}

% For definitions
%\newtheorem{defn}{Definition}[section]
%\newtheorem{thrm}{Theorem}[section]
%\newtheorem{ex}[Example}[section]
\newtheorem*{ex}{Example}
\newtheorem*{defn}{Definition}
\newtheorem*{thrm}{Theorem}
\newtheorem*{lemma}{Lemma}
\newtheorem*{result}{Result}

% For circled text
\usepackage{tikz}
\newcommand*\circled[1]{\tikz[baseline=(char.base)]{
            \node[shape=circle,draw,inner sep=0.8pt] (char) {#1};}}

\usepackage{pgf}
\usetikzlibrary{arrows, automata}

% For equation system alignment
\usepackage{systeme,mathtools}
% Usage:
%	\[
%	\sysdelim.\}\systeme{
%	3z +y = 10,
%	x + y +  z = 6,
%	3y - z = 13}

\newenvironment{problem}[2][Problem]{\begin{trivlist}
\item[\hskip \labelsep {\bfseries #1}\hskip \labelsep {\bfseries #2.}]}{\end{trivlist}}
%If you want to title your bold things something different just make another thing exactly like this but replace "problem" with the name of the thing you want, like theorem or lemma or whatever
 
%used for matrix vertical line
\makeatletter
\renewcommand*\env@matrix[1][*\c@MaxMatrixCols c]{%
  \hskip -\arraycolsep
  \let\@ifnextchar\new@ifnextchar
  \array{#1}}
\makeatother 
 
% Change chapter numbering
\newcommand{\mychapter}[2]{
	\setcounter{chapter}{#1}
	\setcounter{section}{0}
	\chapter*{#2}
	\addcontentsline{toc}{chapter}{#2}
}

\usepackage{graphicx}
\graphicspath{ {images/} }

\begin{document}
 
\tableofcontents{}
\mychapter{1}{2018-02-14}
\section{Unit Testing}
Qualities of unit tests:
\begin{enumerate}
\item Single function
\item Each test is independent and self contained
\item No output unless there is a failure
\item Check all return values and side effects. Side effects are things like changes in DB and changes to global variables and others. Asserts are used to check.
\end{enumerate}
\subsection{Coming up with Unit Tests}
Think of bad stuff that people can put in. Think of the limits of data input and output. Make sure if using RNG that a seed is set to verify the same results in multiple tests.\\

\subsection{FIRST Properties of Good Tests}
\begin{itemize}
\item {[}F{]}ast
\item {[}I{]}solated
\item {[}R{]}epeatable
\item {[}S{]}elf-validating
\item {[}T{]}imely
\end{itemize}

\subsection{Right-BICEP}
\begin{tabular}{ll}
Right & Are the results right?\\
B & Are all the boundary conditions correct?\\
I & Can you check inverse relationships?\\
C & Can you cross-check results using other means?\\
E & Can you force error conditions to happen?\\
P & Are performance characteristics within bounds?\\
\end{tabular}

\begin{defn}
Boundary conditions are the range of values that values must remain within. An example of this is entering in patient ages where to qualify for a treatment they must be between 50 and 70 yrs old. Anything outside of this range will be denied. An example for our quad solver is for input a we cannot allow the number 0 to be input. Thus we get a boundary condition.
\end{defn}

\begin{defn}
To force error conditions is just to apply inputs or methods that will definitely cause errors. Then you look for the appropriate error flagging and responses from the program.
\end{defn}

\subsection{Boundary Conditions: "CORRECT"}
\begin{itemize}
\item Conformance\\  Does the value conform to an expected format?
\item Ordering\\  Is the set of values ordered or unordered as appropriate?
\item Range\\  Is the value within reasonable min and max values?
\item Reference\\  Does the code reference anything external that isn't under direct control of the code itself?
\item Existence\\  Does the value exist (is it non-null, nonzero, present in a set, and so on)?
\item Cardinality\\  Are there exactly enough values?
\item Time (abs and rel)\\  Is everything happening in order? At the right time? In time?
\end{itemize}


\end{document}