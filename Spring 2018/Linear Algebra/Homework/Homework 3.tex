\documentclass[12pt]{article}
\usepackage[margin=1in]{geometry} 
\usepackage{amsmath,amsthm,amssymb,amsfonts}
\usepackage{tabto}

% Spacers:
% BEGIN BLOCK------------------------------------------
% END BLOCK============================================




\newcommand{\N}{\mathbb{N}}
\newcommand{\Z}{\mathbb{Z}}

% CUSTOM SETTINGS
% BEGIN BLOCK------------------------------------------
% For equation system alignment
\usepackage{systeme,mathtools}
% Usage:
%	\[
%	\sysdelim.\}\systeme{
%	3z +y = 10,
%	x + y +  z = 6,
%	3y - z = 13}


% For definitions
\newtheorem{defn}{Definition}[section]
\newtheorem{thrm}{Theorem}[section]

% For circled text
\usepackage{tikz}
\newcommand*\circled[1]{\tikz[baseline=(char.base)]{
            \node[shape=circle,draw,inner sep=0.8pt] (char) {#1};}}

\newenvironment{problem}[2][Problem]{\begin{trivlist}
\item[\hskip \labelsep {\bfseries #1}\hskip \labelsep {\bfseries #2.}]}{\end{trivlist}}
%If you want to title your bold things something different just make another thing exactly like this but replace "problem" with the name of the thing you want, like theorem or lemma or whatever
 
%used for matrix vertical line
\makeatletter
\renewcommand*\env@matrix[1][*\c@MaxMatrixCols c]{%
  \hskip -\arraycolsep
  \let\@ifnextchar\new@ifnextchar
  \array{#1}}
\makeatother 

% END BLOCK============================================

\newtheorem*{lemma}{Lemma} %added
\newtheorem*{result}{Result} %added
\newtheorem*{theorem}{Theorem} %added


% HEADER
% BEGIN BLOCK------------------------------------------
\usepackage{fancyhdr}
 
\pagestyle{fancy}
\fancyhf{}
\lhead{Homework \#3}
\chead{\thepage}
\rhead{Bryan Greener}
% END BLOCK============================================

% TITLE
% BEGIN BLOCK------------------------------------------
\title{Bryan Greener}
\author{MATH 2300 CRN:15163}
\date{2018-01-17}
\begin{document}
\maketitle
% END BLOCK============================================

\TabPositions{4cm}
\begin{enumerate}
\item[2.80.] Solve each system:\\
	\begin{enumerate}
	\item[(b)]
	\sysdelim{.}{.}\systeme[xyzt]{x+2y-3z+2t=2,2x+5y-8z+6t=5,3x+4y-5z+2t=4}
	\begin{align*}
	\begin{bmatrix}[rrrr|r]
	1 & 2 & -3 & 2 & 2\\
	2 & 5 & -8 & 6 & 5\\
	3 & 4 & -5 & 2 & 4\\
	\end{bmatrix}&
	\begin{bmatrix}[r]
	R_1\\ R_2\\ R_3\\
	\end{bmatrix}\\
	%
	\begin{bmatrix}[r]
	R_1\\
	R_2 - 2R_1\\
	R_3-3R_1\\
	\end{bmatrix}
	\begin{bmatrix}[rrrr|r]
	1 & 2 & -3 & 2 & 2\\
	0 & 1 & -2 & 2 & 1\\
	0 & -2 & 4 & -4 & -2\\
	\end{bmatrix}&
	\begin{bmatrix}[r]
	R_{4}\\ R_{5}\\ R_{6}\\	
	\end{bmatrix}\\
	%
	\begin{bmatrix}[r]
	R_4\\
	R_5\\
	R_6 + 2R_5\\
	\end{bmatrix}
	\begin{bmatrix}[rrrr|r]
	1 & 2 & -3 & 2 & 2\\
	0 & 1 & -2 & 2 & 1\\
	0 & 0 & 0 & 0 & 0\\
	\end{bmatrix}&
	\begin{bmatrix}[r]
	R_{7}\\ R_{8}\\ R_{9}\\	
	\end{bmatrix}\\
	%
	\begin{bmatrix}[r]
	R_7 - 2R_8\\
	R_8\\
	R_9\\
	\end{bmatrix}
	\begin{bmatrix}[rrrr|r]
	1 & 0 & 1 & -2 & 0\\
	0 & 1 & -2 & 2 & 1\\
	0 & 0 & 0 & 0 & 0\\
	\end{bmatrix}\\
	%
	y-2z+2t&=1\\
	y&=1+2z-2t\\
	x+z-2t&=0\\
	x&=-z+2t\\
	\end{align*}
	\end{enumerate}
\item[1.55.] Let $u=(1,-2,4)$, $v=(3,5,1)$, and $w=(2,1,-3)$. 
	\begin{enumerate}
	\item[(b)]
	Find $4u-v-3w$.
	\begin{align*}
		4u-v-3w
		&= 4(1,-2,4)-(3,5,1)-3(2,1,-3)\\
		&= (4,-8,16)-(3,5,1)-(6,3,-9)\\
		&= (-5,-16,24)
	\end{align*}
	\end{enumerate}
\item[1.56.] For the vectors in problem 1.55, find:
	\begin{enumerate}
	% 1.56.a
	\item $u \cdot v$, $u \cdot w$, $v \cdot w$
	\begin{align*}
		u \cdot v &= (1,-2,4)\cdot(3,5,1) = 3-10+4 = -3\\
		u \cdot w &= (1,-2,4)\cdot(2,1,-3)= 2-2-12 = -12\\
		v \cdot w &= (3,5,1) \cdot(2,1,-3)= 6+5-3  = 8
	\end{align*}
	% 1.56.b
	\item $||u||$, $||v||$, $||w||$
	\begin{align*}
		||u|| &= \sqrt{(1,-2,4)^2} = \sqrt{1+4+16} = \sqrt{21}\\
		||v|| &= \sqrt{(3,5,1)^2}  = \sqrt{9+25+1} = \sqrt{35}\\
		||w|| &= \sqrt{(2,1,-3)^2} = \sqrt{4+1+9}  = \sqrt{14}
	\end{align*}
	\end{enumerate}
\item[1.60.] Determine k so that u and v are orthagonal:
	\begin{enumerate}
	% 1.60.a
	\item $u=(3,k,-2)$, $v=(6,-4,-3)$\\
		Let $k=6$, then
		\[ u \cdot v = 18 - 24 + 6 = 0 \]
	% 1.60.b
	\item $u=(5,k,-4,2)$, $v=(1,-3,2,2k)$\\
		Let $k=3$, then
		\[ u \cdot v = 5 - 9 - 8 + 12 =  0 \]
	\end{enumerate}
\item[1.61.] Find x and y where:
	\begin{enumerate}
	% 1.61.a
	\item $(x,x+y)=(y-2,6)$\\
		Let $x=2$ and $y=4$, then
		\[ (x,x+y) = (2, 2+4) = (2,6) = (2-2,6) = (y-2,6) \]
	% 1.61.b
	\item $x(1,2)=-4(y,3)$\\
		Let $x=-6$ and $y=\frac{3}{2}$, then
		\[ x(1,2) = (-6,2(-6)) = (-6,-12) = (-4(\frac{3}{2}),-12) = -4(y,3) \]
	\end{enumerate}
	
\item[1.64.] Find $x$, $y$, and $z$ where
	$x \begin{bmatrix}[r] 1\\ 2\\ 3\\ \end{bmatrix}
	+ y \begin{bmatrix}[r] 2\\ 5\\ -1\\ \end{bmatrix}
	+ z \begin{bmatrix}[r] 4\\ -2\\ 3\\ \end{bmatrix}
	= \begin{bmatrix}[r] 9\\ -3\\ 16\\ \end{bmatrix}$\\
		
	\[ x \begin{bmatrix}[r] 1\\ 2\\ 3\\ \end{bmatrix}
	+ y \begin{bmatrix}[r] 2\\ 5\\ -1\\ \end{bmatrix}
	+ z \begin{bmatrix}[r] 4\\ -2\\ 3\\ \end{bmatrix}
	= \begin{bmatrix}[r] 9\\ -3\\ 16\\ \end{bmatrix}
	= \begin{bmatrix}[r] x\\ 2x\\ 3x\\ \end{bmatrix}
	+ \begin{bmatrix}[r] 2y\\ 5y\\ -y\\ \end{bmatrix}
	+ \begin{bmatrix}[r] 4z\\ -2z\\ 3z\\ \end{bmatrix}
	= \begin{bmatrix}[r] x+2y+4z\\ 2x+5y-2z\\ 3x-y+3z\\ \end{bmatrix}
	\]
	Let $x=3,y=-1,z=2$, then
	\[ \begin{bmatrix}[r] (3)+2(-1)+4(2)\\ 2(3)+5(-1)-2(2)\\ 3(3)-(-1)+3(2)\\ \end{bmatrix}
	= \begin{bmatrix}[r] 9\\ -3\\ 16\\ \end{bmatrix} \] 
\item[1.65.] Normalize each vector:
	\begin{enumerate}
	% 1.65.a
	\item $u=(5,-7)$\\
		\[ \hat{u} = \frac{\vec{u}}{||\vec{u}||} = \frac{(5,-7)}{\sqrt{5^2 + (-7)^2}} = \frac{(5,-7)}{\sqrt{74}} = (\frac{5}{\sqrt{74}},\frac{-7}{\sqrt{74}}) \]
	
	% 1.65.b
	\item $v=(1,2,-2,4)$\\
		\[ \hat{v} = \frac{\vec{v}}{||\vec{v}||} = \frac{(1,2,-2,4)}{\sqrt{1^2 + 2^2 + (-2)^2 + 4^2}} = \frac{(1,2,-2,4)}{\sqrt{25}} = (\frac{1}{5},\frac{2}{5},-\frac{2}{5},\frac{4}{5}) \]
		
	% 1.65.c
	\item $w=(\frac{1}{2},-\frac{1}{3},\frac{3}{4})$\\
		\[ \hat{w} = \frac{\vec{w}}{||\vec{w}||} = \frac{(\frac{1}{2},-\frac{1}{3},\frac{3}{4})}{\sqrt{(\frac{1}{2})^2 + (-\frac{1}{3})^2 + (\frac{3}{4})^2}} =  (\frac{6}{\sqrt{133}},-\frac{4}{\sqrt{133}},\frac{9}{\sqrt{133}}) \]
	\end{enumerate}

\bigskip
Problems 1.67 to 1.70 refer to the following matrices:
\[ 	A= \begin{bmatrix}[rr] 1 & 2\\ 3 & -4\\ \end{bmatrix}, \quad
	B= \begin{bmatrix}[rr] 5 & 0\\ -6 & 7\\ \end{bmatrix}, \quad
	C= \begin{bmatrix}[rrr] 1 & -3 & 4\\ 2 & 6 & -5\\ \end{bmatrix}, \quad
	D= \begin{bmatrix}[rrr] 3 & 7 & -1\\ 4 & -8 & 9\\ \end{bmatrix} \]
\item[1.67.] Find:
	\begin{enumerate}
	% 1.67.a
	\item $5A-2B$
		\[ 	5A-2B 
			= 5 \begin{bmatrix}[rr] 1 & 2\\ 3 & -4\\ \end{bmatrix}
			- 2 \begin{bmatrix}[rr] 5 & 0 \\ -6 & 7\\ \end{bmatrix}
			= \begin{bmatrix}[rr] 5 & 10\\ 15 & -20\\ \end{bmatrix}
			- \begin{bmatrix}[rr] 10 & 0\\ -12 & 14\\ \end{bmatrix}
			= \begin{bmatrix}[rr] -5 & 10\\ 27 & -34\\ \end{bmatrix} \]				
	
	% 1.67.b
	\item $C+D$
		\[ 	C+D 
			= \begin{bmatrix}[rrr] 1 & -3 & 4\\ 2 & 6 & -5\\ \end{bmatrix}
			+ \begin{bmatrix}[rrr] 3 & 7 & -1\\ 4 & -8 & 9\\ \end{bmatrix}
			= \begin{bmatrix}[rrr] 4 & 4 & 3\\  6 & -2 & 4\\ \end{bmatrix} \]	
	
	% 1.67.c
	\item $2C-3D$
		\[ 	2C-3D
			= 2 \begin{bmatrix}[rrr] 1 & -3 & 4\\ 2 & 6 & -5\\ \end{bmatrix}
			- 3 \begin{bmatrix}[rrr] 3 & 7 & -1\\ 4 & -8 & 9\\ \end{bmatrix}
			= \begin{bmatrix}[rrr] 2 & -6 & 8\\ 4 & 12 & -10\\ \end{bmatrix}
			- \begin{bmatrix}[rrr] 9 & 21 & -3\\ 12 & -24 & 27\\ \end{bmatrix} \]
			\[ = \begin{bmatrix}[rrr] -7 & -27 & 11\\ -8 & 36 & -37\\ \end{bmatrix} \]
	\end{enumerate}
\item[1.68.] Find:
	\begin{enumerate}
	\item $AB$
		\[ 	AB
			= \begin{bmatrix}[rr] 1 & 2\\ 3 & -4\\ \end{bmatrix}
			\begin{bmatrix}[rr] 5 & 0\\ -6 & 7\\ \end{bmatrix}
			= \begin{bmatrix}[rr] (1)(5)+(2)(-6) & (1)(0)+(2)(7)\\
								  (3)(5)+(-4)(-6) & (3)(0)+(-4)(7)\\ \end{bmatrix}
			= \begin{bmatrix}[rr] -7 & 14\\ 39 & 28\\ \end{bmatrix} \]
	\item $BA$
		\[ 	BA
			= \begin{bmatrix}[rr] 5 & 0\\ -6 & 7\\ \end{bmatrix}
			\begin{bmatrix}[rr] 1 & 2\\ 3 & -4\\ \end{bmatrix}
			= \begin{bmatrix}[rr] (5)(1)+(0)(3) & (5)(2)+(0)(-4)\\
								  (-6)(1)+(7)(3) & (-6)(2)+(7)(-4)\\ \end{bmatrix}
			= \begin{bmatrix}[rr] 5 & 10\\ 15 & -40\\ \end{bmatrix} \]
			
	\end{enumerate}
\item[1.70.] Find:
	\begin{enumerate}
	% 1.70.a
	\item $A^T$
		\[ 	A^T = \begin{bmatrix}[rr] 1 & 3\\ 2 & -4\\ \end{bmatrix} \]
		
	% 1.70.c	
	\item[(c)] $C^T C$
		\begin{align*}
		C^TC 
		&= \begin{bmatrix}[rr] 1 & 2\\ -3 & 6\\ 4 & -5\\ \end{bmatrix}
		\begin{bmatrix}[rrr] 1 & -3 & 4\\ 2 & 6 & -5\\ \end{bmatrix}\\
		&= \begin{bmatrix}[rrr]
		(1)(1)+(2)(2) & (1)(-3)+(2)(6) & (1)(4)+(2)(-5)\\
		(-3)(1)+(6)(2) & (-3)(-3)+(6)(6) & (-3)(4)+(6)(-5)\\
		(4)(1)+(-5)(2) & (4)(-3)+(-5)(6) & (4)(4)+(-5)(-5)\\
		\end{bmatrix}\\
		&= \begin{bmatrix}[rrr] 
		5 & 9 & -6\\ 
		9 & 45 & -42\\ 
		-6 & -42 & 41\\ 
		\end{bmatrix} 				
		\end{align*}
	
	\end{enumerate}
\item[1.76.] Let $A=\begin{bmatrix}[rr] 1 & 2\\ 3 & 6\\ \end{bmatrix}$.
	Find a 2 x 3 matrix $B$ with distinct entries such that $AB = 0$.\\
	Let $B= \begin{bmatrix}[rrr] 2 & 6 & 8\\ -1 & -3 & -4\\ \end{bmatrix}$, then
	\[ AB
	= \begin{bmatrix}[rrr]
	(1)(2)+(2)(-1) & (1)(6)+(2)(-3) & (1)(8)+(2)(-4)\\
	(3)(2)+(6)(-1) & (3)(6)+(6)(-3) & (3)(8)+(6)(-4)\\
	\end{bmatrix}
	= \begin{bmatrix}[rrr]
	0 & 0 & 0\\
	0 & 0 & 0\\
	\end{bmatrix} \]
	
\item[1.78.] Suppose $e_1=[1,0,0]$, $e_2=[0,1,0]$, $e_3=[0,0,1]$, and suppose $A=
	\begin{bmatrix}[rrrr]
	a_1 & a_2 & a_3 & a_4\\
	b_1 & b_2 & b_3 & b_4\\
	c_1 & c_2 & c_3 & c_4\\
	\end{bmatrix}$\\
	Find $e_1A$, $e_2A$, and $e_3A$.
	\begin{align*}
	e_1A &=
	\begin{bmatrix}[rrrr]
	1a_1+0b_1+0c_1 & 1a_2+0b_2+0c_2 & 1a_3+0b_3+0c_3 & 1a_4+0b_4+0c_4\\
	\end{bmatrix}\\
	&= \begin{bmatrix}[rrrr] a_1 & a_2 & a_3 & a_4\\ \end{bmatrix}\\
	e_2A &= 
	\begin{bmatrix}[rrrr]
	0a_1+1b_1+0c_1 & 0a_2+1b_2+0c_2 & 0a_3+1b_3+0c_3 & 0a_4+1b_4+0c_4\\
	\end{bmatrix}\\
	&= \begin{bmatrix}[rrrr] b_1 & b_2 & b_3 & b_4\\ \end{bmatrix}\\
	e_3A &= 
	\begin{bmatrix}[rrrr]
	0a_1+0b_1+1c_1 & 0a_2+0b_2+1c_2 & 0a_3+0b_3+1c_3 & 0a_4+0b_4+1c_4\\
	\end{bmatrix}\\
	&= \begin{bmatrix}[rrrr] c_1 & c_2 & c_3 & c_4\\ \end{bmatrix}
	\end{align*}
\end{enumerate}






\end{document}
