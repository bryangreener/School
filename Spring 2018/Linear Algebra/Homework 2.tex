\documentclass[12pt]{article}
\usepackage[margin=1in]{geometry} 
\usepackage{amsmath,amsthm,amssymb,amsfonts}
\usepackage{tabto}

% Spacers:
% BEGIN BLOCK------------------------------------------
% END BLOCK============================================




\newcommand{\N}{\mathbb{N}}
\newcommand{\Z}{\mathbb{Z}}

% CUSTOM SETTINGS
% BEGIN BLOCK------------------------------------------
% For equation system alignment
\usepackage{systeme,mathtools}
% Usage:
%	\[
%	\sysdelim.\}\systeme{
%	3z +y = 10,
%	x + y +  z = 6,
%	3y - z = 13}


% For definitions
\newtheorem{defn}{Definition}[section]
\newtheorem{thrm}{Theorem}[section]

% For circled text
\usepackage{tikz}
\newcommand*\circled[1]{\tikz[baseline=(char.base)]{
            \node[shape=circle,draw,inner sep=0.8pt] (char) {#1};}}

\newenvironment{problem}[2][Problem]{\begin{trivlist}
\item[\hskip \labelsep {\bfseries #1}\hskip \labelsep {\bfseries #2.}]}{\end{trivlist}}
%If you want to title your bold things something different just make another thing exactly like this but replace "problem" with the name of the thing you want, like theorem or lemma or whatever
 
%used for matrix vertical line
\makeatletter
\renewcommand*\env@matrix[1][*\c@MaxMatrixCols c]{%
  \hskip -\arraycolsep
  \let\@ifnextchar\new@ifnextchar
  \array{#1}}
\makeatother 

% END BLOCK============================================

\newtheorem*{lemma}{Lemma} %added
\newtheorem*{result}{Result} %added
\newtheorem*{theorem}{Theorem} %added


% HEADER
% BEGIN BLOCK------------------------------------------
\usepackage{fancyhdr}
 
\pagestyle{fancy}
\fancyhf{}
\lhead{Homework \#2}
\rhead{Bryan Greener}
\cfoot{\thepage}
% END BLOCK============================================

% TITLE
% BEGIN BLOCK------------------------------------------
\title{Bryan Greener}
\author{MATH 2300 CRN:15163}
\date{2018-01-14}
\begin{document}
\maketitle
% END BLOCK============================================

\TabPositions{4cm}

\begin{enumerate}
% 2.82
\item [2.82.] Find the values of k such that each of the following systems of unknowns x, y, and z has (i) a unique solution, (ii) no solution, (iii) an infinite number of solutions.
	\begin{enumerate}
	% 2.82.a	
	\item
	\sysdelim{.}{.}\systeme[xyz]{x-2y=1,x-y+kz=-2,ky+4z=6}
	\medskip
		\begin{enumerate}
		% 2.82.a.i
		\item [(i)]
		\begin{align*}
		\begin{bmatrix}[rrr|r]
		1 & -2 & 0 & 1\\
		1 & -1 & k & -2\\
		0 & k & 4 & 6\\
		\end{bmatrix}&
		\begin{bmatrix}[r]
		R_1\\ R_2\\ R_3\\
		\end{bmatrix}\\
		%
		\begin{bmatrix}[r]
		R_1\\
		R_2 - R_1\\
		R_3\\
		\end{bmatrix}
		\begin{bmatrix}[rrr|r]
		1 & -2 & 0 & 1\\
		0 & 1 & k & -3\\
		0 & k & 4 & 6\\
		\end{bmatrix}&
		\begin{bmatrix}[r]
		R_{4}\\ R_{5}\\ R_{6}\\
		\end{bmatrix}\\
		%
		\begin{bmatrix}[r]
		R_4\
		R_5\\
		R_6 - kR_5\\
		\end{bmatrix}
		\begin{bmatrix}[rrr|r]
		1 & -2 & 0 & 1\\
		0 & 1 & k & -3\\
		0 & 0 & 4-k^2 & 6+3k\\
		\end{bmatrix}&\\
		(4-k^2)z &= 6+3k\\
		z &= \frac{6+3k}{4-k^2}\\
		\end{align*}
		To obtain a unique solution, we let $k \neq \pm 2$.\\
		
		% 2.82.a.ii
		\item [(ii)]
		Using our result from 2.82.a.i, letting $k=-2$ gives us
		\begin{align*}
		z &= \frac{6+3(-2)}{4-(-2)^2}\\
		&= \frac{0}{0}
		\end{align*}
		Thus, when $k=-2$, the system has no solutions.\\
		
		% 2.82.a.iii
		\item [(iii)] 
		Using our result from 2.82.a.i, letting $k=2$ gives us
		\begin{align*}
		z &= \frac{6+3(2)}{4-(2)^2}\\
		&= \frac{12}{0}
		\end{align*}
		Thus, when $k=2$, the system has infinite solutions.\\
		
		\end{enumerate}	
		\medskip
	% 2.82.b
	\item
	\sysdelim{.}{.}\systeme[xyz]{x+y+kz=1,x+ky+z=1,kx+y+z=1}
	\medskip
		\begin{enumerate}
		% 2.82.b.i			
		\item [(i)]
		\begin{align*}
		\begin{bmatrix}[rrr|r]
		1 & 1 & k & 1\\
		1 & k & 1 & 1\\
		k & 1 & 1 & 1\\
		\end{bmatrix}&
		\begin{bmatrix}[r]
		R_1\\ R_2\\ R_3\\
		\end{bmatrix}\\
		%
		\begin{bmatrix}[r]
		R_1\\
		R_2 - R_1\\
		R_3 - kR_1\\
		\end{bmatrix}
		\begin{bmatrix}[rrr|r]
		1 & 1 & k & 1\\
		0 & k-1 & 1-k & 0\\
		0 & 1-k & 1-k^2 & 0\\
		\end{bmatrix}&
		\begin{bmatrix}[r]
		R_{4}\\ R_{5}\\ R_{6}\\
		\end{bmatrix}\\
		%
		\begin{bmatrix}[r]
		R_4\\
		R_5\\
		R_6 + R_5\\
		\end{bmatrix}
		\begin{bmatrix}[rrr|r]
		1 & 1 & k & 1\\
		0 & k-1 & 1-k & 0\\
		0 & 0 & 2 - k^2 - k & 0\\
		\end{bmatrix}&
		\begin{bmatrix}[r]
		R_{7}\\ R_{8}\\ R_{9}\\
		\end{bmatrix}\\
		%
		2-k^2-k &=0\\
		k^2+k-2 &= 0\\
		(k+2)(k-1) &=0\\
		k=1, \quad k&=-2
		\end{align*}	
		Thus, for a unique solution, $k \neq 1$ and $k \neq -2$.\\
		% 2.82.b.ii
		\item [(ii)]
		Using our result from 2.82.b.i, letting $k=-2$ gives us
		\begin{align*}
		(2-k^2-k)z &= 0\\
		(2-(-2)^2-(-2))z &= 0\\
		0 &= 0
		\end{align*}
		Thus, when $k=-2$ the system has no solutions.\\

		\pagebreak
		% 2.82.b.iii
		\item [(iii)]
		Using our result from 2.82.b.i, letting $k = 1$ gives us
		\begin{align*}
		(2-k^2-k)z &= 0\\
		(2-(1)^2-(1))z &= 0\\
		0 &= 0
		\end{align*}
		Thus, when $k=1$ the system has infinite solutions.\\
		
		\end{enumerate}	
		\medskip
	%2.82.c
	\item
	\sysdelim{.}{.}\systeme[xyz]{x+2y+2z=5,x+ky+3z=7,x+11y+kz=11}
	\medskip
		\begin{enumerate}
		% 2.82.c.i
		\item [(i)]
		\begin{align*}
		\begin{bmatrix}[rrr|r]
		1 & 2 & 2 & 5\\
		1 & k & 3 & 7\\
		1 & 11 & k & 11\\
		\end{bmatrix}&
		\begin{bmatrix}[r]
		R_1\\ R_2\\ R_3\\
		\end{bmatrix}\\
		%
		\begin{bmatrix}[r]
		R_1\\
		R_3 - R_1\\
		R_2 - R_1\\
		\end{bmatrix}
		\begin{bmatrix}[rrr|r]
		1 & 2 & 2 & 5\\
		0 & 9 & k-2 & 6\\
		0 & k-2 & 1 & 2\\
		\end{bmatrix}&
		\begin{bmatrix}[r]
		R_{4}\\ R_{5}\\ R_{6}\\
		\end{bmatrix}\\
		%
		\begin{bmatrix}[r]
		R_4\\
		(\frac{1}{9})R_5\\
		R_6\\
		\end{bmatrix}
		\begin{bmatrix}[rrr|r]
		1 & 2 & 2 & 5\\
		0 & 1 & \frac{k-2}{9} & \frac{2}{3}\\
		0 & k-2 & 1 & 2\\		
		\end{bmatrix}&
		\begin{bmatrix}[r]
		R_{7}\\ R_{8}\\ R_{9}\\
		\end{bmatrix}\\
		%
		\begin{bmatrix}[r]
		R_7\\
		R_8\\
		R_9 - (k-2)R_8\\
		\end{bmatrix}
		\begin{bmatrix}[rrr|r]
		1 & 2 & 2 & 5\\
		0 & 1 & \frac{k-2}{9} & \frac{2}{3}\\
		0 & 0 & -\frac{(k-2)^2}{9} & \frac{2(5-k)}{3}\\
		\end{bmatrix}&
		\begin{bmatrix}[r]
		R_{10}\\ R_{11}\\ R_{12}\\
		\end{bmatrix}\\
		%
		-\frac{(k-2)^2}{9} &= (\frac{2}{3})(5-k)\\
		k = 5, \quad k &= -1
		\end{align*}
		Thus, for a unique solution,$k \neq -1$ and $k \neq 5$. 
		% 2.82.c.ii
		\item [(ii)]
		Using our result from 2.82.c.i, letting $k = 5$ gives us
		\begin{align*}
		-(\frac{(k-2)^2}{9})z &= (\frac{2}{3})(5-k)\\
		-(\frac{((5)-2)^2}{9})z &= (\frac{2}{3})(5-(5))\\
		-(\frac{(3)^2}{9})z &= (\frac{2}{3})(0)\\
		-(1)z &= 0\\
		-z &= 0
		\end{align*}
		Thus there are no solutions when $k = 5$.
		
		% 2.82.c.iii
		\item [(iii)]
		Using our result from 2.82.c.i, letting $k = -1$ gives us
		\begin{align*}
		-(\frac{(k-2)^2}{9})z &= (\frac{2}{3})(5-k)\\
		-(\frac{((-1)-2)^2}{9})z &= (\frac{2}{3})(5-(-1))\\
		-(\frac{(-3)^2}{9})z &= (\frac{2}{3})(6)\\
		-(1)z &= 4\\
		-z &= 4
		\end{align*}
		I couldn't get this solution to come out correctly. I checked the book to verify my answer which is how I got $k=-1$ as a solution however no matter how many times I reworked this problem I couldn't end up with a solution that satisfied this condition.
		\end{enumerate}
	\end{enumerate}
\medskip
% 2.83
\item [2.83.] Determine whether or not each system has a nonzero solution.
	\begin{enumerate}
	% 2.83.a	
	\item
	\sysdelim{.}{.}\systeme[xyz]{x+3y-2z=0,x-8y+8z=0,3x-2y+4z=0}
	\begin{align*}
	\begin{bmatrix}[rrr|r]
	1 & 3 & -2 & 0\\
	1 & -8 & 8 & 0\\
	3 & -2 & 4 & 0\\
	\end{bmatrix}&
	\begin{bmatrix}[r]
	R_1\\ R_2\\ R_3\\
	\end{bmatrix}\\
	%
	\begin{bmatrix}[r]
	R_1\\
	R_2\\
	R_3 - 2R_1\\
	\end{bmatrix}
	\begin{bmatrix}[rrr|r]
	1 & 3 & -2 & 0\\
	1 & -8 & 8 & 0\\
	1 & -8 & 8 & 0\\
	\end{bmatrix}&
	\begin{bmatrix}[r]
	R_4\\ R_5\\ R_6\\
	\end{bmatrix}\\
	%
	\begin{bmatrix}[r]
	R_4\\
	R_5\\
	R_6 - R_5\\
	\end{bmatrix}
	\begin{bmatrix}[rrr|r]
	1 & 3 & -2 & 0\\
	1 & -8 & 8 & 0\\
	0 & 0 & 0 & 0\\
	\end{bmatrix}&
	\end{align*}
	Since we now have fewer equations than unknowns, there exists a nonzero solution.
	\medskip
	
	\pagebreak
	% 2.83.b
	\item
	\sysdelim{.}{.}\systeme[xyz]{x+3y-2z=0,2x-3y+z=0,3x-2y+2z=0}
	\begin{align*}
	\begin{bmatrix}[rrr|r]
	1 & 3 & -2 & 0\\
	2 & -3 & 1 & 0\\
	3 & -2 & 2 & 0\\
	\end{bmatrix}&
	\begin{bmatrix}[r]
	R_1\\ R_2\\ R_3\\
	\end{bmatrix}\\
	%
	\begin{bmatrix}[r]
	R_1\\
	R_2 - 2R_1\\
	R_3 - 3R_1\\
	\end{bmatrix}
	\begin{bmatrix}[rrr|r]
	1 & 3 & -2 & 0\\
	0 & -9 & 5 & 0\\
	0 & -11 & 8 & 0\\
	\end{bmatrix}&
	\end{align*}
	Since there is no way to reduce and remove any equations in order to get free variables, there are no nonzero solutions to this system.
	\medskip
	
	%2.83.c
	\item
	\sysdelim{.}{.}\systeme[xyzt]{x+2y-5z+4t=0,2x-3y+2z+3t=0,4x-7y+z-6t=0}
	\begin{theorem}[2.6]
	A homogeneous system with more unknowns than equations has a nonzero solution.
	\end{theorem}
	By theorem 2.6, since this system has four unknowns and only three equations, then this system has nonzero solutions.
	\medskip	
	\end{enumerate}
	
% 2.86
\item [2.86.] Reduce A to echelon form and then to row canonical form.
	\begin{enumerate}
	
	% 2.86.a	
	\item
	\begin{align*}
	A = 
	\begin{bmatrix}[rrrrr]
	1 & 2 & -1 & 2 & 1\\
	2 & 4 & 1 & -2 & 3\\
	3 & 6 & 2 & -6 & 5\\
	\end{bmatrix}
	\end{align*}
	
	\begin{align*}
	\begin{bmatrix}[rrrrr]
	1 & 2 & -1 & 2 & 1\\
	2 & 4 & 1 & -2 & 3\\
	3 & 6 & 2 & -6 & 5\\
	\end{bmatrix}&
	\begin{bmatrix}[r]
	R_1\\ R_2\\ R_3\\
	\end{bmatrix}\\
	%
	\begin{bmatrix}[r]
	R_1\\	
	R_2 - 2R_1\\
	R_3 - 3R_1\\
	\end{bmatrix}
	\begin{bmatrix}[rrrrr]
	1 & 2 & -1 & 2 & 1\\
	0 & 0 & 3 & -6 & 1\\
	0 & 0 & 5 & -12 & 2\\
	\end{bmatrix}&
	\begin{bmatrix}[r]
	R_4\\ R_5\\ R_6\\
	\end{bmatrix}\\
	%
	\begin{bmatrix}[r]
	R_4\\
	R_6-2R_5\\	
	R_5\\
	\end{bmatrix}
	\begin{bmatrix}[rrrrr]
	1 & 2 & -1 & 2 & 1\\
	0 & 0 & -1 & 0 & 0\\	
	0 & 0 & 3 & -6 & 1\\
	\end{bmatrix}&
	\begin{bmatrix}[r]
	R_7\\ R_8\\ R_9\\
	\end{bmatrix}\\
	%
	\begin{bmatrix}[r]
	R_7 - R_8\\
	(-1)R_8\\
	R_9 + 3R_8\\
	\end{bmatrix}
	\begin{bmatrix}[rrrrr]
	1 & 2 & 0 & 2 & 1\\
	0 & 0 & 1 & 0 & 0\\
	0 & 0 & 0 & -6 & 1\\
	\end{bmatrix}&
	\begin{bmatrix}[r]
	R_{10}\\ R_{11}\\ R_{12}\\
	\end{bmatrix}\\
	%
	\begin{bmatrix}[r]
	R_{10} + (\frac{1}{3})R_{12}\\
	R_{11}\\
	(-\frac{1}{6})R_{12}\\
	\end{bmatrix}
	\begin{bmatrix}[rrrrr]
	1 & 2 & 0 & 0 & \frac{4}{3}\\
	0 & 0 & 1 & 0 & 0\\
	0 & 0 & 0 & 1 & -\frac{1}{6}\\
	\end{bmatrix}&	
	\end{align*}
	% 2.86.b
	\item
	\begin{align*}
	A = 
	\begin{bmatrix}[rrrrr]
	2 & 3 & -2 & 5 & 1\\
	3 & -1 & 2 & 0 & 4\\
	4 & -5 & 6 & -5 & 7\\
	\end{bmatrix}
	\end{align*}
	
	\begin{align*}
	\begin{bmatrix}[rrrrr]
	2 & 3 & -2 & 5 & 1\\
	3 & -1 & 2 & 0 & 4\\
	4 & -5 & 6 & -5 & 7\\
	\end{bmatrix}&
	\begin{bmatrix}[r]
	R_1\\ R_2\\ R_3\\
	\end{bmatrix}\\
	%
	\begin{bmatrix}[r]
	R_3 - R_2\\
	R_2\\
	2R_1 - R_3\\
	\end{bmatrix}
	\begin{bmatrix}[rrrrr]
	1 & -4 & 4 & -5 & 3\\
	3 & -1 & 2 & 0 & 4\\
	0 & 11 & -10 & 15 & -5\\
	\end{bmatrix}&
	\begin{bmatrix}[r]
	R_4\\ R_5\\ R_6\\
	\end{bmatrix}\\
	%	
	\begin{bmatrix}[r]
	R_4\\
	R_5 - 3R_4\\
	R_6\\
	\end{bmatrix}
	\begin{bmatrix}[rrrrr]
	1 & -4 & 4 & -5 & 3\\
	0 & 11 & -10 & 15 & -5\\
	0 & 11 & -10 & 15 & -5\\
	\end{bmatrix}&
	\begin{bmatrix}[r]
	R_7\\ R_8\\ R_9\\
	\end{bmatrix}\\
	%	
	\begin{bmatrix}[r]
	R_7\\
	R_8\\
	R_9 - R_8\\
	\end{bmatrix}
	\begin{bmatrix}[rrrrr]
	1 & -4 & 4 & -5 & 3\\
	0 & 11 & -10 & 15 & -5\\
	0 & 0 & 0 & 0 & 0\\
	\end{bmatrix}&
	\begin{bmatrix}[r]
	R_{10}\\ R_{11}\\ R_{12}\\
	\end{bmatrix}\\
	%	
	\begin{bmatrix}[r]
	R_{10}\\
	(\frac{1}{11})R_{11}\\
	R_{12}\\
	\end{bmatrix}
	\begin{bmatrix}[rrrrr]
	1 & -4 & 4 & -5 & 3\\
	0 & 1 & -\frac{10}{11} & \frac{15}{11} & -\frac{5}{11}\\
	0 & 0 & 0 & 0 & 0\\
	\end{bmatrix}&
	\begin{bmatrix}[r]
	R_{13}\\ R_{14}\\ R_{15}\\
	\end{bmatrix}\\
	%	
	\begin{bmatrix}[r]
	R_{13} + 4R_{14}\\
	R_{14}\\
	R_{15}\\
	\end{bmatrix}
	\begin{bmatrix}[rrrrr]
	1 & 0 & \frac{4}{11} & \frac{5}{11} & \frac{13}{11}\\
	0 & 1 & -\frac{10}{11} & \frac{15}{11} & -\frac{5}{11}\\
	0 & 0 & 0 & 0 & 0\\
	\end{bmatrix}&
	\end{align*}
	\end{enumerate}
	
% 2.88
\item [2.88.] Using only 0s and 1s, list all possible 2 x 2 matrices in row canonical form.
\begin{align*}
\begin{bmatrix}[rr]
1 & 0\\
0 & 1\\
\end{bmatrix}
\begin{bmatrix}[rr]
1 & 1\\
0 & 0\\
\end{bmatrix}
\begin{bmatrix}[rr]
1 & 0\\
0 & 0\\
\end{bmatrix}
\begin{bmatrix}[rr]
0 & 1\\
0 & 0\\
\end{bmatrix}
\begin{bmatrix}[rr]
0 & 0\\
0 & 0\\
\end{bmatrix}
\end{align*}

% 2.89
\item [2.89.] Using only 0s and 1s, list all possible 3 x 3 matrices in row canonical form.
\begin{align*}
\begin{bmatrix}[rrr]
0 & 0 & 0\\
0 & 0 & 0\\
0 & 0 & 0\\
\end{bmatrix}
\begin{bmatrix}[rrr]
0 & 0 & 1\\
0 & 0 & 0\\
0 & 0 & 0\\
\end{bmatrix}
\begin{bmatrix}[rrr]
0 & 1 & 0\\
0 & 0 & 0\\
0 & 0 & 0\\
\end{bmatrix}
\begin{bmatrix}[rrr]
0 & 1 & 1\\
0 & 0 & 0\\
0 & 0 & 0\\
\end{bmatrix}
\begin{bmatrix}[rrr]
1 & 0 & 0\\
0 & 0 & 0\\
0 & 0 & 0\\
\end{bmatrix}
\begin{bmatrix}[rrr]
1 & 0 & 1\\
0 & 0 & 0\\
0 & 0 & 0\\
\end{bmatrix}
\begin{bmatrix}[rrr]
1 & 1 & 0\\
0 & 0 & 0\\
0 & 0 & 0\\
\end{bmatrix}
\begin{bmatrix}[rrr]
1 & 1 & 1\\
0 & 0 & 0\\
0 & 0 & 0\\
\end{bmatrix}\\
%
\begin{bmatrix}[rrr]
0 & 1 & 0\\
0 & 0 & 1\\
0 & 0 & 0\\
\end{bmatrix}
\begin{bmatrix}[rrr]
1 & 0 & 0\\
0 & 0 & 1\\
0 & 0 & 0\\
\end{bmatrix}
\begin{bmatrix}[rrr]
1 & 0 & 1\\
0 & 1 & 0\\
0 & 0 & 0\\
\end{bmatrix}
\begin{bmatrix}[rrr]
1 & 1 & 0\\
0 & 0 & 1\\
0 & 0 & 0\\
\end{bmatrix}
\begin{bmatrix}[rrr]
1 & 0 & 0\\
0 & 1 & 0\\
0 & 0 & 0\\
\end{bmatrix}
\begin{bmatrix}[rrr]
1 & 0 & 1\\
0 & 1 & 1\\
0 & 0 & 0\\
\end{bmatrix}
\begin{bmatrix}[rrr]
1 & 0 & 0\\
0 & 1 & 1\\
0 & 0 & 0\\
\end{bmatrix}
\begin{bmatrix}[rrr]
1 & 0 & 0\\
0 & 1 & 0\\
0 & 0 & 1\\
\end{bmatrix}
\end{align*}

\pagebreak
% 2.92
\item [2.92.] Consider the following system of unknowns x and y:
\[ \sysdelim{.}{.}\systeme[xy]{ax+by=1,cx+dy=0} \]
Show that if $ad-bc \neq 0$, then the system has the unique solution
\[ x=\frac{d}{ad-bc}, \quad y=-\frac{c}{ad-bc} \]
And show that if $ad-bc=0$ and if $c \neq 0$ or $d \neq 0$, then the system has no solution.

\begin{align*}
\begin{bmatrix}[rr|r]
a & b & 1\\
c & d & 0\\
\end{bmatrix}&
\begin{bmatrix}[r]
R_1\\ R_2\\
\end{bmatrix}\\
%
\begin{bmatrix}[r]
R_1\\
(\frac{1}{c})R_2\\
\end{bmatrix}
\begin{bmatrix}[rr|r]
a & b & 1\\
1 & \frac{d}{c} & 0\\
\end{bmatrix}&
\begin{bmatrix}[r]
R_3\\ R_4\\
\end{bmatrix}\\
%
\begin{bmatrix}[r]
R_3\\
R_4 - (\frac{1}{a})R_3\\
\end{bmatrix}
\begin{bmatrix}[rr|r]
a & b & 1\\
0 & \frac{d}{c} - \frac{b}{a} & -\frac{1}{a}\\
\end{bmatrix}&
\begin{bmatrix}[r]
R_5\\ R_6\\
\end{bmatrix}\\
%
(\frac{d}{c} - \frac{b}{a})y &= -\frac{1}{a}\\
y &= -\frac{c}{ad-bc}\\
cx+dy &= 0\\
cx+d(-\frac{c}{ad-bc}) &= 0\\
x &= \frac{d}{ad-bc}
\end{align*}
Let $ad-bc=0$, $c=k$, $d=l$ for some nonzero integers $k$ and $l$.
\begin{align*}
y &= -\frac{k}{0}\\
x &= \frac{l}{0}
\end{align*}
Since letting either $c$ or $d$ equal a nonzero integer results in division by 0, the system has no solution in either case.
\end{enumerate}

\end{document}