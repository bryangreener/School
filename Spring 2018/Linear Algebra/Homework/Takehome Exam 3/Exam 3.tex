\documentclass[12pt]{article}
\usepackage[margin=1in]{geometry} 
\usepackage{amsmath,amsthm,amssymb,amsfonts}
\usepackage{tabto}
\usepackage{hyperref}

% Spacers:
% BEGIN BLOCK------------------------------------------
% END BLOCK============================================




\newcommand{\N}{\mathbb{N}}
\newcommand{\Z}{\mathbb{Z}}

% CUSTOM SETTINGS
% BEGIN BLOCK------------------------------------------
% For equation system alignment
\usepackage{systeme,mathtools}
% Usage:
%	\[
%	\sysdelim.\}\systeme{
%	3z +y = 10,
%	x + y +  z = 6,
%	3y - z = 13}


% For definitions
\newtheorem{defn}{Definition}[section]
\newtheorem{thrm}{Theorem}[section]

% For circled text
\usepackage{tikz}
\newcommand*\circled[1]{\tikz[baseline=(char.base)]{
            \node[shape=circle,draw,inner sep=0.8pt] (char) {#1};}}

\newenvironment{problem}[2][Problem]{\begin{trivlist}
\item[\hskip \labelsep {\bfseries #1}\hskip \labelsep {\bfseries #2.}]}{\end{trivlist}}
%If you want to title your bold things something different just make another thing exactly like this but replace "problem" with the name of the thing you want, like theorem or lemma or whatever
 
%used for matrix vertical line
\makeatletter
\renewcommand*\env@matrix[1][*\c@MaxMatrixCols c]{%
  \hskip -\arraycolsep
  \let\@ifnextchar\new@ifnextchar
  \array{#1}}
\makeatother 

% END BLOCK============================================

\newtheorem*{lemma}{Lemma} %added
\newtheorem*{result}{Result} %added
\newtheorem*{theorem}{Theorem} %added
\theoremstyle{definition}
\newtheorem*{solution}{Solution} %added
\theoremstyle{plain}

% HEADER
% BEGIN BLOCK------------------------------------------
\usepackage{fancyhdr}
 
\pagestyle{fancy}
\fancyhf{}
\lhead{Exam \#3}
\rhead{Bryan Greener}
\cfoot{\thepage}
% END BLOCK============================================

% TITLE
% BEGIN BLOCK------------------------------------------
\title{Bryan Greener}
\author{MATH 2300 CRN:15163}
\date{2018-04-13}
\begin{document}
\maketitle
% END BLOCK============================================

\TabPositions{4cm}

\begin{enumerate}
\item[2.]Suppose a matrix is given as
\[ M=\begin{bmatrix}[rr]1&2\\1&1\\\end{bmatrix} \]
and define a pair of sequences according to
\begin{center}
$A_1=1$, $B_1=1$, and $\begin{bmatrix}[r]A_{n+1}\\B_{n+1}\\\end{bmatrix}=M\begin{bmatrix}[r]A_n\\B_n\\\end{bmatrix}$, $(n\geq 1)$
\end{center}
(a) Determine the first 10 terms of each of these sequences. (b) Determine the limit $\lim_{n\rightarrow\infty}(A_n/B_n)$, and prove your result. (Hint: Diagonalize $M$).
\begin{solution}
\begin{enumerate}
\item
\[ \begin{bmatrix}[r]3\\2\\\end{bmatrix},\begin{bmatrix}[r]7\\5\\\end{bmatrix},\begin{bmatrix}[r]17\\12\\\end{bmatrix},\begin{bmatrix}[r]41\\29\\\end{bmatrix},\begin{bmatrix}[r]99\\70\\\end{bmatrix},\begin{bmatrix}[r]239\\169\\\end{bmatrix},\begin{bmatrix}[r]577\\408\\\end{bmatrix},\begin{bmatrix}[r]1393\\985\\\end{bmatrix},\begin{bmatrix}[r]3363\\2378\\\end{bmatrix},\begin{bmatrix}[r]8119\\5741\\\end{bmatrix} \]

\item Next we diagonalize $M$ and get $P=\begin{bmatrix}[rr]\sqrt{2}&-\sqrt{2}\\1&1\\\end{bmatrix}$, $D=\begin{bmatrix}[rr]1+\sqrt{2}&0\\0&1-\sqrt{2}\\\end{bmatrix}$, and $P^{-1}=\begin{bmatrix}[rr]\frac{1}{2\sqrt{2}}&\frac{1}{2}\\-\frac{1}{2\sqrt{2}}&\frac{1}{2}\\\end{bmatrix}$. So $M^n=PD^nP^{-1}$. Thus $\lim_{n\rightarrow\infty}M^n=\lim_{n\rightarrow\infty}PD^nP^{-1}=PBP^{-1}$ where
\[ B=\begin{bmatrix}[rr]\infty &0\\0&0\\\end{bmatrix} \]
Thus $\lim_{n\rightarrow\infty}M^n = PBP^{-1}=\begin{bmatrix}[rr]\infty&\infty\\\infty&\infty\\\end{bmatrix}$.\\
Since every item in the limit of $M$ is infinity, then multiplying $M\begin{bmatrix}[r]A_n\\B_n\\\end{bmatrix}$ for $n\rightarrow\infty$ will result in $\begin{bmatrix}[r]\infty\\\infty\\\end{bmatrix}$.\\
Let $M$ be a $2\times2$ matrix $\begin{bmatrix}[rr]a&b\\c&d\\\end{bmatrix}$. Then
\[ \begin{bmatrix}[r]A_{n+1}\\B_{n+1}\\\end{bmatrix} = \begin{bmatrix}[rr]a&b\\c&d\\\end{bmatrix}\begin{bmatrix}[r]A_n\\B_n\\\end{bmatrix} = \begin{bmatrix}[r]aA_n+bB_n\\cA_n+dB_n\\\end{bmatrix} \]
where $A_1=1$ and $B_1=1$. 
\end{enumerate}
\end{solution}

\item[3]The owner of a rapidly expanding business finds that for the first five months of the year the sales are \$4.0, \$4.4, \$5.2, \$6.4, and \$8.0 (thousands). After plotting these figures, the owner speculates that they follow a quadratic function. Find the last-squares quadratic polynomial fit to these data and use it to project sales for the remaining seven months of the year.
	\begin{center}
	\begin{tabular}{c|l}
	Month&Sales\\
	\hline
	1&\$4,000.00\\
	2&\$4,400.00\\
	3&\$5,200.00\\
	4&\$6,400.00\\
	5&\$8,000.00\\
	6&\$9,797.86\\
	7&\$11,980.75\\
	8&\$14.499.47\\
	9&\$17,354.01\\
	10&\$20,544.39\\
	11&\$24,070.59\\
	12&\$27,932.62\\
	\end{tabular}
	\end{center}

\item Demonstrate an example of the Cayley-Hamilton theorem with a $5 \times 5$ matrix.
\begin{solution}
Let $A=\begin{bmatrix}[rrrrr]1&1&1&1&1\\0&2&1&1&1\\0&0&3&1&1\\0&0&0&4&1\\0&0&0&0&5\\\end{bmatrix}$.
Note that matrix $A$ is triangular with diagonal elements 1,2,3,4, and 5, and so its characteristic polynomial is
\[ \Delta(t)=(t-1)(t-2)(t-3)(t-4)(t-5) \]
So
\begin{align*}
\Delta(t)&=(t-1)(t-2)(t-3)(t-4)(t-5)\\
&= t^5 - 15 t^4 + 85 t^3 - 225 t^2 + 274 t - 120\\
&= \begin{bmatrix}[rrrrr]
	1&31&211&781&2101\\
	0&32&211&781&2101\\
	0&0&243&781&2101\\
	0&0&0&1024&2101\\
	0&0&0&0&3125\\\end{bmatrix}\\
	&+
	50625 \begin{bmatrix}[rrrrr]
	1&15&65&175&369\\
	0&16&65&175&369\\
	0&0&81&175&369\\
	0&0&0&256&369\\
	0&0&0&0&625\\\end{bmatrix}\\
	&+
	614125\begin{bmatrix}[rrrrr]
	1&7&19&37&61\\
	0&8&19&37&61\\
	0&0&27&37&61\\
	0&0&0&64&61\\
	0&0&0&0&125\\\end{bmatrix}\\
	&+
	50625\begin{bmatrix}[rrrrr]
	1&3&5&7&9\\
	0&4&5&7&9\\
	0&0&9&7&9\\
	0&0&0&16&9\\
	0&0&0&0&25\\\end{bmatrix}\\
	&+
	\begin{bmatrix}[rrrrr]
	274&274&274&274&274\\
	0&548&274&274&274\\
	0&0&822&274&274\\
	0&0&0&1096&274\\
	0&0&0&0&1370\\\end{bmatrix}\\
	&+
	\begin{bmatrix}[rrrrr]
	-120&0&0&0&0\\
	0&-120&0&0&0\\
	0&0&-120&0&0\\
	0&0&0&-120&0\\
	0&0&0&0&-120\\\end{bmatrix}
\end{align*}

\end{solution}


\item[5.] Use iterative methods to find the largest eigenvalue of the matrix
\[ M = \begin{bmatrix}[rrr]6&-4&1\\-4&6&-1\\1&-1&11\\\end{bmatrix} \]
and give a basis for its eigenspace. (Search using the key words "iterative" and "largest eigenvalue".)
\begin{solution}
Eigenvalues: 12, 9, 2\\
Basis for eigenspace:  
\[ \left\{ \begin{bmatrix}[r]1\\1\\0\\\end{bmatrix},\begin{bmatrix}[r]-2\\2\\2\\\end{bmatrix},\begin{bmatrix}[r]1\\-1\\2\\\end{bmatrix} \right\} \]
\end{solution}
\end{enumerate}
\end{document}