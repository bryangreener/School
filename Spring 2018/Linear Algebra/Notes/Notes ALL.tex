\documentclass{report}
%\setcounter{tocdepth}{1}
%\setcounter{secnumdepth}{4}
\usepackage[margin=1in]{geometry} 
\usepackage{amsmath,amsthm,amssymb,amsfonts}
\usepackage{tabto}
\usepackage[yyyymmdd]{datetime}
\renewcommand{\dateseparator}{--}
\newcommand{\N}{\mathbb{N}}
\newcommand{\Z}{\mathbb{Z}}

% For definitions
%\newtheorem{defn}{Definition}[section]
%\newtheorem{thrm}{Theorem}[section]
%\newtheorem{ex}[Example}[section]
\newtheorem*{ex}{Example}
\newtheorem*{defn}{Definition}
\newtheorem*{thrm}{Theorem}
\newtheorem*{lemma}{Lemma}
\newtheorem*{result}{Result}


% For circled text
\usepackage{tikz}
\newcommand*\circled[1]{\tikz[baseline=(char.base)]{
            \node[shape=circle,draw,inner sep=0.8pt] (char) {#1};}}

% For equation system alignment
\usepackage{systeme,mathtools}
% Usage:
%	\[
%	\sysdelim.\}\systeme{
%	3z +y = 10,
%	x + y +  z = 6,
%	3y - z = 13}

\newenvironment{problem}[2][Problem]{\begin{trivlist}
\item[\hskip \labelsep {\bfseries #1}\hskip \labelsep {\bfseries #2.}]}{\end{trivlist}}
%If you want to title your bold things something different just make another thing exactly like this but replace "problem" with the name of the thing you want, like theorem or lemma or whatever
 
%used for matrix vertical line
\makeatletter
\renewcommand*\env@matrix[1][*\c@MaxMatrixCols c]{%
  \hskip -\arraycolsep
  \let\@ifnextchar\new@ifnextchar
  \array{#1}}
\makeatother 
 
 




\begin{document}
 
\tableofcontents{}

\chapter{2018-01-25}
Problem 3.99 in homework:\\
We need to solve the unknowns such that $M^TM$ is equal to the identity matrix.
	\begin{align*}
	M &= \begin{bmatrix}[r] R_1\\ R_2\\ R_3\\ \end{bmatrix}\\
	MM^T &= \begin{bmatrix}[r] R_1\\ R_2\\ R_3\\ \end{bmatrix}
	\begin{bmatrix}[rrr] R_1 & R_2 & R_3\\ \end{bmatrix}\\
	&=
	\begin{bmatrix}[rrr]
	R_1R_1^T & R_1R_2^T & R_1R_3^T\\
	R_2R_1^T & R_2R_2^T & R_2R_3^T\\
	R_3R_1^T & R_3R_2^T & R_3R_3^T\\
	\end{bmatrix}\\
	&=
	\begin{bmatrix}[rrr]
	R_1 \cdot R_1 & R_1 \cdot R_2 & R_1 \cdot R_3\\
	\vdots & \vdots & \vdots\\
	\vdots & \vdots & \vdots\\
	\end{bmatrix}\\
	&=
	\begin{bmatrix}[rrr]
	1 & 0 & 0\\
	0 & 1 & 0\\
	0 & 0 & 1\\
	\end{bmatrix}
	\end{align*}
	
For problem 3.96.a:\\
\begin{proof}
Assume A and B are symmetric. Notice that $(A+B)^T=A^T+B^T = A + B$ by the law of transposes. Thus, $A+B$ is also symmetric.
\end{proof}

\section{Subspaces}
\begin{ex} V is any vector space implying $\{\vec{0}\}$ and $V$ are vector subspaces.\\
Confirm that $\{\vec{0}\}$ satisfies the subspace criterion.\\
Notice $\{\vec{0}\} \neq \{\}$. Also $\vec{0} + \vec{0} = \vec{0}\in \{\vec{0}\}$ implying closure under addition.\\
Also $\lambda\vec{0} = \vec{0}$ for all $\lambda$ implying closure under multiplication.
\end{ex}
\begin{thrm}
Every subspace of $\mathbb{R}^3$ is either
	\begin{enumerate}
	\item [0-D] The zero subspace $\{\begin{bmatrix}[r] 0\\ 0\\ 0\\ \end{bmatrix} \}$
	\item [1-D] A line passing through $\begin{bmatrix}[r] 0\\ 0\\ 0\\ \end{bmatrix}$
	\item [2-D] A plane passing through $\begin{bmatrix}[r] 0\\ 0\\ 0\\ \end{bmatrix}$
	\item [3-D] All of $\mathbb{R}^3$
	\end{enumerate}
\end{thrm}
\begin{ex}
$\{\begin{bmatrix}[r] x\\ y\\ z\\ \end{bmatrix} : x^2+y^2+z^2=1 \}$ is a sphere. It is a subset of $\mathbb{R}^3$ but is not a vector subspace. Vector subspaces proposed in the theorem above are all flat objects.
\end{ex}
\begin{thrm}
Let $A$ be an m x n matrix, and let $V$ be the set of all n-tuples (as columns) $\vec{V}$ such that $A\vec{v} = \vec{0}$. Then $V$ is a vector subspace of $\mathbb{R}^n$.\\
$A$ is (m x n), $\vec{v}$ is (n x 1).
\end{thrm}
\begin{proof}
Use the subspace criterion. (page 159 Theorem 4.2)\\
\begin{enumerate}
	\item Notice $A\vec{0}_n = \vec{0}_m$. Thus $\vec{0}_n \in V$.
	\item Assume $\vec{v}, \vec{w} \in V$. We must show $\vec{v} + \vec{w} \in V$.\\
	Notice $A(\vec{v} + \vec{w}) = A\vec{v} + A\vec{w} = \vec{0} + \vec{0} = \vec{0}$ by distributive law of matrix multiplication. Thus $V$ is closed under addition.
	\item Assume $\lambda \in \mathbb{R}$ and $\vec{v} \in V$. We must show $\lambda\vec{v} \in V$.\\
	Notice $A(\lambda\vec{v}) = \lambda(A\vec{v}) = \lambda\vec{0} = \vec{0}$ by property of matrix algebra. 
\end{enumerate}
\end{proof}

%====================================================

\chapter{2018-01-26}
\begin{defn}
Let $A$ be an m x n matrix. The nullspace of $A$ is the set of all vectors $\vec{v}$ such that $A\vec{v} = \vec{0}$.\\
Notation: $N(A) = null(A) = nulspace(A) = \{\vec{v}: A\vec{v} = \vec{0}\}$
\end{defn}

\begin{defn}
$M$ is orthogonal is $MM^T = I$.
\end{defn}

Five Concepts Required for Linear Algebra:
\begin{enumerate}
\item Linear Combinations
\item Span of Set (Spanning Set)
\item Linear Independence
\item Basis of a Vector Space
\item Dimension
\end{enumerate}

\begin{defn}
Let $V$ be a vector space and $S \subset V$. A linear combination of elements of $S$ is any expression having the form $C_1\vec{v}_1+C_2\vec{v}_2+\dots+C_n\vec{v}_n$ where $C_1,C_2,...,C_n$ are scalars and $\vec{v}_1,\vec{v}_2,...,\vec{v}_n$ are in S.
\end{defn}

\begin{ex}
Write $(12,4)$ as a linear combination of $(1,0)$ and $(0,1)$.\\
Solution: $(12,4) = 12(1,0)+4(0,1)$
\end{ex}
\begin{ex}
Write $3+4x+x^2$ as a linear combination of $1$, $x$, and $x^2$.\\
Solution: $3+4x+x^2=3(1)+4(x)+1(x^2)$
\end{ex}
\begin{ex}
Write $(12,4)$ as a linear combination of $(0,1)$ and $(1,1)$.\\
Solution: $(12,4) = x(0,1) + y(1,1)$.\\
$12 = 0x+1y$, $4=1x+1y$\\
$y=12$, $x=-8$\\
$(12,4) = -8(0,1)+12(1,1)$
\end{ex}

%==========================================

\chapter{2018-01-29}
\begin{defn}
Suppose $V$ is a vector space, and $S\subset V$ is any subset. The span of $S$ is the set of all linear combinations of elements of $S$.
\[ span(S) = \{ c_1\vec{v_1}+c_2\vec{v_2}+\dots+c_n\vec{v_n}: c_1,c_2,...,c_n \in \mathbb{F}, \vec{v_1},\vec{v_2},...,\vec{v_n} \in S \} \]
If $S=\{\}$, then $span(S)=\{\vec{0}\}$
\end{defn}
\begin{ex}
$span\{(1,0),(0,1)\} =$ all of $\mathbb{R}^2$\\
Notice $(x,y)=x(1,0)+y(0,1)$
\end{ex}
\begin{ex}
$span\{(0,1,0),(0,0,1)\} \subset \mathbb{R}^3$ is the $(y,z)$ plane in $\mathbb{R}^3$.
\end{ex}
\begin{ex}
$span\{(1,0,0),(2,0,0)\}$ is the x-axis in $\mathbb{R}^3$.\\
Since one of these vectors is a scaled version of the other, then we can remove it since it is redundant. Essentially if we have multiple vectors where a single elementary row operation can make two equal, then it can be removed. 
\end{ex}
\begin{thrm}
Suppose $V$ is a vector space, and $S \subset V$ is any subset. Then the span of $S$ is a vector subspace of $V$.
\end{thrm}
\begin{proof}
Use the subspace criterion.\\
Special case $S=\{\} \Rightarrow span(S)=\{\vec{0}\}$ the zero subspace.\\
Assume $S$ is not empty.
	\begin{enumerate}
	\item Notice $\vec{0} = 0\vec{v}$ for any $\vec{v} \in S$\\
	$\Rightarrow \vec{0} \in span(S)$.
	\item We use a shortcut to combine parts two and three of the subspace criterion.\\
	Assume $\lambda \in \mathbb{F}$ and $\vec{v},\vec{w} \in span(S)$. Prove $\lambda\vec{v}+\vec{w} \in span(S)$.
	\end{enumerate}
\end{proof}

\section{Spanning Set}
\begin{defn}
Suppose $V$ is a vector space and $W$ is a subspace. A spanning set for $W$ is a subset $S$ such that $span(S) = W$.
\end{defn}
\begin{ex}
Does the given set span $\mathbb{R}^3$?\\
$S_1=\{(1,0,0),(0,1,0),(0,0,1)\}$ YES\\
$S_2=\{(1,0,0),(0,1,0)\}$ NO (only two vectors)\\
$S_3=\{(1,0,0),(0,1,0),(0,0,1),(1,1,1)\}$ YES\\
$S_4=\{(1,0,0),(1,1,0),(1,1,1)\}$ YES\\
$S_4:(x,y,x)=(x-y)(1,0,0)+(y-z)(1,1,0)+z(1,1,1)$\\
$S_5=\{(1,2,3),(5,6,7)\}$ NO (only two vectors)\\
$S_6=\{(1,0,0),(0,1,0),(0,0,0)\}$ NO\\
$S_7=\{(0,1,1),(1,0,1),(1,1,0)\}$ YES (tip: write out as matrix where each vector is a column then do rref)\\
$S_8=\{(0,1,-1),(1,0,-1),(1,-1,0)\}$ NO
\end{ex}

%=================================================
\chapter{2018-01-30}
\section{Row and Column Spaces}
\begin{defn}
Suppose $A$ is an m x n matrix. The row space of $A$ is the span of the rows of $A$. The column space of $A$ is the span of the columns of $A$.
\end{defn}
\begin{ex}
\[ \begin{bmatrix}[rrrrr] 1 & 2 & 3 & 0 & 6\\ 0 & 0 & 0 & 1 & 7\\ \end{bmatrix} \]
\[ colspace(A) = span\{\begin{bmatrix}[r]1\\0\\ \end{bmatrix}, \begin{bmatrix}[rr]2\\0\\ \end{bmatrix}, ...\} \subset \mathbb{R}^2 = span\{\begin{bmatrix}[r]1\\0\\ \end{bmatrix}, \begin{bmatrix}[r]0\\1\\ \end{bmatrix} \} \]
\[ rowspace(A) = \{\begin{bmatrix}[rrrrr]1&2&3&0&6\\ \end{bmatrix},\begin{bmatrix}[rrrrr]0&0&0&1&7\\ \end{bmatrix}\} \subset \mathbb{R}^5 \]
In our colspace, we were able to reduce our span with elementary row operations but in our rowspace we aren't able to reduce it any further.
\end{ex}
Our previous example can be written as the following: $\sysdelim{.}{.}\systeme[x_1x_2x_3x_4x_5]{x_1+2x_2+3x_3+6x_5=0,x_4+7x_6=0}$\\
Recall $nullspace(A)=\{\vec{v}: A\vec{v} = 0\}$\\
\[ \vec{v} = \begin{bmatrix}[l] x_1\\x_2\\x_3\\x_4\\x_5\\ \end{bmatrix} = \begin{bmatrix}[l] -2x_2-3x_3-6x_5\\ free\\ free\\ -7x_5\\ free\\ \end{bmatrix} = \begin{bmatrix}[l] -2a-3b-6c\\a\\b\\-7c\\c\\ \end{bmatrix} = a\begin{bmatrix}[r]-2\\1\\0\\0\\0\\ \end{bmatrix}+b\begin{bmatrix}[r]-3\\0\\1\\0\\0\\ \end{bmatrix} + c\begin{bmatrix}[r]-6\\0\\0\\-7\\1\\ \end{bmatrix}\]
Spanning set for null space of $A$:
\[ s= \{\begin{bmatrix}[r]-2\\1\\0\\0\\0\\ \end{bmatrix}, \begin{bmatrix}[r]-3\\0\\1\\0\\0\\ \end{bmatrix},\begin{bmatrix}[r]-6\\0\\0\\-7\\1\\ \end{bmatrix} \} \]
To tell if you can reduce vectors, look at the zero rows. Since multiplying by a scalar will always give a zero in those rows, this means that there is no combination in which we get a nonzero value in the zero row. This means there are no redundancies. 

\section{Class Questions}
\# 4.36 in homework: $(a,b)+(c,d) = (a+c,b+d)$\\
Above is a function for how to add vectors together.\\
$k(a,b)=(ka,0)$ This is the function for vector scalars.\\
Verify that this satisfies $A_4$\\
Suppose that $(a,b),(c,d)$ are given. Notice $(a,b)\vec{+}(c,d)=(a+c,b+d)$ by definition of "$\vec{+}$". By the commutative property of addition, we can rewrite this as $(c+a,a+b) = (c,d)\vec{+}(a,b)$ by definition of "$\vec{+}$".\\
$A_1-A_4$: These hold because the addition is defined as usual.\\
To show $M_4$ is violated, find a specific $\vec{u}$ such that $1\vec{u} \neq \vec{u}$.
\section{Linear Independence}













\end{document}