\documentclass[12pt]{article}
\usepackage[margin=1in]{geometry} 
\usepackage{amsmath,amsthm,amssymb,amsfonts}
\usepackage{tabto}

% Spacers:
% BEGIN BLOCK------------------------------------------
% END BLOCK============================================




\newcommand{\N}{\mathbb{N}}
\newcommand{\Z}{\mathbb{Z}}

% CUSTOM SETTINGS
% BEGIN BLOCK------------------------------------------
% For equation system alignment
\usepackage{systeme,mathtools}
% Usage:
%	\[
%	\sysdelim.\}\systeme{
%	3z +y = 10,
%	x + y +  z = 6,
%	3y - z = 13}


% For definitions
\newtheorem{defn}{Definition}[section]
\newtheorem{thrm}{Theorem}[section]

% For circled text
\usepackage{tikz}
\newcommand*\circled[1]{\tikz[baseline=(char.base)]{
            \node[shape=circle,draw,inner sep=0.8pt] (char) {#1};}}

\newenvironment{problem}[2][Problem]{\begin{trivlist}
\item[\hskip \labelsep {\bfseries #1}\hskip \labelsep {\bfseries #2.}]}{\end{trivlist}}
%If you want to title your bold things something different just make another thing exactly like this but replace "problem" with the name of the thing you want, like theorem or lemma or whatever
 
%used for matrix vertical line
\makeatletter
\renewcommand*\env@matrix[1][*\c@MaxMatrixCols c]{%
  \hskip -\arraycolsep
  \let\@ifnextchar\new@ifnextchar
  \array{#1}}
\makeatother 

% END BLOCK============================================

\newtheorem*{lemma}{Lemma} %added
\newtheorem*{result}{Result} %added
\newtheorem*{theorem}{Theorem} %added


% HEADER
% BEGIN BLOCK------------------------------------------
\usepackage{fancyhdr}
 
\pagestyle{fancy}
\fancyhf{}
\lhead{Homework \#2}
\rhead{Bryan Greener}
\cfoot{\thepage}
% END BLOCK============================================

% TITLE
% BEGIN BLOCK------------------------------------------
\title{Bryan Greener}
\author{MATH 2300 CRN:15163}
\date{2018-01-14}
\begin{document}
\maketitle
% END BLOCK============================================

\TabPositions{4cm}

\begin{enumerate}
\item Suppose you have algorithms with the five running times listed below. How much slower do each of these algorithms get when you (a) double the input size, or (b) increase the input size by one?
	\begin{enumerate}
	\item $n^2$
		\begin{enumerate}
		\item $(2n)^2 = 4n^2$ Increase runtime by factor of 4.
		\item $(n+1)^2 = n^2+2n+1$ Increase by $2n+1$.
		\end{enumerate}
	\item $n^3$
		\begin{enumerate}
		\item $(2n)^3 = 8n^3$. Increase runtime by factor of 8.
		\item $(n+1)^3 = (n^2+2n+1)(n+1) = n^3+3n^2+n+1$ Increase by $3n^2+3n+1$.
		\end{enumerate}
	\item $100n^2$
		\begin{enumerate}
		\item $100(2n)^2 = 400n^2 = 4(100n^2)$ Increase runtime by factor of 4.
		\item $100(n+1)^2 = 100n^2+200n+100$ Increase by $200n+100$.
		\end{enumerate}
	\item $n \log(n)$
		\begin{enumerate}
		\item $(2n)\log(2n) = 2n\log(n)+2n\log(2) = 2n\log(n)$ Increase by factor of 2.
		\item $(n+1)\log(n+1) = \log(n+1)^{n+1} = n\log(n)+\log(n+1)+n(\log(n+1)-\log(n))$ Increase by $\log(n+1)+n(\log(n+1)-\log(n)$.
		\end{enumerate}
	\item $2^n$
		\begin{enumerate}
		\item $2^{2n} = (2^n)^2$ Increase by original runtime squared. 
		\item $2^(n+1) = 2^n*2^1 = 2*2^n$ Increase by a factor of 2.
		\end{enumerate}
	\end{enumerate}
\item Suppose you have algorithms with the six running times listed below. Suppose you have a computer that can perform $10^{10}$ operations per second, and you need to computer a result in at most an hour of computation. For each of the algorithms, what is the largest input size of n for which you would be able to get the result within an hour?
	\begin{enumerate}
	\item $n^2$\\
	\begin{align*}
	n^2 &= 3600s*10^{10}\\
	n &= \sqrt{3600s*10^{10}}\\
	&= 600,000
	\end{align*}
	
	\item $n^3$\\
	\begin{align*}
	n^3 &= 3600*10^{10}\\
	n &= (3600*10^{10})^{1/3}\\
	&= 10,000x6^{2/3}
	\end{align*}

	\item $100n^2$\\
	\begin{align*}
	100n^2 &= 3600*10^{10}\\
	n^2 &= \frac{3600*10^{10}}{100}\\
	n &= 600000
	\end{align*}

	\item $n \log(n)$\\
	\begin{align*}
	n \log(n) &= 3600*10^{10}\\
	\log(n)^n &= 3600*10^{10}\\
	n^n &= 2^{3600*10^{10}}
	\end{align*}

	\item $2^n$\\
	\begin{align*}
	2^n &= 3600*10^{10}\\
	n\log_2(2) &= \log_2(36*10^{12})\\
	n &= \log_2(6^2*10^{12})\\
	&= 2log_2(6)+12\log_2(10)\\
	&\approx 45
	\end{align*}

	\item $2^{2^n}$\\
	\begin{align*}
	2^{2^n} &= 36*10^{12}\\
	\log(2^{2^n}) &= \log(26*10^{12})\\
	2^n &= 2log_2(6)+12\log_2(10)\\
	\log(2^n) &= \log(45)\\
	n &\approx 5
	\end{align*}
	
	\end{enumerate}
\item Take the following list of functions and arrange them in ascending order of growth rate. That is, if function $g(n)$ immediately following function $f(n)$ in your list, then it should be the case that $f(n)$ is $O(g(n))$.
	\begin{align*}
	f_1(n) &= n^{2.5}\\
	f_2(n) &= sqrt(2n)\\
	f_3(n) &= n + 10\\
	f_4(n) &= 10^n\\
	f_5(n) &= 100^n\\
	f_6(n) &= n^2\log(n)
	\end{align*}
	In order from left to right in terms of growth rate, we get: $f_2,f_3,f_6,f_1,f_4,f_5$
	
\item Take the following list of functions and arrange them in ascending order of growth rate.
	\begin{align*}
	g_1(n) &= 2^{\sqrt{\log(n)}}\\
	g_2(n) &= 2^n\\
	g_3(n) &= n^{\frac{4}{3}}\\
	g_4(n) &= n(\log(n))^3\\
	g_5(n) &= n^{\log(n)}\\
	g_6(n) &= n^{2^n}\\
	g_7(n) &= 2^{n^2}
	\end{align*}
	In order from left to right in terms of growth rate, we get: $g_1,g_3,g_4,g_5,g_2,g_7,g_6$

\item Assume you have functions $f$ and $g$ such that $f(n)$ is $O(g(n))$. For each of the following statements, decide whether you think it is true or false and give a proof or counterexample.
	\begin{enumerate}
	\item $\log_2(f(n))$ is $O(\log_2(g(n)))$\\
	Let $f(n)=2$ and $g(n)=1$. Then $\log_2(f(n))=\log_2(2) = 1$ and $\log_2(g(n) = \log_2(1) = 0$. Thus, this is a case where $f(n) \geq O(g(n))$ and so the original statement is false. 
	
	\item $2^{f(n)}$ is $O(2^{g(n)})$\\
	Let $f(n) = 2n$ and $g(n) = n$. Then $2^{f(n)} = 2^{2n} = 4^n$ and $2^{g(n)} = 2^n$. Since $4^n \geq 2^n$, the original statement is false.
	\item $f(n)^2$ is $O(g(n)^2)$\\
	
	\end{enumerate}


\end{enumerate}

\end{document}