\documentclass[12pt]{article}
\usepackage[margin=1in]{geometry} 
\usepackage{amsmath,amsthm,amssymb,amsfonts}
\usepackage{tabto}
\usepackage{hyperref}

\usepackage{arydshln} % gives hdashline and cdashline
\newcommand*{\tempb}{\multicolumn{1}{:c}{}} % Used for block matrices

% Spacers:
% BEGIN BLOCK------------------------------------------
% END BLOCK============================================




\newcommand{\N}{\mathbb{N}}
\newcommand{\Z}{\mathbb{Z}}

% CUSTOM SETTINGS
% BEGIN BLOCK------------------------------------------
% For equation system alignment
\usepackage{systeme,mathtools}
% Usage:
%	\[
%	\sysdelim.\}\systeme{
%	3z +y = 10,
%	x + y +  z = 6,
%	3y - z = 13}


% For definitions
\newtheorem{defn}{Definition}[section]
\newtheorem{thrm}{Theorem}[section]

% For circled text
\usepackage{tikz}
\usetikzlibrary{matrix}
\newcommand*\circled[1]{\tikz[baseline=(char.base)]{
            \node[shape=circle,draw,inner sep=0.8pt] (char) {#1};}}

\newenvironment{problem}[2][Problem]{\begin{trivlist}
\item[\hskip \labelsep {\bfseries #1}\hskip \labelsep {\bfseries #2.}]}{\end{trivlist}}
%If you want to title your bold things something different just make another thing exactly like this but replace "problem" with the name of the thing you want, like theorem or lemma or whatever
 
%used for matrix vertical line
\makeatletter
\renewcommand*\env@matrix[1][*\c@MaxMatrixCols c]{%
  \hskip -\arraycolsep
  \let\@ifnextchar\new@ifnextchar
  \array{#1}}
\makeatother 

% END BLOCK============================================

\newtheorem*{lemma}{Lemma} %added
\newtheorem*{result}{Result} %added
\newtheorem*{theorem}{Theorem} %added
\theoremstyle{definition}
\newtheorem*{solution}{Solution} %added
\theoremstyle{plain}

% HEADER
% BEGIN BLOCK------------------------------------------
\usepackage{fancyhdr}
 
\pagestyle{fancy}
\fancyhf{}
\lhead{Homework \#15}
\rhead{Bryan Greener}
\cfoot{\thepage}
% END BLOCK============================================

% TITLE
% BEGIN BLOCK------------------------------------------
\title{Bryan Greener}
\author{MATH 2300 CRN:15163}
\date{2018-04-20}
\begin{document}
\maketitle
% END BLOCK============================================

\TabPositions{4cm}
\begin{enumerate}
\item[11.16]Let $L$ be the linear transformation on $\mathbb{R}^2$ that reflects points across the line $y=kx$, where $k>0$. (See Fig. 11-1)
	\begin{enumerate}
	\item Show that $v_1=(1,k)$ and $v_2=(k,-1)$ are eigenvectors of $L$.\\
		The vector $v_1=(1,k)$ lies on the line $y=kx$. So $L(v_1)=v_1$. Thus $v_1$ is an eigenvector of $L$ belonging to the eigenvalue $\lambda_1=1$. $v_2=(k,-1)$ is perpendicular to the line $y=kx$ so $L(v_2)=-v_2$. Therefore $v_2$ is an eigenvector of $L$ beloging to the eigenvalue $\lambda_2=-1$.
	\item Show that $L$ is diagonalizable, and find such a diagonal representation of $D$.\\
		Our set $S=\{v_1,v_2\}$ is a basis of $\mathbb{R}^2$ of eigenvectors of $L$. So $L$ is diagonalizable with $D=\begin{bmatrix}[rr]1&0\\0&-1\\\end{bmatrix}$.
	\end{enumerate}
\item[11.28]Let $A=\begin{bmatrix}[rrr]3&1&3\\1&3&1\\1&1&3\\\end{bmatrix}$.
	\begin{enumerate}
	\item Find the characteristic polynomial $\Delta(t)$ and all eigenvalues of $A$.\\
		By using the trace, determinant, and the determinants $A_{1,1},A_{2,2},A_{3,3}$, we get the characteristic polynomial $\Delta(t) = t^3-9t^2+24t-20 = (t-2)^2(t-5)$. Thus our eigenvalues are $\lambda=2$ with a multiplicity of 2 and $\lambda=5$. 
	\item Find a maximal set $S$ of nonzero orthogonal eigenvectors of $A$.\\
		Let $\lambda=2$:
		\[ M-\lambda I = \begin{bmatrix}[rrr]1&1&1\\1&1&1\\1&1&1\\\end{bmatrix} \]
		Thus since we have two repeated rows we will have two free variables. Thus by parameterizing this solution we get the vector $v_1=(0,1,-1)$. Next we find a second vector which is orthogonal to both $v_1$ and our resulting matrix. By setting $0,1,1$ and $1,1,1$ into a matrix and reducing then solving the system, we get our second vector $v_1=(2,-1,-1)$.\\
		Next we repeat this process and find $M-\lambda I$ for $\lambda=5$ to get $v_3=(1,1,1)$. we now have three eigenvectors which are perpendicular to each other. These three vectors form our maximal set of nonzero orthogonal eigenvectors of $A$.
	\item Find an orthogonal matrix $P$ such that $D=P^{-1}AP$ is diagonal.\\
		We normalize $v_1,v_2,v_3$ to get an orthonormal basis. Setting these orthonormal vectors as columns in a matrix gives us
		\[ P = \begin{bmatrix}[rrr]0&\frac{2}{\sqrt{6}}&\frac{1}{\sqrt{3}}\\\frac{1}{\sqrt{2}}&-\frac{1}{\sqrt{6}}&\frac{1}{\sqrt{3}}\\-\frac{1}{\sqrt{2}}&-\frac{1}{\sqrt{6}}&\frac{1}{\sqrt{3}}\\\end{bmatrix} \]
		Thus $D=P^TAp = \begin{bmatrix}[rrr]2&0&0\\0&2&0\\0&0&5\\\end{bmatrix}$.
	\end{enumerate}
	
\item[11.51]Let $A$ be an n-square matrix. Without using the Cayley-Hamilton theorem, prove that $A$ is a root of nonzero polynomial of degree $\leq n^2$.
	\begin{proof}
	Let $N=n^2$. Consider the $N+1$ matrices $1,A,A^2,...,A^n$. Recall that the vector space $V$ of $n\times n$ matrices has dimensions $N=n^2$. Thus the above $N+1$ matrices are linearly dependent. Thus there exist scalars $a_0,a_1,...,a_N$ not all zero for which
	\[ a_NA^N+\cdots + a_1A+a_0 I = 0 \]
	Thus $A$ is the root of the polynomial $f(t)=a_Nt^N+\cdots +a_1t+a_0$.
	\end{proof}
\item[12.35]Give an example of a quadratic form $q(x,y)$ such that $q(u)=0$ and $q(v)=0$ but $q(u+v)\neq 0$.\\
	Let $q(x,y)=x^2-y^2$ and $u=(1,1),v=(1,-1)$. Then $q(u)=1^2-1^2 = 0$ and $q(v)=1^2-(-1)^2=0$ but $q(u+v)=(1+1)^2-(1-1)^2=4-0$. Thus both $q(u)=0$ and $q(v)=0$ but $q(u+v)\neq 0$.
\item[12.37]Show that congruence of matrices is an equivalence relation.
	\begin{proof}
	Let $W\subset V$ be a subspace of $V$. Define a relation by $\vec{u}\equiv\vec{v}$ if $\vec{v}-\vec{u}\in W$. We prove that this is an equivalence relation on $V$.
	\begin{enumerate}
	\item[(i)]Show $A \equiv A$ $\forall A$.\\
		Let $\vec{u}\in V$. Notice that $\vec{u}-\vec{u}=\vec{0}\in W$. Thus $\vec{u}\equiv\vec{u}$.
	\item[(ii)]If $A \equiv B$ then $B \equiv A$.\\
		Assume that $\vec{v}\equiv\vec{u}$. Notice that $\vec{u}-\vec{u}=(\vec{v}-\vec{u})$. However $\vec{v}-\vec{u}\in W$ and $W$ is closed under negation. Thus $\vec{u}-\vec{u}\in W$ and so $\vec{v}\equiv\vec{u}$.
	\item[(iii)]If $A \equiv B$ and $B \equiv C$ then $A \equiv C$.\\
		Assume that $\vec{u}\equiv\vec{v}$ and $\vec{v}\equiv\vec{w}$ then $\vec{v}-\vec{u}\in W$ and $\vec{w}-\vec{v}\in W$. However $\vec{w}-\vec{u}=(\vec{w}-\vec{v})+(\vec{v}-\vec{u})$ and $W$ is closed under addition. Therefore $\vec{w}-\vec{u}\in W$ and so $\vec{u}\equiv\vec{w}$.
	\end{enumerate}
	\end{proof}


\item[12.38]Consider a real quadratic polynomial $q(x_1,...,x_n)=\sum_{i,j=1}^na_{i,j}x_ix_j$, where $a_{i,j}=a_{j,i}$.
	\begin{enumerate}
	\item If $a_{1,1}\neq 0$, show that the substitution
	\[ x_1=y_1-\dfrac{1}{a_{1,1}}(a_{1,2}y_2+\cdots+a_{1,n}y_n), x_2=y_2,...,x_n=y_n \]
	yields the equation $q(x_1,...,x_n)=a_{1,1}y_1^2+q^\prime(y_2,...,y_n)$, where $q^\prime$ is also a quadratic polynomial.
	\item If $a_{1,1}=0$ but, say, $a_{1,2}\neq 0$, show that the substitution
	\[ x_1=y_1+y_2,x_2=y_1-y_2,x_3=y_3,...,x_n=y_n \]
	yields the equation $q(x_1,...,x_n)=\sum b_{i,j}y_iy_j$, where $b_{1,1}\neq 0$, i.e., reduces this case to case (a).
	\end{enumerate}
	This method of diagonalizing $q$ is known as \textit{completing the square}.
\item[12.43]Determine whether or not each quadratic form is positive definite.
	\begin{enumerate}
	\item $q(x,y,z)=x^2+4xy+5y^2+6xz+2yz+4z^2$;\\
		This equation gives us the symmetric matrix
		\[ M= \begin{bmatrix}[rrr]1&2&3\\2&5&1\\3&1&4\\\end{bmatrix} \]
		This matrix results in a negative eigenvalue and so this matrix is not positive definite.
	\item $q(x,y,z)=x^2+2xy+2y^2+4xz+6yz+7z^2$;\\
		This equation gives us the symmetric matrix
		\[ M = \begin{bmatrix}[rrr]1&1&2\\1&2&3\\2&3&7\\\end{bmatrix} \]\\
		Since all eigenvalues of this matrix are $> 0$, this matrix is positive definite.
	\end{enumerate}
\end{enumerate}
\end{document}