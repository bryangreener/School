\documentclass[12pt]{article}
\usepackage[margin=1in]{geometry} 
\usepackage{amsmath,amsthm,amssymb,amsfonts}
\usepackage{tabto}
\usepackage{hyperref}

% Spacers:
% BEGIN BLOCK------------------------------------------
% END BLOCK============================================




\newcommand{\N}{\mathbb{N}}
\newcommand{\Z}{\mathbb{Z}}

% CUSTOM SETTINGS
% BEGIN BLOCK------------------------------------------
% For equation system alignment
\usepackage{systeme,mathtools}
% Usage:
%	\[
%	\sysdelim.\}\systeme{
%	3z +y = 10,
%	x + y +  z = 6,
%	3y - z = 13}


% For definitions
\newtheorem{defn}{Definition}[section]
\newtheorem{thrm}{Theorem}[section]

% For circled text
\usepackage{tikz}
\newcommand*\circled[1]{\tikz[baseline=(char.base)]{
            \node[shape=circle,draw,inner sep=0.8pt] (char) {#1};}}

\newenvironment{problem}[2][Problem]{\begin{trivlist}
\item[\hskip \labelsep {\bfseries #1}\hskip \labelsep {\bfseries #2.}]}{\end{trivlist}}
%If you want to title your bold things something different just make another thing exactly like this but replace "problem" with the name of the thing you want, like theorem or lemma or whatever
 
%used for matrix vertical line
\makeatletter
\renewcommand*\env@matrix[1][*\c@MaxMatrixCols c]{%
  \hskip -\arraycolsep
  \let\@ifnextchar\new@ifnextchar
  \array{#1}}
\makeatother 

% END BLOCK============================================

\newtheorem*{lemma}{Lemma} %added
\newtheorem*{result}{Result} %added
\newtheorem*{theorem}{Theorem} %added
\theoremstyle{definition}
\newtheorem*{solution}{Solution} %added
\theoremstyle{plain}

% HEADER
% BEGIN BLOCK------------------------------------------
\usepackage{fancyhdr}
 
\pagestyle{fancy}
\fancyhf{}
\lhead{Homework \#8}
\rhead{Bryan Greener}
\cfoot{\thepage}
% END BLOCK============================================

% TITLE
% BEGIN BLOCK------------------------------------------
\title{Bryan Greener}
\author{MATH 2300 CRN:15163}
\date{2018-02-14}
\begin{document}
\maketitle
% END BLOCK============================================

\TabPositions{4cm}

\begin{enumerate}
\item[6.22] The vectors $u_1=(1,-2)$ and $u_2=(4,-7)$ form a basis $S$ of $\mathbb{R}^2$. Find the coordinate vector $[v]$ of $v$ relative to $S$, where: (a) $v=(5,3)$; (b) $v=(1,1)$; (c) $v=(3,-6)$; (d) $v=(a,b)$.
	\begin{enumerate}
	\item 
		\begin{align*}
		\begin{bmatrix}[r]5\\3\\\end{bmatrix} &= x\begin{bmatrix}[r]1\\-2\\\end{bmatrix}+y\begin{bmatrix}[r]4\\-7\\\end{bmatrix}\\
		5&=x+4y\\
		3&=-2x-7y
		\end{align*}
		The solution to this system is $y=13,x=-47$. Thus $v=-47u_1+13u_2$ and so $[v]_S=[-47,13]$.
	\item
		\[ \sysdelim{.}{.}\systeme[xy]{1=x+4y,1=-2x-7y} \]
		The solution to this system is $y=3,x=-11$. Thus $v=-11u_1+3u_2$ and so $[v]_S=[-11,3]$.
	\item
		\[ \sysdelim{.}{.}\systeme[xy]{3=x+4y,-6=-2x-7y} \]
		The solution to this system is $y=0,x=3$. Thus $v=3u_1+0u_2$ and so $[v]_S=[3,0]$.
	\item
		\[ \sysdelim{.}{.}\systeme[xy]{a=x+4y,b=-2x-7y} \]
		The solution to this system is $y=b+2a, x=-7a-4b$. Thus $v=(-7a-4b)u_1+(b+2a)u_2$ and so $[v]_S=[b+2a,-7a-4b]$.
	\end{enumerate}
	
\item[6.24] The vectors $u_1=(1,1,1)$, $u_2=(1,1,0)$, and $u_3=(1,0,0)$ form a basis of $\mathbb{R}^3$. Find the coordinate vector $[v]$ of $v$ relative to $S$, where: (a) $v=(3,-4,2)$; (b) $v=(a,b,c)$.
	\begin{enumerate}
	\item
		\[ \sysdelim{.}{.}\systeme[xyz]{3=x+y+z, \-4=x+y,2=x} \]
		The solution to this system is $x=2,y=-6,z=7$. Thus $v=2u_1-6u_2+7u_3$ and so $[v]_S=[2,-6,7]$.
	\item
		\[ \sysdelim{.}{.}\systeme[xyz]{a=x+y+z,b=x+y,c=x} \]
		The solution to this system is $x=c, y=b-c, z=a-b$. Thus $v=(a-b)u_1+(b-c)u_2+(c)u_3$ and so $[v]_S=[c,b-c,a-b]$.
	\end{enumerate}
	
\item[6.25] Consider the basis $S=\{t^3+t^2, t^2+t, t+1, 1\}$ of the vector space $\mathbb{P}_3(t)$. Find the coordinate vector $[v]$ of $v$ relative to $S$, where: (a) $v=2t^3+t^2-3t+2$; (b) $v=t^2+2t-3$; (c) $v=at^3+bt^2+ct+d$.
	\begin{enumerate}
	\item
		\begin{align*}
		2t^3+t^2-3t+2 &= x(t^3+t^2)+y(t^2+t)+z(t+1)+s(1)\\
		&= xt^3+xt^2+yt^2+yt+zt+z+s\\
		&= xt^3+(x+y)t^2+(y+z)t+(z+s)\\
		2=x, \quad 1&=x+y, \quad -3=y+z, \quad 2=z+s
		\end{align*}
		The solution to this system is $x=2, y=-1, z=-2, s=4$. Thus $[v]=[2,-1,-2,4]$.
	\item
		Using the resulting equation $xt^3+(x+y)t^2+(y+z)t+(z+s)$ from the previous question, we get the system
		\[ \sysdelim{.}{.}\systeme[xyzs]{0=x,1=x+y,2=y+z, \-3=z+s} \]
		The solution to this system is $x=0, y=1, z=1, s=-4$. Thus $[v]=[0,1,1-4]$.
	\item
		Using the resulting equation $xt^3+(x+y)t^2+(y+z)t+(z+s)$ from the previous questions, we get the system
		\[ \sysdelim{.}{.}\systeme[xyzs]{a=x,b=x+y,c=y+z,d=z+s} \]
		The solution to this system is $x=a,y=b-a,z=c-(b-a),s=d-(c-(b-a))$. Thus $[v]=[a,b-a,c-b+a,d-c+b-a]$.
	\end{enumerate}
	
\item[6.26] In the vector space $M=M_{2,2}$ of 2 x 2 matrices find the coordinate vector $[A]$ of the matrix $A$ relative to the basis
\[ S=\{\begin{bmatrix}[rr]1&1\\1&1\\\end{bmatrix},\begin{bmatrix}[rr]1&-1\\1&0\\\end{bmatrix},\begin{bmatrix}[rr]1&1\\0&0\\\end{bmatrix},\begin{bmatrix}[rr]1&0\\0&0\\\end{bmatrix}\} \]
where (a) $A=\begin{bmatrix}[rr]3&-5\\6&7\\\end{bmatrix}$; (b) $A=\begin{bmatrix}[rr]a&b\\c&d\\\end{bmatrix}$.
	\begin{enumerate}
	\item
		\begin{align*}
		A&= \begin{bmatrix}[rr]3&-5\\6&7\\\end{bmatrix} = x\begin{bmatrix}[rr]1&1\\1&1\\\end{bmatrix}+y\begin{bmatrix}[rr]1&-1\\1&0\\\end{bmatrix}+z\begin{bmatrix}[rr]1&1\\0&0\\\end{bmatrix}+t\begin{bmatrix}[rr]1&0\\0&0\\\end{bmatrix}\\
		&=\begin{bmatrix}[rr]x+y+z+t & x-y+z\\ x+y & x\\\end{bmatrix}\\
		&x+y+z+t = 3, \quad x-y+z = -5, \quad x+y = 6, \quad x = 7
		\end{align*}
		The solution to this system is $x=7, y=-1, z=-13, t=10$. Thus $[A]=[7,-1,-13,10]$.
	\item
		Using the resulting matrix from the previous question gives us the following system
		\[ \sysdelim{.}{.}\systeme[xyzt]{a=x+y+z+t,b=x-y+z,c=x+y,d=x} \]
		The solution to this system is $x=d, y=c-d, z=(b-d)+(c-d), t= (a-d)-(c-d)-((b-d)+(c-d))$. Thus $[A]=[d,c-d,b+c-2d,a-b-2c+2d]$.
	\end{enumerate}
	
\item[6.28] In the space $M=M_{2,3}$ determine whether or not the following matrices are linearly dependent:
\[ A=\begin{bmatrix}[rrr]1&2&1\\3&1&2\\\end{bmatrix}, \quad B=\begin{bmatrix}[rrr]2&4&3\\7&5&6\\\end{bmatrix}, \quad C=\begin{bmatrix}[rrr]1&2&3\\5&7&6\\\end{bmatrix} \]
If they are linearly dependent, find the dimension and a basis of the subspace $W$ spanned by the matrices.
	\begin{solution}
	We form a matrix $M$ from the coordinate vectors of the given matrices as follows:
	\[ M=\begin{bmatrix}[rrrrrr]1&2&1&3&1&2\\2&4&3&7&5&6\\1&2&3&5&7&6\\\end{bmatrix} \xrightarrow[]{rref} \begin{bmatrix}[rrrrrr]1&2&0&2&-2&0\\0&0&1&1&3&2\\0&0&0&0&0&0\\\end{bmatrix}\]
	Thus dim$W=2$ and our basis is $\begin{bmatrix}[rrr]1&2&0\\2&-2&0\\\end{bmatrix},\begin{bmatrix}[rrr]0&0&1\\1&3&2\\\end{bmatrix}$.
	\end{solution}

\item[6.29] Consider the subspaces $U=\mathrm{span}(u_1,u_2,u_3)$ and $W=\mathrm{span}(w_1,w_2,w_3)$ of $\mathbb{P}_3(t)$, where:
\begin{align*}
&u_1=t^3+t^2+2t+2, \quad &&u_2=2t^3+3t^2+7t+5, \quad &&u_3=t^3+3t^2+8t+4\\
&w_1=t^3+t^2+t+2, \quad &&w_2=t^3+3t^2+7t+4, \quad &&w_3=2t^3+3t^2+6t+5
\end{align*}
Find: (a) dim$(U+W)$; (b) dim$(U \cap W)$.
	\begin{enumerate}
	\item $M_1=\begin{bmatrix}[rrrr]1&1&2&2\\2&3&7&5\\1&3&8&4\\\end{bmatrix} \xrightarrow[]{rref} \begin{bmatrix}[rrrr]1&0&-1&1\\0&1&3&1\\0&0&0&0\\\end{bmatrix}$\\
	$M_2=\begin{bmatrix}[rrrr]1&1&1&2\\1&3&7&4\\2&3&6&5\\\end{bmatrix} \xrightarrow[]{rref} \begin{bmatrix}[rrrr]1&0&0&1\\0&1&0&1\\0&0&1&0\\\end{bmatrix}$
	From these matrices, we get dim$U=2$ and dim$W=3$.
	Next, we form matrix $M$ by combining our resulting vectors $M_1$ and $M_2$.
	\[ M = \begin{bmatrix}[rrrr]1&0&-1&1\\0&1&3&1\\1&0&0&1\\0&1&0&1\\0&0&1&0\\\end{bmatrix} \xrightarrow[]{rref} \begin{bmatrix}[rrrr]1&0&0&1\\0&1&0&1\\0&0&1&0\\0&0&0&0\\0&0&0&0\\\end{bmatrix} \]
	Thus dim$(U+W)=3$.\\
	Therefore dim$(U \cap W) = \mathrm{dim}U+\mathrm{dim}W-\mathrm{dim}(U+W)=2+3-3=2$.
	\end{enumerate}

\item[6.30] Find the change-of-basis matrix $P$ from the usual basis $E$ of $\mathbb{R}^2$ to the basis $S$, the change-of-basis matrix $Q$ from $S$ back to $E$, and the coordinate vector $[v]$ of $v=(a,b)$ relative to $S$, where: (a) $S=\{(1,2),(3,5)\}$; (b) $S=\{(1,-3),(3,-8)\}$.
	\begin{enumerate}
	\item We start by writing the basis vectors in $S$ as columns.
	\[ P = \begin{bmatrix}[rr]1&3\\2&5\\\end{bmatrix} \]
	Next we find the inverse of $P$.
	\[ Q = P^{-1} = \begin{bmatrix}[rr]-5&3\\2&-1\\\end{bmatrix} \]
	Finally we use the fact that $[v]_S=P^{-1}[v]_E$ and $[v]_E=[a,b]^T$ to form the following:
	\[ [v]_S = P^{-1}[v]_E=\begin{bmatrix}[rr]-5&3\\2&-1\\\end{bmatrix}\begin{bmatrix}[r]a\\b\\\end{bmatrix}=\begin{bmatrix}[r]-5a+3b\\2a-b\\\end{bmatrix} \]
	\item We start by writing the basis vectors in $S$ as columns.
	\[ P = \begin{bmatrix}[rr]1&3\\-3&-8\\\end{bmatrix} \]
	Next we find the inverse of $P$.
	\[ Q = P^{-1} = \begin{bmatrix}[rr]-8&-3\\3&1\\\end{bmatrix} \]
	Finally we use the fact that $[v]_S=P^{-1}[v]_E$ and $[v]_E=[a,b]^T$ to form the following:
	\[ [v]_S = P^{-1}[v]_E=\begin{bmatrix}[rr]-8&-3\\3&1\\\end{bmatrix}\begin{bmatrix}[r]a\\b\\\end{bmatrix}=\begin{bmatrix}[r]-8a-3b\\3a+b\\\end{bmatrix} \]
	\end{enumerate}
	
\item[6.31] Consider the bases $S=\{(1,2),(2,3)\}$ and $S^\prime=\{(1,3),(1,4)\}$ of $\mathbb{R}^2$. Find: (a) the change-of-basis matrix $P$ from $S$ to $S^\prime$; (b) the change-of-basis matrix $Q$ from $S^\prime$ back to $S$.
	\begin{enumerate}
	\item Let $u_1,u_2$ be the basis vectors of $S$ and $v_1,v_2$ be the basis vectors of $S^\prime$. We write each of the basis vectors of $S^\prime$ as a linear combination of the original basis vectors of $S$. So
	\[ (1,3)=x(1,2)+y(2,3)\quad\Rightarrow\quad\sysdelim{.}{.}\systeme[xy]{x+2y=1,2x+3y=3}\quad\Rightarrow\quad x=3, y=-1 \]
	\[ (1,4)=x(1,2)+y(2,3)\quad\Rightarrow\quad\sysdelim{.}{.}\systeme[xy]{x+2y=1,2x+3y=4} \quad\Rightarrow\quad x=5,y=-2 \]
	Thus $v_1=3u_1-u_2$ and $v_2=5u_1-2u_2$. Writing coefficients of $u_1$ and $u_2$ as columns gives us the change-of-basis matrix $P$ from $S$ to $S^\prime$.
	\[ P = \begin{bmatrix}[rr]3&5\\-1&-2\\\end{bmatrix} \]
	\item Next to find the change-of-basis matrix $Q$ from $S^\prime$ to $S$, we use the fact that $Q=P^{-1}$.
	\[ Q = P^{-1} = \begin{bmatrix}[rr]2&5\\-1&-3\\\end{bmatrix} \]
	\end{enumerate}
	
\item[6.32] Suppose that the $x$ and $y$ axes in the plane $\mathbb{R}^2$ are rotated counterclockwise $30\deg$ to yield new $x^\prime$ and $y^\prime$ axes for the plane. Find: (a) the unit vectors in the direction of the new $x^\prime$ and $y^\prime$ axes; (b) the change-of-basis matrix $P$ for the new coordinate system; (c) the new coordinates of each of the following points under the new coordinate system: $A(1,3), B(2,-5), C(a,b)$.
	\begin{enumerate}
	\item 
	\[ \begin{bmatrix}[rr]\cos(30^{\circ} )&-\sin(30^{\circ})\\\sin(30^{\circ})&\cos(30^{\circ})\\\end{bmatrix} \]
	This matrix results in the two unit vectors $u_1=[\frac{\sqrt{3}}{2},\frac{1}{2}]$ and $u_2=[-\frac{1}{2},\frac{\sqrt{3}}{2}]$. Using this result, we get the change of basis matrix $P=\begin{bmatrix}[rr]\frac{\sqrt{3}}{2}&-\frac{1}{2}\\\frac{1}{2}&\frac{\sqrt{3}}{2}\\\end{bmatrix}$. To get the new coordinates we multiply $P^{-1}A$.
	\[ P^{-1}A = \begin{bmatrix}[rr]\frac{\sqrt{3}}{2}&\frac{1}{2}\\-\frac{1}{2}&\frac{2}{\sqrt{3}}\\\end{bmatrix}\begin{bmatrix}[r]1\\3\\\end{bmatrix} = [\frac{3+\sqrt{3}}{2}, -\frac{1+3\sqrt{3}}{2} \]
	\item 
	\[ P^{-1}B = P^{-1}\begin{bmatrix}[r]2\\-5\\\end{bmatrix} = [-\frac{5}{2}+\sqrt{3}, -1-\frac{5\sqrt{3}}{2}] \]
	\item
	\[ P^{-1}C = P^{-1}\begin{bmatrix}[r]a\\b\\\end{bmatrix} = [\frac{\sqrt{3}a}{2}+\frac{b}{2}, \frac{\sqrt{3}b}{2}-\frac{a}{2}] \]
	\end{enumerate}
	
\item[6.33] Find the change-of-basis matrix $P$ from the usual basis $E$ of $\mathbb{R}^3$ to the basis $S$, the change-of-basis matrix $Q$ from $S$ back to $E$ and the coordinate vector $[v]$ of $v=(a,b,c)$ relative to $S$, where $S$ consists of the vectors: (a) $u_1=(1,1,0)$, $u_2=(0,1,2)$, $u_3=(0,1,1)$; (b) $u_1=(1,0,1)$, $u_2=(1,1,2)$, $u_3=(1,2,4)$; (c) $u_1=(1,2,1)$, $u_2=(1,3,4)$, $u_3=(2,5,6)$.
	\begin{enumerate}
	\item We start by writing the basis vectors as columns.
	\[ P = \begin{bmatrix}[rrr]1&0&0\\1&1&1\\0&2&1\\\end{bmatrix} \]
	Next we find $Q$ by taking the inverse of matrix $P$.
	\[ Q = P^{-1} = \begin{bmatrix}[rrr]1&0&0\\1&-1&1\\-2&2&-1\\\end{bmatrix} \]
	Finally we use the fact that $[v]_S = P^{-1}[v]_E$ and $[v]_E=[a,b,c]^T$ to get the following:
	\[ [v]_S=P^{-1}[v]_E = \begin{bmatrix}[rrr]1&0&0\\1&-1&1\\-2&2&-1\\\end{bmatrix}\begin{bmatrix}[r]a\\b\\c\\\end{bmatrix} = \begin{bmatrix}[r]a\\a-b+c\\-2a+2b-c\\\end{bmatrix} \]
	\end{enumerate}
	
\item[6.34] Suppose $S_1$, $S_2$, and $S_3$ are bases of a vector space $V$. Suppose also that $P$ is the change-of-basis matrix from $S_1$ to $S_2$ and $Q$ is the change-of-basis matrix from $S_2$ to $S_3$. Prove that the product $PQ$ is the change-of-basis matrix from $S_1$ to $S_3$.
	\begin{proof}
	Assume to the contrary that if $P$ is a change-of-basis matrix from $S_1$ to $S_2$ and $Q$ is a change-of-basis matrix from $S_2$ to $S_3$, then $PQ$ is not a change-of-basis matrix from $S_1$ to $S_3$.\\
	Note that the change-of-basis matrix $Q$ is equal to the inverse of the change-of-basis matrix $P$ and multiplying $P$ by $Q$ gives us the identity matrix.
	\end{proof}

\item[8.53] Figure 8-9 is a diagram of maps:
\[ f: A\rightarrow B, g: B \rightarrow A, h: C \rightarrow B, F: B\rightarrow C, G:A\rightarrow C \]
Determine whether each of the following defines a mapping, and if it does, find its domain and codomain: (a) $g \circ f$; (b) $h \circ f$; (c) $F \circ f$; (d) $G \circ f$; (e) $g \circ h$; (f) $h \circ G \circ g$.
	\begin{enumerate}
	\item
	\item
	\item
	\item
	\item
	\item
	\end{enumerate}
\item[8.54.a] Let $f: \mathbb{R} \rightarrow \mathbb{R}$ and $g: \mathbb{R} \rightarrow \mathbb{R}$ be defined by $f(x)=x^2+3x+1$ and $g(x)=2x-3$. Find formulas defining the composition mapping $f \circ g$.
\item[8.55] For each of the following mappings $f: \mathbb{R} \rightarrow \mathbb{R}$ find a formula for the inverse mapping: (a) $f(x)=3x-7$; (b) $f(x)=x^3+2$.
	\begin{enumerate}
	\item
	\item
	\end{enumerate}
\item[8.57] Show that the following mappings are linear:
	\begin{enumerate}
	\item $F: \mathbb{R}^3\rightarrow\mathbb{R}^2$ defined by $F(x,y,z)=(x+2y-3z,4x-5y+6z)$.
	\item $F:\mathbb{R}^2\rightarrow\mathbb{R}^2$ defined by $F(x,y=(ax+by,cx+dy)$, where $a,b,c,d$ belong to $\mathbb{R}$.
	\end{enumerate}
\item[8.58.a] Show that the following mappings are not linear: (a) $F:\mathbb{R}^2\rightarrow\mathbb{R}^2$ defined by $F(x,y)=(x^2,y^2)$.
\item[8.59] Find $F(a,b)$, where the linear map $F:\mathbb{R}^2 \rightarrow \mathbb{R}^2$ is defined by $F(1,2)=(3,-1)$ and $F(0,1)=(2,1)$.




\end{enumerate}
\end{document}