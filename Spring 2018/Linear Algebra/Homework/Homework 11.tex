\documentclass[12pt]{article}
\usepackage[margin=1in]{geometry} 
\usepackage{amsmath,amsthm,amssymb,amsfonts}
\usepackage{tabto}
\usepackage{hyperref}

% Spacers:
% BEGIN BLOCK------------------------------------------
% END BLOCK============================================




\newcommand{\N}{\mathbb{N}}
\newcommand{\Z}{\mathbb{Z}}

% CUSTOM SETTINGS
% BEGIN BLOCK------------------------------------------
% For equation system alignment
\usepackage{systeme,mathtools}
% Usage:
%	\[
%	\sysdelim.\}\systeme{
%	3z +y = 10,
%	x + y +  z = 6,
%	3y - z = 13}


% For definitions
\newtheorem{defn}{Definition}[section]
\newtheorem{thrm}{Theorem}[section]

% For circled text
\usepackage{tikz}
\newcommand*\circled[1]{\tikz[baseline=(char.base)]{
            \node[shape=circle,draw,inner sep=0.8pt] (char) {#1};}}

\newenvironment{problem}[2][Problem]{\begin{trivlist}
\item[\hskip \labelsep {\bfseries #1}\hskip \labelsep {\bfseries #2.}]}{\end{trivlist}}
%If you want to title your bold things something different just make another thing exactly like this but replace "problem" with the name of the thing you want, like theorem or lemma or whatever
 
%used for matrix vertical line
\makeatletter
\renewcommand*\env@matrix[1][*\c@MaxMatrixCols c]{%
  \hskip -\arraycolsep
  \let\@ifnextchar\new@ifnextchar
  \array{#1}}
\makeatother 

% END BLOCK============================================

\newtheorem*{lemma}{Lemma} %added
\newtheorem*{result}{Result} %added
\newtheorem*{theorem}{Theorem} %added
\theoremstyle{definition}
\newtheorem*{solution}{Solution} %added
\theoremstyle{plain}

% HEADER
% BEGIN BLOCK------------------------------------------
\usepackage{fancyhdr}
 
\pagestyle{fancy}
\fancyhf{}
\lhead{Homework \#11}
\rhead{Bryan Greener}
\cfoot{\thepage}
% END BLOCK============================================

% TITLE
% BEGIN BLOCK------------------------------------------
\title{Bryan Greener}
\author{MATH 2300 CRN:15163}
\date{2018-03-19}
\begin{document}
\maketitle
% END BLOCK============================================

\TabPositions{4cm}

\begin{enumerate}
\item[2.73]Find the line $L$ of best fit for each set of points:
	\begin{enumerate}
	\item $(1,4),(2,6),(3,10),(4,15),(5,20)$
		\begin{align*}
		(1,4): 4&=a+1b\\
		(2,6): 6&=a+2b\\
		(3,10): 10&=a+3b\\
		(4,15): 15&=a+4b\\
		(5,20): 20&=a+5b
		\end{align*}
		\[ \begin{bmatrix}[rr]4&1\\6&2\\10&3\\15&4\\20&5\\\end{bmatrix}\begin{bmatrix}[r]a\\b\\\end{bmatrix}=\begin{bmatrix}[r]4\\6\\10\\15\\20\\\end{bmatrix} \]
		\begin{align*}
		M^TM&=\begin{bmatrix}[rrrrr]1&1&1&1&1\\1&2&3&4&5\\\end{bmatrix}\begin{bmatrix}[rr]1&1\\1&2\\1&3\\1&4\\1&5\\\end{bmatrix} = \begin{bmatrix}[rr]5&15\\15&55\\\end{bmatrix}\\
		M^T\vec{w}&=\begin{bmatrix}[rrrrr]1&1&1&1&1\\1&2&3&4&5\\\end{bmatrix}\begin{bmatrix}[r]4\\6\\10\\15\\20\\\end{bmatrix} = \begin{bmatrix}[r]55\\206\\\end{bmatrix}\\
		\vec{v}&\approx \begin{bmatrix}[rr]5&15\\15&55\\\end{bmatrix}^{-1}\begin{bmatrix}[r]55\\206\\\end{bmatrix} \approx \begin{bmatrix}[r]-1.3\\4.1\\\end{bmatrix}
		\end{align*}
		Thus our y-intercept is $\approx -1.3$ and our slope is $\approx 4.1$. Therefore the line of best fit is $y \approx 4.1x-1.3$.
	\end{enumerate}

\item[7.69]Let $U$ and $W$ be subspaces of a finite-dimensional inner product space $V$. Show that:
	\begin{enumerate}
	\item $(U+W)^\perp = U^\perp \cap W^\perp$
	\end{enumerate}

\item[7.70]Find the Fourier coefficient $c$ and the project $cw$ of $v$ along $w$, where:
	\begin{enumerate}
	\item[(b)] $v=(1,3,1,2)$ and $w=(1,-2,7,4)$ in $\mathbb{R}^4$.
		\[ <v,w>=1-6+7+8=10 \quad \mathrm{and} \quad <w,w>=1+4+49+16=70 \]
		So $c=\frac{10}{70}=\frac{1}{7}$ and $\mathrm{proj}(v,w)=cw=(\frac{1}{7},-\frac{2}{7},1,\frac{4}{7})$.
	\end{enumerate}

\item[7.71]Let $U$ be the subspace of $\mathbb{R}^4$ spanned by:
\[ v_1=(1,1,1,1), \quad v_2=(1,-1,2,2), \quad v_3=(1,2,-3,-4) \]
	\begin{enumerate}
	\item Apply the Gram-Schmidt algorithm to find an orthogonal and an orthonormal basis for $U$.
		\begin{align*}
		w_2&=v_2-\frac{<v_2,w_1>}{||w_1||^2}w_1 = (1,-1,2,2)-\frac{4}{4}(1,1,1,1) = (0,-2,1,1)\\
		w_3&=v_3-\frac{<v_3,w_1>}{||w_1||^2}w_1-\frac{<v_3,w_2>}{||w_2||^2}w_2\\
		&= (1,2,-3,-4)-\frac{-4}{4}(1,1,1,1)-\frac{-11}{6}(0,-2,1,1) = (2,-\frac{2}{3},-\frac{1}{6},-\frac{7}{6})
		\end{align*}
		Next we normalize the orthogonal basis: $||w_1||^2 = 4$, $||w_2||^2=6$, and $||w_3||^2=\frac{121}{6}$. Thus the following form an orthogonal basis of $U$:
		\[ u_1=\frac{1}{2}(1,1,1,1), \quad u_2=\frac{1}{\sqrt{6}}(0,-2,1,1), \quad \frac{\sqrt{6}}{11}(2,-\frac{2}{3},-\frac{1}{6},-\frac{7}{6}) \]
	\item Find the projection of $v=(1,2,-3,4)$ onto $U$.\\
	Since $U$ is orthogonal, we need to compute the Fourier coefficients.
	\begin{align*}
	c_1 &= \frac{<v,u_1>}{||u_1||^2} = 1\\
	c_2 &= \frac{<v,u_2>}{||u_2||^2} = -\frac{1}{2}\\
	c_3 &= \frac{<v,u_3>}{||u_3||^2} = -\frac{29}{210}\\
	u&=\mathrm{proj}(v,U)=c_1u_1+c_2u_2+c_3u_3 = 1(1,1,1,1)-\frac{1}{2}(0,-2,1,1)-\frac{29}{210}(12,-4,-1,-7)
	\end{align*}
	\end{enumerate}

\item[7.73]Consider the subspace $W=P_2(t)$ of $P(t)$ with inner product $<f,g>=\int_0^1f(t)g(t)dt$. Find the projection of $f(t)=t^3$ onto $W$. (Hint: Use the orthogonal polynomials $1,2t-1,6t^2-6t+1$ obtained in Problem 7.21.)

\item[7.76]Find a $3\times 3$ orthogonal matrix $P$ whose first two rows are multiples of $u=(1,1,1)$ and $v=(1,-2,3)$, respectively.

\item[7.81]Find the matrix $C$ which represents the usual inner product on $|mathbb{R}^3$ relative to the basis $S$ of $\mathbb{R}^3$ consisting of the vectors $u_1=(1,1,1),u_2=(1,2,1),$ and $u_3=(1,-1,3)$.

\item[7.85]Suppose $B$ is a real nonsingular matrix. Show that: (a) $B^TB$ is symmetric; (b) $B^TB$ is positive definite.
	\begin{enumerate}
	\item
	\item
	\end{enumerate}

\item[10.51]Evaluate (a) $\begin{vmatrix}2&6\\4&1\\\end{vmatrix}$; (b) $\begin{vmatrix}5&1\\3&-2\\\end{vmatrix}$; (c) $\begin{vmatrix}-2&8\\-5&-3\\\end{vmatrix}$; (d) $\begin{vmatrix}4&9\\1&-3\\\end{vmatrix}$; (e) $\begin{vmatrix}a+b&a\\b&a+b\\\end{vmatrix}$
	\begin{enumerate}
	\item
	\item
	\item
	\item[(e)]
	\end{enumerate}

\item[10.52]Find all $t$ such that: (a) $\begin{vmatrix}t-4&3\\2&t-9\\\end{vmatrix} = 0$; (b) $\begin{vmatrix}t-1&4\\3&t-2\\\end{vmatrix} = 0$.
	\begin{enumerate}
	\item
	\item
	\end{enumerate}

\item[10.53]Compute the determinant of each matrix:
	\begin{enumerate}
	\item $\begin{bmatrix}[rrr]2&1&1\\0&5&-2\\1&-3&4\\\end{bmatrix}$
	\item $\begin{bmatrix}[rrr]3&-2&-4\\2&5&-1\\0&6&1\\\end{bmatrix}$
	\item $\begin{bmatrix}[rrr]-2&-1&4\\6&-3&-2\\4&1&2\\\end{bmatrix}$
	\end{enumerate}

\item[10.54]Evaluate the determinant of each matrix: (a) $\begin{bmatrix}[rrrr]1&2&2&3\\1&0&-2&0\\3&-1&1&-2\\4&-3&0&2\\\end{bmatrix}$; (b) $\begin{bmatrix}[rrrr]2&1&3&2\\3&0&1&-2\\1&-1&4&3\\2&2&-1&1\\\end{bmatrix}$
	\begin{enumerate}
	\item
	\end{enumerate}

\item[two of 7.74 7.78 7.84(a)]
\end{enumerate}
\end{document}