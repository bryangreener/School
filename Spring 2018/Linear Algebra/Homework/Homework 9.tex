\documentclass[12pt]{article}
\usepackage[margin=1in]{geometry} 
\usepackage{amsmath,amsthm,amssymb,amsfonts}
\usepackage{tabto}
\usepackage{hyperref}

% Spacers:
% BEGIN BLOCK------------------------------------------
% END BLOCK============================================




\newcommand{\N}{\mathbb{N}}
\newcommand{\Z}{\mathbb{Z}}

% CUSTOM SETTINGS
% BEGIN BLOCK------------------------------------------
% For equation system alignment
\usepackage{systeme,mathtools}
% Usage:
%	\[
%	\sysdelim.\}\systeme{
%	3z +y = 10,
%	x + y +  z = 6,
%	3y - z = 13}


% For definitions
\newtheorem{defn}{Definition}[section]
\newtheorem{thrm}{Theorem}[section]

% For circled text
\usepackage{tikz}
\newcommand*\circled[1]{\tikz[baseline=(char.base)]{
            \node[shape=circle,draw,inner sep=0.8pt] (char) {#1};}}

\newenvironment{problem}[2][Problem]{\begin{trivlist}
\item[\hskip \labelsep {\bfseries #1}\hskip \labelsep {\bfseries #2.}]}{\end{trivlist}}
%If you want to title your bold things something different just make another thing exactly like this but replace "problem" with the name of the thing you want, like theorem or lemma or whatever
 
%used for matrix vertical line
\makeatletter
\renewcommand*\env@matrix[1][*\c@MaxMatrixCols c]{%
  \hskip -\arraycolsep
  \let\@ifnextchar\new@ifnextchar
  \array{#1}}
\makeatother 

% END BLOCK============================================

\newtheorem*{lemma}{Lemma} %added
\newtheorem*{result}{Result} %added
\newtheorem*{theorem}{Theorem} %added
\theoremstyle{definition}
\newtheorem*{solution}{Solution} %added
\theoremstyle{plain}

% HEADER
% BEGIN BLOCK------------------------------------------
\usepackage{fancyhdr}
 
\pagestyle{fancy}
\fancyhf{}
\lhead{Homework \#9}
\rhead{Bryan Greener}
\cfoot{\thepage}
% END BLOCK============================================

% TITLE
% BEGIN BLOCK------------------------------------------
\title{Bryan Greener}
\author{MATH 2300 CRN:15163}
\date{2018-02-25}
\begin{document}
\maketitle
% END BLOCK============================================

\TabPositions{4cm}

\begin{enumerate}
\item[8.70] For each of the following linear maps $F$ find a basis as well as the dimension of the kernel and the image of $F$:
	\begin{enumerate}
	\item $F:\mathbb{R}^3\rightarrow\mathbb{R}^3$ defined by $F(x,y,z)=(x+2y-3z,2x+5y-4z,x+4y+z)$.\\
	We start by finding the images of the usual basis of $\mathbb{R}^3$.
	\[ F(1,0,0)=(1,2,1) \quad F(0,1,0)=(2,5,4) \quad F(0,0,1)=(-3,-4,1) \]
	Next we put these row vectors into a matrix and find the rref.
	\[ \begin{bmatrix}[rrr]1&2&1\\2&5&4\\-3&-4&1\\\end{bmatrix} \xrightarrow[]{\mathrm{rref}} \begin{bmatrix}[rrr]1&0&-3\\0&1&2\\0&0&0\\\end{bmatrix} \]
	Thus $(1,0,-3),(0,1,2)$ form the basis for the image of $F$. Therefore $\mathrm{dim}(\mathrm{im}(F)) = 2$.\\
	Next we set $F(v)=0$ where $v=(x,y,z)$.
	\[ F(x,y,z)=(x+2y-3z,2x+5y-4z,x+4y+z)=(0,0,0) \]
	This gives us a homogeneous system whose solution space is $\mathrm{ker}(F)$.
	\[ \sysdelim{.}{.}\systeme[xyz]{x+2y-3z=0,2x+5y-4z=0,x+4y+z=0} \quad \mathrm{or} \quad \sysdelim{.}{.}\systeme[xyz]{x-7z=0,y+2z=0} \]
	Since our only free variable is $z$, $\mathrm{dim}(\mathrm{ker}(F)) = 1$. Let $z=1$ then $y=-2$ and $x=7$. Thus $(7,-2,1)$ forms a basis of $\mathrm{ker}(F)$.
	\item $F:\mathbb{R}^3\rightarrow\mathbb{R}^3$ defined by $F(x,y,z,t)=(x+2y+3z+2t,2x+4y+7z+5t,x+2y+6z+5t)$.\\
	We start by finding the images of the usual basis of $\mathbb{R}^4$.
	\[ F(1,0,0,0)=(1,2,1) \quad F(0,1,0,0)=(2,4,2) \quad F(0,0,1,0)=(3,7,6) \quad F(0,0,0,1)=(2,5,5) \]
	Next we put these row vectors into a matrix and find the rref.
	\[ \begin{bmatrix}[rrr]1&2&1\\2&4&2\\3&7&6\\2&5&5\\\end{bmatrix} \xrightarrow[]{\mathrm{rref}} \begin{bmatrix}[rrr]1&0&-5\\0&1&3\\0&0&0\\0&0&0\\\end{bmatrix} \]
	Thus $(1,0,-5),(0,1,3)$ form the basis for the image of $F$. Therefore $\mathrm{dim}(\mathrm{im}(F))=2$.\\
	Next we set $F(v)=0$ where $v=(x,y,z,t)$. This gives us a homogeneous system whose solution space is $\mathrm{ker}(F)$.
	\[ \sysdelim{.}{.}\systeme[xyzt]{x+2y+3z+2t=0,2x+4y+7z+5t=0,x+2y+6z+5t=0} \quad \mathrm{or} \quad \sysdelim{.}{.}\systeme[xyzt]{x+2y-t=0,z+t=0} \]
	Our free variables are $y$ and $t$ so $\mathrm{dim}(\mathrm{ker}(F))=2$. First let $y=1,t=0$, then $x=-2$ and $z=0$. Next let $y=0,t=1$, then $x=1$ and $z=-1$. Thus $\{(-2,1,0,0),(1,0,-1,1)\}$ forms a basis of $\mathrm{ker}(F)$.
	\end{enumerate}

\item[8.71.a] For each of the following linear maps $G$ find a basis as well as the dimension of the kernel and the image of $G$: $G:\mathbb{R}^3\rightarrow\mathbb{R}^2$ defined by $G(x,y,z)=(x+y+z,2x+2y+2z)$.\\
	We start by finding the images of the usual basis of $\mathbb{R}^3$.
	\[ G(1,0,0)=(1,2) \quad G(0,1,0)=(1,2) \quad G(0,0,1)=(1,2) \]
	We can already see that we have redundant rows so $(1,2)$ will form a basis for the image of $F$. Therefore $\mathrm{dim}(\mathrm{im}(F))=1$.\\
	Next we set $F(v)=0$ where $v=(x,y,z)$. This gives us a homogeneous system whose solution space is $\mathrm{ker}(F)$.
	\[ \sysdelim{.}{.}\systeme[xyz]{x+y+z=0,2x+2y+2z=0} \quad or \quad \sysdelim{.}{.}\systeme[xyz]{x+y+z=0} \]
	Our free variables are $y$ and $z$ so $\mathrm{dim}(\mathrm{ker}(F))=2$. First let $y=1,z=0$, then $z=-1$. This result is the same when we let $y=0,z=1$. Thus $\{(-1,1,0),(-1,0,1)\}$ forms a basis of $\mathrm{ker}(F)$.

\item[8.72.a] Each of the following matrices determines a linear map from $\mathbb{R}^4$ into $\mathbb{R}^3$:
	\[ A=\begin{bmatrix}[rrrr]1&2&0&1\\2&-1&2&-1\\1&-3&2&-2\\\end{bmatrix} \]
	Find a basis as well as the dimension of the kernel and the image of each linear map.\\
	\begin{solution}
	We start by finding the rref of the transpose of matrix $A$.
	\[ \mathrm{rref}A^T = \begin{bmatrix}[rrr]1&0&-1\\0&1&1\\0&0&0\\0&0&0\\\end{bmatrix} \]
	Thus $\{(1,0,-1),(0,1,1)\}$ forms a basis of $\mathrm{im}(A)$ and $\mathrm{dim}(\mathrm{im}(A))=2$.\\
	Next we need to reduce our original matrix to echelon form and solve the resulting system.
	\[ \mathrm{rref}(A)=\begin{bmatrix}[rrrr]1&2&0&1\\0&1&-\frac{2}{5}&\frac{3}{5}\\0&0&0&0\\\end{bmatrix} \quad or \quad \sysdelim{.}{.}\systeme[xyzt]{x+2y+t=0,5y-2z+3t=0} \]
	Our free variables are $z$ and $t$. Thus $\mathrm{dim}(\mathrm{ker}(A))=2$. Let $z=1,t=0$, then $x=-\frac{4}{5}$ and $y=\frac{2}{5}$. Next let $z=0,t=1$, then $x=\frac{1}{5}$ and $y=-\frac{3}{5}$.\\
	Therefore $\{(-4,2,1,0),(1,-3,0,1)\}$ forms a basis of $\mathrm{ker}(A)$.
	\end{solution}

\item[8.73] Find a linear mapping $F:\mathbb{R}^3\rightarrow\mathbb{R}^3$ for whose image is spanned by $(1,2,3)$ and $(4,5,6)$.
	\begin{solution}
	\[ \begin{bmatrix}[rr]1&4\\2&5\\3&6\\\end{bmatrix} \]
	\end{solution}

\item[8.75] Let $V=P_{10}(t)$, the vector space of polynomials of degree $\leq 10$. Consider the linear map $\mathbb{R}^4: V\rightarrow V$, where $\mathbb{D}^4$ denotes the fourth derivative of $\frac{d^4f}{dt^4}$. Find a basis and the dimension of: (a) the image of $\mathbb{D}^4$; (b) the kernel of $\mathbb{D}^4$.
	\begin{enumerate}
	\item $\mathrm{im}(D^4) = P_6(t)$ which gives us the basis $\{1,t,t^2,...,t^6\}$. Thus $\mathrm{dim}(\mathrm{im}(D^4))=7$.
	\item $\mathrm{ker}(D^4)=P_3(t)$ which gives us the basis $\{1,t,t^2,t^3\}$. Thus $\mathrm{dim}(\mathrm{ker}(D^4))=4$.
	\end{enumerate}		

\item[8.76] Suppose $F:V\rightarrow U$ is linear. Show that (a) the image of any subspace of $V$ is a subspace of $U$; (b) the preimage of any subspace of $U$ if a subspace of $V$.
	\begin{enumerate}
	\item
	\begin{proof}
	
	\end{proof}
	\item
	\end{enumerate}

\item[8.80] Let $H:\mathbb{R}^2\rightarrow\mathbb{R}^2$ be defined by $H(x,y)=(y,2x)$. Using the maps $F$ and $G$ in problem 8.79, find formulas defining the mappings: (a) $H\circ F$ and $H\circ G$; (b) $F\circ H$ and $G\circ H$; (c) $H\circ(F+G)$ and $H\circ F + H\circ G$.\\
Formulas from 8.79: $F(x,y,z)=(y,x+z)$, $G(x,y,z)=(2z,x-y)$
	\begin{enumerate}
	\item $[H \circ F](x,y,z)=H(F(x,y,z)) = H(y,x+z) = (x+z,2y)$\\
	$[H \circ G](x,y,z)=H(G(x,y,z)) = H(2z,x-y) = (x-y,4z)$
	\item This has no solution since $H$ is of lower dimension than $F$ and $G$.
	\item $[H \circ (F+G)](x,y,z) = H(F(x,y,z)) + H(F(x,y,z)) = H(y,x+z)+H(2z,x-y) = (x+z,2y)+(x-y,4z) = (2x-y+z,2y+4z)$
	\end{enumerate}
	
\item[8.85] Determine whether or not each linear map is nonsingular. If not, find a nonzero vector $v$ whose image is 0; otherwise find a formula for the inverse map:
	\begin{enumerate}
	\item $F:\mathbb{R}^3\rightarrow\mathbb{R}^3$ defined by $F(x,y,z)=(x+y+z,2x+3y+5z,x+3y+7z)$.\\
	\[ F = \begin{bmatrix}[rrr]1&1&1\\2&3&5\\1&3&7\\\end{bmatrix} \xrightarrow[]{\mathrm{rref}} \begin{bmatrix}[rrr]1&0&-2\\0&1&3\\0&0&0\\\end{bmatrix} \]
	Thus $\mathrm{rank}(F)=2$ and $\mathrm{dim}(\mathrm{kernel}(F))=1$.\\
	By setting our free variable $z=1$, we get the nullspace of $\left\{\begin{bmatrix}[r]2\\-3\\1\\\end{bmatrix}\right\}$.
	\item $G:\mathbb{R}^3\rightarrow\mathbb{P}_2(t)$ defined by $G(x,y,z)=(x+y)t^2+(x+2y+2z)t+y+z$.\\
	\[ G=\begin{bmatrix}[rrr]1&1&0\\1&2&2\\0&1&1\\\end{bmatrix} \xrightarrow[]{\mathrm{rref}} \begin{bmatrix}[rrr]1&0&0\\0&1&0\\0&0&1\\\end{bmatrix} \]
	Thus $\mathrm{rank}(G=3$ and $\mathrm{dim}(\mathrm{ker}(G))=0$ and our system is nonsingular so we must find the inverse.
	\[ G^{-1} = \begin{bmatrix}[rrr]0&1&-2\\1&-1&2\\-1&1&-1\\\end{bmatrix} \]
	We then rewrite this matrix in the original format.
	\[ G^{-1}(xt^2+yt+z)=(y-2z)t^2+(x-y+2z)t+(-x+y-z) \]
	\item $H:\mathbb{R}^2\rightarrow\mathbb{P}_2(t)$ defined by $H(x,y)=(x+2y)t^2+(x-y)t+x+y$.
	\[ H=\begin{bmatrix}[rr]1&2\\1&-1\\1&1\\\end{bmatrix} \xrightarrow[]{\mathrm{rref}} \begin{bmatrix}[rr]1&0\\0&1\\0&0\\\end{bmatrix} \]
	Thus $\mathrm{rank}(H)=2$ and $\mathrm{dim}(\mathrm{ker}(H)) = 0$ and our system is nonsingular but this matrix is not invertible.
	\end{enumerate}

\item[8.95] Suppose $F:V\rightarrow U$ is linear and $k$ is a nonzero scalar. Prove that the maps $F$ and $kF$ have the same kernel and the same image.
	\begin{proof}
	Since $F$ is a linear mapping then the image of $F$ is a subspace of $U$ and the kernel of $F$ is a subspace of $V$.\\
	Since $F(0)=0$ then $0\in \mathrm{im}(F)$. Let $k,a,b$ be scalars and $u,u^\prime \in \mathrm{im}(F)$. Then there exist vectors $\vec{v},\vec{v^\prime} \in V$ such that $F(\vec{v})=u$ and $F(\vec{v^\prime})=u^\prime$. Then
	\[ kF(a\vec{v}+b\vec{v^\prime}) = F(ka\vec{v}+kb\vec{v^\prime}) = kaF(\vec{v})+kbF(\vec{v^\prime}) = kau+kbu^\prime \in \mathrm{im}(F) \]
	Thus the image of $kF$ is a subspace of $U$.\\
	Since $F(0)=0$ then $0\in \mathrm{ker}(F)$. Let $k,a,b$ be scalars and $v,w\in\mathrm{ker}(F)$. Since $v$ and $w$ belong to the kernel of $F$, then $F(v)=0$ and $F(w)=0$. Thus
	\[ kF(av+bw)=kaF(v)+kbF(w)=ka0+kb0=0 \]
	and so $kav+kbw \in \mathrm{ker}(F)$. Therefore the kernel of $F$ is a subspace of $V$.\\
	Finally since $F$ is closed under scalar multiplication, then both $F$ and $kF$ have the same image and kernel.
	\end{proof}

\item[8.101] Let $v$ and $w$ be elements of a real vector space $V$. The line segment $L$ from $v$ to $v+w$ is defined to be the set of vectors $v+tw$ for $0\leq t \leq 1$.
	\begin{enumerate}
	\item Show that the line segment $L$ between vectors $v$ and $u$ consists of the points: (i) $(1-t)v+tu$ for $0 \leq t \leq 1$, and (ii) $t_1v+t_2u$ for $t_1+t_2=1, t_1\geq 0, t_2\geq 0$.
		\begin{enumerate}
		\item[(i)] $v+tw = (1-t)v+tu) \Rightarrow $
		\item[(ii)]
		\end{enumerate}
	\item Let $F:V\rightarrow U$ be linear. Show that the image $F(L)$ of a line segment $L$ in $V$ is a line segment in $U$.
	\end{enumerate}

\item[9.27] Let $F:\mathbb{R}^2\rightarrow\mathbb{R}^2$ be defined by $F(x,y)=(4x+5y,2x-y)$.
	\begin{enumerate}
	\item Find the matrix $A$ representing $F$ in the usual basis $E$.
	\item Find the matrix $B$ representing $F$ in the basis
		\[ S=\{u_1,u_2\}=\{(1,4),(2,9)\} \]
	\item Find $P$ such that $B=P^{-1}AP$.
	\item For $v=(a,b)$, find $[v]_S$ and $[F(v)]_S$. Verify that $[F]_S[v]_S=[F(v)]_S$.
	\end{enumerate}
	
	
	
	
\end{enumerate}

\end{document}