\documentclass{report}
\usepackage[margin=1in]{geometry} 
\usepackage{amsmath,amsthm,amssymb,amsfonts}
\usepackage{tabto}
\usepackage[yyyymmdd]{datetime}
\renewcommand{\dateseparator}{--}
\newcommand{\N}{\mathbb{N}}
\newcommand{\Z}{\mathbb{Z}}

% For definitions
%\newtheorem{defn}{Definition}[section]
%\newtheorem{thrm}{Theorem}[section]
%\newtheorem{ex}[Example}[section]
\newtheorem*{ex}{Example}
\newtheorem*{defn}{Definition}
\newtheorem*{thrm}{Theorem}
\newtheorem*{lemma}{Lemma}
\newtheorem*{result}{Result}

% For circled text
\usepackage{tikz}
\newcommand*\circled[1]{\tikz[baseline=(char.base)]{
            \node[shape=circle,draw,inner sep=0.8pt] (char) {#1};}}

\usepackage{pgf}
\usetikzlibrary{arrows, automata}

% For equation system alignment
\usepackage{systeme,mathtools}
% Usage:
%	\[
%	\sysdelim.\}\systeme{
%	3z +y = 10,
%	x + y +  z = 6,
%	3y - z = 13}

\newenvironment{problem}[2][Problem]{\begin{trivlist}
\item[\hskip \labelsep {\bfseries #1}\hskip \labelsep {\bfseries #2.}]}{\end{trivlist}}
%If you want to title your bold things something different just make another thing exactly like this but replace "problem" with the name of the thing you want, like theorem or lemma or whatever
 
%used for matrix vertical line
\makeatletter
\renewcommand*\env@matrix[1][*\c@MaxMatrixCols c]{%
  \hskip -\arraycolsep
  \let\@ifnextchar\new@ifnextchar
  \array{#1}}
\makeatother 
 
% Change chapter numbering
\newcommand{\mychapter}[2]{
	\setcounter{chapter}{#1}
	\setcounter{section}{0}
	\chapter*{#2}
	\addcontentsline{toc}{chapter}{#2}
}

\usepackage{graphicx}
\graphicspath{ {images/} }

\begin{document}
 
\tableofcontents{}
\mychapter{1}{2018-02-14}
\section{Unit Testing}
Qualities of unit tests:
\begin{enumerate}
\item Single function
\item Each test is independent and self contained
\item No output unless there is a failure
\item Check all return values and side effects. Side effects are things like changes in DB and changes to global variables and others. Asserts are used to check.
\end{enumerate}
\subsection{Coming up with Unit Tests}
Think of bad stuff that people can put in. Think of the limits of data input and output. Make sure if using RNG that a seed is set to verify the same results in multiple tests.\\

\subsection{FIRST Properties of Good Tests}
\begin{itemize}
\item {[}F{]}ast
\item {[}I{]}solated
\item {[}R{]}epeatable
\item {[}S{]}elf-validating
\item {[}T{]}imely
\end{itemize}


\end{document}