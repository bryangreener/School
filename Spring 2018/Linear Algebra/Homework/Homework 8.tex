\documentclass[12pt]{article}
\usepackage[margin=1in]{geometry} 
\usepackage{amsmath,amsthm,amssymb,amsfonts}
\usepackage{tabto}
\usepackage{hyperref}

% Spacers:
% BEGIN BLOCK------------------------------------------
% END BLOCK============================================




\newcommand{\N}{\mathbb{N}}
\newcommand{\Z}{\mathbb{Z}}

% CUSTOM SETTINGS
% BEGIN BLOCK------------------------------------------
% For equation system alignment
\usepackage{systeme,mathtools}
% Usage:
%	\[
%	\sysdelim.\}\systeme{
%	3z +y = 10,
%	x + y +  z = 6,
%	3y - z = 13}


% For definitions
\newtheorem{defn}{Definition}[section]
\newtheorem{thrm}{Theorem}[section]

% For circled text
\usepackage{tikz}
\newcommand*\circled[1]{\tikz[baseline=(char.base)]{
            \node[shape=circle,draw,inner sep=0.8pt] (char) {#1};}}

\newenvironment{problem}[2][Problem]{\begin{trivlist}
\item[\hskip \labelsep {\bfseries #1}\hskip \labelsep {\bfseries #2.}]}{\end{trivlist}}
%If you want to title your bold things something different just make another thing exactly like this but replace "problem" with the name of the thing you want, like theorem or lemma or whatever
 
%used for matrix vertical line
\makeatletter
\renewcommand*\env@matrix[1][*\c@MaxMatrixCols c]{%
  \hskip -\arraycolsep
  \let\@ifnextchar\new@ifnextchar
  \array{#1}}
\makeatother 

% END BLOCK============================================

\newtheorem*{lemma}{Lemma} %added
\newtheorem*{result}{Result} %added
\newtheorem*{theorem}{Theorem} %added


% HEADER
% BEGIN BLOCK------------------------------------------
\usepackage{fancyhdr}
 
\pagestyle{fancy}
\fancyhf{}
\lhead{Homework \#7}
\rhead{Bryan Greener}
\cfoot{\thepage}
% END BLOCK============================================

% TITLE
% BEGIN BLOCK------------------------------------------
\title{Bryan Greener}
\author{MATH 2300 CRN:15163}
\date{2018-02-14}
\begin{document}
\maketitle
% END BLOCK============================================

\TabPositions{4cm}

\begin{enumerate}
\item[6.22] The vectors $u_1=(1,-2)$ and $u_2=(4,-7)$ form a basis $S$ of $\mathbb{R}^2$. Find the coordinate vector $[v]$ of $v$ relative to $S$, where: (a) $v=(5,3)$; (b) $v=(1,1)$; (c) $v=(3,-6)$; (d) $v=(a,b)$.
	\begin{enumerate}
	\item
	\item
	\item
	\item
	\end{enumerate}
\item[6.24] The vectors $u_1=(1,1,1)$, $u_2=(1,1,0)$, and $u_3=(1,0,0)$ form a basis of $\mathbb{R}^3$. Find the coordinate vector $[v]$ of $v$ relative to $S$, where: (a) $v=(3,-4,2)$; (b) $v=(a,b,c)$.
	\begin{enumerate}
	\item
	\item
	\end{enumerate}
\item[6.25] Consider the basis $S=\{t^3+t^2, t^2+t, t+1, 1\}$ of the vector space $\mathbb{P}_3(t)$. Find the coordinate vector $[v]$ of $v$ relative to $S$, where: (a) $v=2t^3+t^2-3t+2$; (b) $v=t^2+2t-3$; (c) $v=at^3+bt^2+ct+d$.
	\begin{enumerate}
	\item
	\item
	\item
	\end{enumerate}
\item[6.26] In the vector space $M=M_{2,2}$ of 2 x 2 matrices find the coordinate vector $[A]$ of the matrix $A$ relative to the basis
\[ S=\{\begin{bmatrix}[rr]1&1\\1&1\\\end{bmatrix},\begin{bmatrix}[rr]1&-1\\1&0\\\end{bmatrix},\begin{bmatrix}[rr]1&1\\0&0\\\end{bmatrix},\begin{bmatrix}[rr]1&0\\0&0\\\end{bmatrix}\} \]
where (a) $A=\begin{bmatrix}[rr]3&-5\\6&7\\\end{bmatrix}$; (b) $A=\begin{bmatrix}[rr]a&b\\c&d\\\end{bmatrix}$.
	\begin{enumerate}
	\item
	\item
	\end{enumerate}
\item[6.28] In the space $M=M_{2,3}$ determine whether or not the following matrices are linearly dependent:
\[ A=\begin{bmatrix}[rrr]1&2&1\\3&1&2\\\end{bmatrix}, \quad B=\begin{bmatrix}[rrr]2&4&3\\7&5&6\\\end{bmatrix}, \quad C=\begin{bmatrix}[rrr]1&2&3\\5&7&6\\\end{bmatrix} \]
If they are linearly dependent, find the dimension and a basis of the subspace $W$ spanned by the matrices.

\item[6.29] Consider the subspaces $U=\mathrm{span}(u_1,u_2,u_3)$ and $W=\mathrm{span}(w_1,w_2,w_3)$ of $\mathbb{P}_3(t)$, where:
\begin{align*}
&u_1=t^3+t^2+2t+2, \quad &&u_2=2t^3+3t^2+7t+5, \quad &&u_3=t^3+3t^2+8t+4\\
&w_1=t^3+t^2+t+2, \quad &&w_2=t^3+3t^2+7t+4, \quad &&w_3=2t^3+3t^2+6t+5
\end{align*}
Find: (a) dim$(U+W)$; (b) dim$(U \cap W)$.

\item[6.30] Find the change-of-basis matrix $P$ from the usual basis $E$ of $\mathbb{R}^2$ to the basis $S$, the change-of-basis matrix $Q$ from $S$ back to $E$, and the coordinate vector $[v]$ of $v=(a,b)$ relative to $S$, where: (a) $S=\{(1,2),(3,5)\}$; (b) $S=\{(1,-3),(3,-8)\}$.
	\begin{enumerate}
	\item
	\item
	\end{enumerate}
\item[6.31] Consider the bases $S=\{(1,2),(2,3)\}$ and $S^\prime=\{(1,3),(1,4)\}$ of $\mathbb{R}^2$. Find: (a) the change-of-basis matrix $P$ from $S$ to $S^\prime$; (b) the change-of-basis matrix $Q$ from $S^\prime$ back to $S$.
	\begin{enumerate}
	\item
	\item
	\end{enumerate}
\item[6.32] Suppose that the $x$ and $y$ axes in the plane $\mathbb{R}^2$ are rotated counterclockwise $30\deg$ to yield new $x^\prime$ and $y^\prime$ axes for the plane. Find: (a) the unit vectors in the direction of the new $x^\prime$ and $y^\prime$ axes; (b) the change-of-basis matrix $P$ for the new coordinate system; (c) the new coordinates of each of the following points under the new coordinate system: $A(1,3), B(2,-5), C(a,b)$.
	\begin{enumerate}
	\item
	\item
	\item
	\end{enumerate}
\item[6.33] Find the change-of-basis matrix $P$ from the usual basis $E$ of $\mathbb{R}^3$ to the basis $S$, the change-of-basis matrix $Q$ from $S$ back to $E$ and the coordinate vector $[v]$ of $v=(a,b,c)$ relative to $S$, where $S$ consists of the vectors: (a) $u_1=(1,1,0)$, $u_2=(0,1,2)$, $u_3=(0,1,1)$; (b) $u_1=(1,0,1)$, $u_2=(1,1,2)$, $u_3=(1,2,4)$; (c) $u_1=(1,2,1)$, $u_2=(1,3,4)$, $u_3=(2,5,6)$.
	\begin{enumerate}
	\item
	\end{enumerate}
\item[6.34] Suppose $S_1$, $S_2$, and $S_3$ are bases of a vector space $V$. Suppose also that $P$ is the change-of-basis matrix from $S_1$ to $S_2$ and $Q$ is the change-of-basis matrix from $S_2$ to $S_3$. Prove that the product $PQ$ is the change-of-basis matrix from $S_1$ to $S_3$.
\item[8.53] Figure 8-9 is a diagram of maps:
\[ f: A\rightarrow B, g: B \rightarrow A, h: C \rightarrow B, F: B\rightarrow C, G:A\rightarrow C \]
Determine whether each of the following defines a mapping, and if it does, find its domain and codomain: (a) $g \circ f$; (b) $h \circ f$; (c) $F \circ f$; (d) $G \circ f$; (e) $g \circ h$; (f) $h \circ G \circ g$.
	\begin{enumerate}
	\item
	\item
	\item
	\item
	\item
	\item
	\end{enumerate}
\item[8.54.a] Let $f: \mathbb{R} \rightarrow \mathbb{R}$ and $g: \mathbb{R} \rightarrow \mathbb{R}$ be defined by $f(x)=x^2+3x+1$ and $g(x)=2x-3$. Find formulas defining the composition mapping $f \circ g$.
\item[8.55] For each of the following mappings $f: \mathbb{R} \rightarrow \mathbb{R}$ find a formula for the inverse mapping: (a) $f(x)=3x-7$; (b) $f(x)=x^3+2$.
	\begin{enumerate}
	\item
	\item
	\end{enumerate}
\item[8.57] Show that the following mappings are linear:
	\begin{enumerate}
	\item $F: \mathbb{R}^3\rightarrow\mathbb{R}^2$ defined by $F(x,y,z)=(x+2y-3z,4x-5y+6z)$.
	\item $F:\mathbb{R}^2\rightarrow\mathbb{R}^2$ defined by $F(x,y=(ax+by,cx+dy)$, where $a,b,c,d$ belong to $\mathbb{R}$.
	\end{enumerate}
\item[8.58.a] Show that the following mappings are not linear: (a) $F:\mathbb{R}^2\rightarrow\mathbb{R}^2$ defined by $F(x,y)=(x^2,y^2)$.
\item[8.59] Find $F(a,b)$, where the linear map $F:\mathbb{R}^2 \rightarrow \mathbb{R}^2$ is defined by $F(1,2)=(3,-1)$ and $F(0,1)=(2,1)$.




\end{enumerate}
\end{document}