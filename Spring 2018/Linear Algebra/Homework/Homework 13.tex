\documentclass[12pt]{article}
\usepackage[margin=1in]{geometry} 
\usepackage{amsmath,amsthm,amssymb,amsfonts}
\usepackage{tabto}
\usepackage{hyperref}

% Spacers:
% BEGIN BLOCK------------------------------------------
% END BLOCK============================================




\newcommand{\N}{\mathbb{N}}
\newcommand{\Z}{\mathbb{Z}}

% CUSTOM SETTINGS
% BEGIN BLOCK------------------------------------------
% For equation system alignment
\usepackage{systeme,mathtools}
% Usage:
%	\[
%	\sysdelim.\}\systeme{
%	3z +y = 10,
%	x + y +  z = 6,
%	3y - z = 13}


% For definitions
\newtheorem{defn}{Definition}[section]
\newtheorem{thrm}{Theorem}[section]

% For circled text
\usepackage{tikz}
\newcommand*\circled[1]{\tikz[baseline=(char.base)]{
            \node[shape=circle,draw,inner sep=0.8pt] (char) {#1};}}

\newenvironment{problem}[2][Problem]{\begin{trivlist}
\item[\hskip \labelsep {\bfseries #1}\hskip \labelsep {\bfseries #2.}]}{\end{trivlist}}
%If you want to title your bold things something different just make another thing exactly like this but replace "problem" with the name of the thing you want, like theorem or lemma or whatever
 
%used for matrix vertical line
\makeatletter
\renewcommand*\env@matrix[1][*\c@MaxMatrixCols c]{%
  \hskip -\arraycolsep
  \let\@ifnextchar\new@ifnextchar
  \array{#1}}
\makeatother 

% END BLOCK============================================

\newtheorem*{lemma}{Lemma} %added
\newtheorem*{result}{Result} %added
\newtheorem*{theorem}{Theorem} %added
\theoremstyle{definition}
\newtheorem*{solution}{Solution} %added
\theoremstyle{plain}

% HEADER
% BEGIN BLOCK------------------------------------------
\usepackage{fancyhdr}
 
\pagestyle{fancy}
\fancyhf{}
\lhead{Homework \#13}
\rhead{Bryan Greener}
\cfoot{\thepage}
% END BLOCK============================================

% TITLE
% BEGIN BLOCK------------------------------------------
\title{Bryan Greener}
\author{MATH 2300 CRN:15163}
\date{2018-04-01}
\begin{document}
\maketitle
% END BLOCK============================================

\TabPositions{4cm}

\begin{enumerate}
\item[11.53]Let $A=\begin{bmatrix}[rr]2&-3\\5&1\\\end{bmatrix}$ and $B=\begin{bmatrix}[rr]1&2\\0&3\\\end{bmatrix}$. Find $f(A)$, $g(A)$, $f(B)$, and $g(B)$ where $f(t)=2t^2-5t+6$ and $g(t)=t^3-2t^2+t+3$.
	\begin{enumerate}
	\item[f(A)]$=2\begin{bmatrix}[rr]2&-3\\5&1\\\end{bmatrix}^2-5\begin{bmatrix}[rr]2&-3\\5&1\\\end{bmatrix}+6\begin{bmatrix}[rr]1&0\\0&1\\\end{bmatrix} = \begin{bmatrix}[rr]-26&-3\\5&-27\\\end{bmatrix}$
	\item[g(A)]$=\begin{bmatrix}[rr]1&2\\0&3\\\end{bmatrix}^3-2\begin{bmatrix}[rr]1&2\\0&3\\\end{bmatrix}^2+\begin{bmatrix}[rr]1&2\\0&3\\\end{bmatrix}+3\begin{bmatrix}[rr]1&0\\0&1\\\end{bmatrix} = \begin{bmatrix}[rr]-40&39\\-65&-27\\\end{bmatrix}$
	\item[f(B)]$=\begin{bmatrix}[rr]3&6\\0&9\\\end{bmatrix}$
	\item[g(B)]$=\begin{bmatrix}[rr]3&12\\0&15\\\end{bmatrix}$
	\end{enumerate}
\item[11.54]Let $B=\begin{bmatrix}[rrr]8&12&0\\0&8&12\\0&0&8\\\end{bmatrix}$. Find a real matrix $A$ such that $B=A^3$.\\
	Let $B = \begin{bmatrix}[rrr]a&b&c\\d&e&f\\g&h&i\\\end{bmatrix}^3$.
	Expanding the cubed matrix then solving the system gives us the following matrix for $A$: $\begin{bmatrix}[rrr]2&1&-\frac{1}{2}\\0&2&1\\0&0&2\\\end{bmatrix}$

\item[11.55]For each matrix, find a polynomial having the matrix as a root:
	\begin{enumerate}
	\item $A=\begin{bmatrix}[rr]2&5\\1&-3\\\end{bmatrix}$\\
	First we find the trace of $A$. $\mathrm{tr}(A)=-1$. Next we find the determinant of $A$. $\mathrm{det}(A)=-11$. Next we plug this into $\Delta (t) = t^2+t-11$.
	\end{enumerate}
	
\item[11.56]Let $A$ be any square matrix and let $f(t)$ be any polynomial. Prove:
	\begin{enumerate}
	\item[(b)] $f(P^{-1}AP)=P^{-1}f(A)P$
	\end{enumerate}
			
\item[11.58]Let $A=\begin{bmatrix}[rr]2&-1\\-2&3\\\end{bmatrix}$
	\begin{enumerate}
	\item Find all eigenvalues and corresponding linearly independent eigenvectors.
	\[ \chi_A (\lambda) = \mathrm{det}(\lambda I_2-A) = \mathrm{det}\begin{bmatrix}[rr]\lambda - 2&-(-1)\\-(-2) & \lambda - 3\\\end{bmatrix} = (\lambda - 2)(\lambda - 3) - 2 = \lambda^2-5\lambda + 4=(\lambda - 1)(\lambda - 4) \]
	So our eigenvalues are 1 and 4.\\
	Let $\lambda=1$
	\[ \lambda I_2 - A = \begin{bmatrix}[rr]-1&1\\2&-2\\\end{bmatrix} \Rightarrow v_1=\begin{bmatrix}[r]1\\1\\\end{bmatrix} \]
	Let $\lambda=4$
	\[ \lambda I_2 - A=\begin{bmatrix}[rr]2&1\\2&1\\\end{bmatrix} \Rightarrow v_4=\begin{bmatrix}[r]1\\-2\\\end{bmatrix} \]
	\item Find a nonsingular matrix $P$ such that $D=P^{-1}AP$ is diagonal. \\
	Let $P=\begin{bmatrix}[rr]v_1&v_4\\\end{bmatrix}=\begin{bmatrix}[rr]1&1\\1&-2\\\end{bmatrix}$.\\
	Notice $P^{-1}AP=-\frac{1}{3}\begin{bmatrix}[rr]-2&-1\\-1&1\\\end{bmatrix}\begin{bmatrix}[rr]2&-1\\-2&3\\\end{bmatrix}\begin{bmatrix}[rr]1&1\\1&-2\\\end{bmatrix}=\begin{bmatrix}[rr]1&0\\0&4\\\end{bmatrix}$
	\item Find $A^8$ and $f(A)$ where $f(t)=t^4-5t^3+7t^2-2t+5$.\\
	From $P^{-1}AP=D$, we get $AP=PD$ and so $A=PDP^{-1}$. Then $A^8=(PDP^{-1})^8$. Expanding this formula, we can cancel all inner $P$ and $P^{-1}$ since multiplying those two together gives the identity matrix. Thus
	\[ A^8=PD^8P^{-1}=\begin{bmatrix}[rr]1&1\\1&-2\\\end{bmatrix}\begin{bmatrix}[rr]1&0\\0&4^8\\\end{bmatrix}\begin{bmatrix}[rr]2&1\\1&-1\\\end{bmatrix}\frac{1}{3} \]
	\[ f(A) = \begin{bmatrix}[rr]2&-1\\-2&3\\\end{bmatrix}^4-5\begin{bmatrix}[rr]2&-1\\-2&3\\\end{bmatrix}^3+7\begin{bmatrix}[rr]2&-1\\-2&3\\\end{bmatrix}^2-2\begin{bmatrix}[rr]2&-1\\-2&3\\\end{bmatrix}+5\begin{bmatrix}[rr]1&0\\0&1\\\end{bmatrix} \]
	\[ = \begin{bmatrix}[rr]2&-6\\2&9\\\end{bmatrix} \]
	\item Find a matrix $B$ such that $B^2=A$.\\
	Let $B=\begin{bmatrix}[rr]a&b\\c&d\\\end{bmatrix}$. Then expanding $B^2$ and setting it equal to matrix $A$, then solving the system gives us the matrix $B=\begin{bmatrix}[rr]\frac{4}{3}&-\frac{1}{3}\\-\frac{2}{3}&\frac{5}{3}\\\end{bmatrix}$.
	
	\end{enumerate}
	
\item[11.60]For each matrix, find all eigenvalues and maximum set $S$ of linearly independent eigenvectors:
	\[ A=\begin{bmatrix}[rrr]1&-3&3\\3&-5&3\\6&-6&4\\\end{bmatrix} \quad B=\begin{bmatrix}[rrr]3&-1&1\\7&-5&1\\6&-6&4\\\end{bmatrix} \quad  C=\begin{bmatrix}[rrr]1&2&2\\1&2&-1\\-1&1&4\\\end{bmatrix} \]
	Which matrices can be diagonalized and why?
	\begin{enumerate}
	\item $\chi_A (\lambda) = \mathrm{det}(\lambda I_3 - A) = \mathrm{det}\begin{bmatrix}[rrr]\lambda-1&3&-3\\-3&\lambda+5 & -3\\-6&6&\lambda - 4\\\end{bmatrix} = (\lambda+2)(\lambda-4)(\lambda+2)$\\
	Thus our eigenvalues are 4 and -2 with a multiplicity of 2.\\
	Let $\lambda=4$
	\[ \lambda I_3-A = \begin{bmatrix}[rrr]3&3&-3\\-3&9&-3\\-6&6&0\\\end{bmatrix} \]
	\[ v_4 = \begin{bmatrix}[r]\frac{1}{2}\\\frac{1}{2}\\1\\\end{bmatrix} \]
	Let $\lambda=-2$	
	\[ \lambda I_3 - A = \begin{bmatrix}[rrr]-3&3&-3\\-3&3&-3\\-6&6&-6\\\end{bmatrix} \]
	\[ u_{2} = \begin{bmatrix}[r]1\\1\\0\\\end{bmatrix}, v_2=\begin{bmatrix}[r]1\\0\\-1\\\end{bmatrix} \]
	\item We repeat the steps above to find our eigenvalues $\chi_A(\lambda)=(\lambda-4)(\lambda-2)(\lambda+4)$. Thus our eigenvalues are 4, -4, and 2.
	\item 
	\end{enumerate}
\item[11.62]Let $A=\begin{bmatrix}[rr]a&b\\c&d\\\end{bmatrix}$ be a real matrix. Find necessary and sufficient conditions on $a,b,c,d$ so that $A$ is diagonalizable, that is, that $A$ has two (real) linearly independent eigenvectors.
\item[11.63]Show that matrices $A$ and $A^T$ have the same eigenvalues. Give an example of a $2 \times 2$ matrix $A$ where $A$ and $A^T$ have different eigenvectors.
\item[11.68]For each symmetric matrix $B$, find its eigenvalues, a maximal orthogonal set $S$ of eigenvectors, and an orthogonal matrix $P$ such that $D=P^{-1}BP$ is diagonal:
	\begin{enumerate}
	\item $B=\begin{bmatrix}[rrr]0&1&1\\1&0&1\\1&1&0\\\end{bmatrix}$
	\item $B=\begin{bmatrix}[rrr]2&2&4\\2&5&8\\4&8&17\\\end{bmatrix}$
	\end{enumerate}
\item[11.69]Find a real $2 \times 2$ symmetrix matrix $A$ with eigenvalues $\lambda = 1$ and $\lambda = 4$ such that $u=(1,1)$ is an eigenvector belonging to $\lambda=1$. Also find a matrix $B$ for which $B^2=A$.
\item ONE OF 11.65 11.66
\item[11.65]Suppose $v$ is an eigenvector of a linear operator $T$ belonging to the eigenvalue $\lambda$. Prove:
	\begin{enumerate}
	\item For $n>0$, $v$ is an eigenvector of $T^n$ belonging to $\lambda^n$.
	\item $f(\lambda)$ is an eigenvalue of $f(T)$ for any polynomial $f(t)$.
	\end{enumerate}
\item[11.66]Let $E:V\rightarrow V$ be a projection mapping, that is $E^2=E$. Show that $E$ is diagonalizable and, in fact, can be represented by the diagonal matrix $M=\begin{bmatrix}[rr]I_r&0\\0&0\\\end{bmatrix}$, where $r$ is the rank of $E$.
	
\end{enumerate}	
\end{document}