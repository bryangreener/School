\documentclass[12pt]{article}
\usepackage[margin=1in]{geometry} 
\usepackage{amsmath,amsthm,amssymb,amsfonts}
\usepackage{tabto}

% Spacers:
% BEGIN BLOCK------------------------------------------
% END BLOCK============================================




\newcommand{\N}{\mathbb{N}}
\newcommand{\Z}{\mathbb{Z}}

% CUSTOM SETTINGS
% BEGIN BLOCK------------------------------------------
% For equation system alignment
\usepackage{systeme,mathtools}
% Usage:
%	\[
%	\sysdelim.\}\systeme{
%	3z +y = 10,
%	x + y +  z = 6,
%	3y - z = 13}


% For definitions
\newtheorem{defn}{Definition}[section]
\newtheorem{thrm}{Theorem}[section]

% For circled text
\usepackage{tikz}
\newcommand*\circled[1]{\tikz[baseline=(char.base)]{
            \node[shape=circle,draw,inner sep=0.8pt] (char) {#1};}}

\newenvironment{problem}[2][Problem]{\begin{trivlist}
\item[\hskip \labelsep {\bfseries #1}\hskip \labelsep {\bfseries #2.}]}{\end{trivlist}}
%If you want to title your bold things something different just make another thing exactly like this but replace "problem" with the name of the thing you want, like theorem or lemma or whatever
 
%used for matrix vertical line
\makeatletter
\renewcommand*\env@matrix[1][*\c@MaxMatrixCols c]{%
  \hskip -\arraycolsep
  \let\@ifnextchar\new@ifnextchar
  \array{#1}}
\makeatother 

% END BLOCK============================================

\newtheorem*{lemma}{Lemma} %added
\newtheorem*{result}{Result} %added
\newtheorem*{theorem}{Theorem} %added


% HEADER
% BEGIN BLOCK------------------------------------------
\usepackage{fancyhdr}
 
\pagestyle{fancy}
\fancyhf{}
\lhead{Homework \#5}
\rhead{Bryan Greener}
\cfoot{\thepage}
% END BLOCK============================================

% TITLE
% BEGIN BLOCK------------------------------------------
\title{Bryan Greener}
\author{MATH 2300 CRN:15163}
\date{2018-01-28}
\begin{document}
\maketitle
% END BLOCK============================================

\TabPositions{4cm}

\begin{enumerate}
\item[4.36.] Let $V$ be the set of ordered pairs $(a,b)$ of real numbers with addition in $V$ and scalar multiplication on $V$ defined by
			\[ (a,b) + (c,d) = (a+c,b+d) \quad k(a,b) = (ka,0) \]
			Show that $V$ satisfies all the axioms of a vector space except $[M_4]$, that is, except $1u=u$. Hence $[M_4]$ is not a consequence of the other axioms.\\~\\
			Let $u=(a,b)$. Then
			\[ 1u = 1(a,b) = (1a, 0) \neq (a,b) = u \]
			Thus since $1u \neq u$, axiom $M_4$ fails and this is not a vector space.
			
\item[4.38.] Let $V$ be the set of ordered pairs $(a,b)$ of real numbers. Show that $V$ is not a vector space over $\mathbb{R}$ with addition in $V$ and scalar multiplication on $V$ defined by
	\begin{enumerate}
	% 4.38.i
	\item[(i)] $(a,b)+(c,d)=(a+d,b+c)$ and $k(a,b)=(ka,kb)$\\
		Let $r=1,s=2,v=(3,4)$. Then
		\begin{align*}
		(r+s)v &= (1+2)(3,4) = 3(3,4) = (9,12)\\
		rv + sv &= 1(3,4) + 2(3,4) = (3,4) + (6,8) = (3+8,4+3) = (11,7)
		\end{align*}
		Since $(r+s)v \neq rv + sv$, axiom $[M_2]$ does not hold.\\
		
	% 4.38.ii
	\item[(ii)] $(a,b)+(c,d)=(a+c,b+d)$ and $k(a,b)=(a,b)$\\
		Let $r=1,s=2,v=(3,4)$. Then
		\begin{align*}
		(r+s)v &= 3(3,4) = (3,4)\\
		rv + sv &= 1(3,4) + 2(3,4) = (3,4) + (3,4) = (3+3,4+4) = (6,8)
		\end{align*}
		Since $(r+s)v \neq rv + sv$, axiom $[M_2]$ does not hold.\\
		
	% 4.38.iii
	\item[(iii)] $(a,b)+(c,d)=(0,0)$ and $k(a,b)=(ka,kb)$\\
		Let $r=1,s=2,v=(3,4)$. Then
		\begin{align*}
		(r+s)v &= 3(3,4) = (9,12)\\
		rv + sv &= 1(3,4) + 2(3,4) = (3,4) + (6,8) = (0,0)
		\end{align*}
		Since $(r+s)v \neq rv + sv$, axiom $[M_2]$ does not hold.\\
	
	% 4.38.iv
	\item[(iv)] $(a,b)+(c,d)=(ac,bd)$ and $k(a,b)=(ka,kb)$
	\end{enumerate}
		Let $r=1,s=2,v=(3,4)$. Then
		\begin{align*}
		(r+s)v &= 3(3,4) = (9,12)\\
		rv + sv &= 1(3,4) + 2(3,4) = (3,4) + (6,8) = (3*6,4*8) = (18,32)
		\end{align*}
		Since $(r+s)v \neq rv + sv$, axiom $[M_2]$ does not hold.\\
	
\item[4.41.] Consider the vectors $u=(1,2,3)$ and $v=(2,3,1)$ in $\mathbb{R}^3$.
	\begin{enumerate}
	% 4.41.a
	\item Write $w=(1,3,8)$ as a linear combination of $u$ and $v$.\\
		\begin{align*}
		\begin{bmatrix}[r]1\\3\\8 \end{bmatrix} 
		= x \begin{bmatrix}[r] 1\\2\\3\\ \end{bmatrix}
		+ y \begin{bmatrix}[r] 2\\3\\1\\ \end{bmatrix}
		= \begin{bmatrix}[r] x+2y\\ 2x+3y\\ 3x+y\\ \end{bmatrix}
		\end{align*}
		This can be rewritten as
		\begin{align*}
		\sysdelim{.}{.}\systeme[xy]{x+2y=1,2x+3y=3,3x+y=8}\\
		x=3, \quad y=-1
		\end{align*}
		Thus $w=3u-v$.\\

	% 4.41.b
	\item Write $w=(2,4,5)$ as a linear combination of $u$ and $v$.\\
		\begin{align*}
		\begin{bmatrix}[r]2\\4\\5\\ \end{bmatrix}
		= x \begin{bmatrix}[r] 1\\2\\3\\ \end{bmatrix}
		+ y \begin{bmatrix}[r] 2\\3\\1\\ \end{bmatrix}
		= \begin{bmatrix}[r] x+2y\\ 2x+3y\\ 3x+y\\ \end{bmatrix}
		\end{align*}
		This can be rewritten as
		\begin{align*}
		\sysdelim{.}{.}\systeme[xy]{x+2y=2,2x+3y=4,3x+y=5}\\
		\begin{bmatrix}[rr|r] 1&2&2\\2&3&4\\3&1&5\\ \end{bmatrix}
		= \begin{bmatrix}[rr|r] 1&2&2\\1&1&2\\1&-2&1\\ \end{bmatrix}
		= \begin{bmatrix}[rr|r] 1&-2&1\\1&1&2\\1&2&2\\ \end{bmatrix}
		= \begin{bmatrix}[rr|r] 1&-2&2\\1&1&2\\0&0&1\\ \end{bmatrix}
		\end{align*}
		This system has no solutions.\\

	% 4.41.c
	\item Find $k$ so that $(1,k,-2)$ is a linear combination of $u$ and $v$.
		\begin{align*}
		\begin{bmatrix}[r]1\\k\\-2\\ \end{bmatrix}
		= x \begin{bmatrix}[r] 1\\2\\3\\ \end{bmatrix}
		+ y \begin{bmatrix}[r] 2\\3\\1\\ \end{bmatrix}
		= \begin{bmatrix}[r] x+2y\\ 2x+3y\\ 3x+y\\ \end{bmatrix}
		\end{align*}
		This can be rewritten as
		\[ \sysdelim{.}{.}\systeme[xy]{x+2y=1,2x+3y=k,3x+y=-2} \]
		So $y=-2-3x$. Plugging this into the first equations gives us $x=-1$ and so $y=1$. Plugging both $x$ and $y$ into the second equation results in $k=1$.		
		
	% 4.41.d
	\item Find a condition on $a$, $b$, and $c$ so that $(a,b,c)$ is a linear combination of $u$ and $v$.
		\begin{align*}
		\begin{bmatrix}[r]a\\b\\c\\ \end{bmatrix}
		= x \begin{bmatrix}[r] 1\\2\\3\\ \end{bmatrix}
		+ y \begin{bmatrix}[r] 2\\3\\1\\ \end{bmatrix}
		= \begin{bmatrix}[r] x+2y\\ 2x+3y\\ 3x+y\\ \end{bmatrix}
		\end{align*}
		This can be rewritten as
		\[ \sysdelim{.}{.}\systeme[xy]{x+2y=a,2x+3y=b,3x+y=c} \]
		
		
	\end{enumerate}
	
\item[4.43.] Write $u$ as a linear combination of the polynomials $v=2t^2+3t-4$ and $w=t^2-2t-3$, where:
	\begin{enumerate}
	% 4.43.a
	\item $u=3t^2+8t-5$

	% 4.43.b
	\item $u=4t^2-6t-1$
	\end{enumerate}
	
\item[4.45.] Determine whether or not $W$ is a subspace of $\mathbb{R}^3$, where $W$ consists of all vectors $(a,b,c)$ in $\mathbb{R}^3$ such that:
	\begin{enumerate}
	% 4.45.a
	\item $a=3b$

	% 4.45.b
	\item $a \leq b \leq c$

	% 4.45.c
	\item $ab=0$
	\end{enumerate}

\item[4.49.] Suppose $U$ and $W$ are subspaces of $V$ for which $U \cup W$ is a subspace. Show that $U \subseteq W$ or $W \subseteq U$.

\item[4.51.] Show that the vectors $u_1,u_2,u_3$ span $\mathbb{R}^3$ [that is, show that any vector $v=(a,b,c)$ in $\mathbb{R}^3$ is a linear combination of $u_1,u_2,u_3$] where:
	\begin{enumerate}
	% 4.51.a	
	\item $u_1=(1,1,1),u_2=(0,1,2),u_3=(0,1,3)$
	
	% 4.51.b
	\item $u_1=(1,1,0),u_2=(0,1,1),u_3=(0,2,1)$
	\end{enumerate}
	
\item[4.52.] Show that the vectors $u_1,u_2,u_3$ do not span $\mathbb{R}^3$. Specifically, find conditions on $a,b,c$ so that $v=(a,b,c)$ in $\mathbb{R}^3$ is a linear combination of $u_1,u_2,u_3$:
	\begin{enumerate}
	% 4.52.a
	\item $u_1=(1,1,1),u_2=(1,2,3),u_3=(0,1,2)$
	
	% 4.52.b	
	\item $u_1=(1,1,0),u_2=(0,1,1),u_3=(1,3,2)$
	\end{enumerate}

\item[4.53.] Show that the polynomials $p_1=(y-1)^2$, $p_2=t-1$, and $p_3=1$ span the space $\mathbb{P}_2(t)$ of polynomials of degree $\leq 2$, that is, show that any polynomial $f=at^2+bt+c$ in $\mathbb{P}_2(t)$ is a linear combination of $p_1,p_2,p_3$.

\item[4.55.] Prove that span$(S)$ is the intersection of all subspaces of $V$ containing $S$.


\end{enumerate}























\end{document}