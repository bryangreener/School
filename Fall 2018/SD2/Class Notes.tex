\documentclass{report}
\usepackage[margin=1in]{geometry} 
\usepackage{amsmath,amsthm,amssymb,amsfonts}
\usepackage{tabto}
\usepackage[yyyymmdd]{datetime}
\renewcommand{\dateseparator}{--}
\newcommand{\N}{\mathbb{N}}
\newcommand{\Z}{\mathbb{Z}}

% For definitions
%\newtheorem{defn}{Definition}[section]
%\newtheorem{thrm}{Theorem}[section]
%\newtheorem{ex}[Example}[section]
\newtheorem*{ex}{Example}
\newtheorem*{defn}{Definition}
\newtheorem*{thrm}{Theorem}
\newtheorem*{lemma}{Lemma}
\newtheorem*{result}{Result}

% For circled text
\usepackage{tikz}
\newcommand*\circled[1]{\tikz[baseline=(char.base)]{
            \node[shape=circle,draw,inner sep=0.8pt] (char) {#1};}}

\usepackage{pgf}
\usetikzlibrary{arrows, automata}

% For equation system alignment
\usepackage{systeme,mathtools}
% Usage:
%	\[
%	\sysdelim.\}\systeme{
%	3z +y = 10,
%	x + y +  z = 6,
%	3y - z = 13}

\newenvironment{problem}[2][Problem]{\begin{trivlist}
\item[\hskip \labelsep {\bfseries #1}\hskip \labelsep {\bfseries #2.}]}{\end{trivlist}}
%If you want to title your bold things something different just make another thing exactly like this but replace "problem" with the name of the thing you want, like theorem or lemma or whatever
 
%used for matrix vertical line
\makeatletter
\renewcommand*\env@matrix[1][*\c@MaxMatrixCols c]{%
  \hskip -\arraycolsep
  \let\@ifnextchar\new@ifnextchar
  \array{#1}}
\makeatother 
 
% Change chapter numbering
\newcommand{\mychapter}[2]{
	\setcounter{chapter}{#1}
	\setcounter{section}{0}
	\chapter*{#2}
	\addcontentsline{toc}{chapter}{#2}
}

\usepackage{graphicx}
\graphicspath{ {images/} }

\begin{document}
 
\tableofcontents{}
\mychapter{1}{2018-02-14}
\section{Unit Testing}
Qualities of unit tests:
\begin{enumerate}
\item Single function
\item Each test is independent and self contained
\item No output unless there is a failure
\item Check all return values and side effects. Side effects are things like changes in DB and changes to global variables and others. Asserts are used to check.
\end{enumerate}
\subsection{Coming up with Unit Tests}
Think of bad stuff that people can put in. Think of the limits of data input and output. Make sure if using RNG that a seed is set to verify the same results in multiple tests.

\subsection{FIRST Properties of Good Tests}
\begin{itemize}
\item {[}F{]}ast
\item {[}I{]}solated
\item {[}R{]}epeatable
\item {[}S{]}elf-validating
\item {[}T{]}imely
\end{itemize}

\subsection{Right-BICEP}
\begin{tabular}{ll}
Right & Are the results right?\\
B & Are all the boundary conditions correct?\\
I & Can you check inverse relationships?\\
C & Can you cross-check results using other means?\\
E & Can you force error conditions to happen?\\
P & Are performance characteristics within bounds?\\
\end{tabular}

\begin{defn}
Boundary conditions are the range of values that values must remain within. An example of this is entering in patient ages where to qualify for a treatment they must be between 50 and 70 yrs old. Anything outside of this range will be denied. An example for our quad solver is for input a we cannot allow the number 0 to be input. Thus we get a boundary condition.
\end{defn}

\begin{defn}
To force error conditions is just to apply inputs or methods that will definitely cause errors. Then you look for the appropriate error flagging and responses from the program.
\end{defn}

\subsection{Boundary Conditions: "CORRECT"}
\begin{itemize}
\item Conformance\\  Does the value conform to an expected format?
\item Ordering\\  Is the set of values ordered or unordered as appropriate?
\item Range\\  Is the value within reasonable min and max values?
\item Reference\\  Does the code reference anything external that isn't under direct control of the code itself?
\item Existence\\  Does the value exist (is it non-null, nonzero, present in a set, and so on)?
\item Cardinality\\  Are there exactly enough values?
\item Time (abs and rel)\\  Is everything happening in order? At the right time? In time?
\end{itemize}

\mychapter{2}{2018-02-26}
\section{Ethics Paper Outline}
\subsection{Main Paper}
This section should be around 4 pages long.
\begin{itemize}
\item Title
\item Introduction
	\begin{itemize}
	\item Lead-in Sentence
	\item Short descriptions of sections in paper
	\item Purpose followed by what's coming	
	\end{itemize}

\item Ethics in General (optional)
\item Ethical Standards\\
	Talk about codes of ethics from the IT/Programmer point of view to keep a connection between the ethics and the purpose of this paper.
	\begin{itemize}
	\item ACM Ethics
	\item IEEE Ethics
	\end{itemize}
\item Laws\\
	Can talk about international, federal, state, local laws. Mention a couple of these that apply in this setting. Example would be HIPAA, FERPA, copyright, patent, etc.
\item Company Policies\\
	NDA, Non-compete, etc.
\end{itemize}
\subsection{Case Studies}
Each case study should be around 2 pages long. Each partner in the project handles one case study on their own.
\begin{itemize}
\item Case Study 1
	\begin{itemize}
	\item Title/Name
	\item Introduction
	\item Case
	\item Their (original study) Analysis
	\item Your Analysis
	\end{itemize}
\item Case Study 2
\item Case Study 3
\end{itemize}

\section{Timeline}
\begin{itemize}
\item CUNIT 2018-02-28
\item After spring break, review quadsolver and make testing plan (written document)\\
These tests do not need to be done in a testing plan, only documented plans:
	\begin{itemize}
	\item Tests: Use Cases
	\item Coverage Tests
	\item Automate Tests
	\item UNIT Tests
	\item 508 Compliance (if appropriate)
	\item if web project: are required browsers supported?
	\item Test different OSs
	\item Essentially specify how different project requirements will be tested.
	\item Usability testing (how to get the typical user to test it)
	\item Performance	
	\end{itemize}
	General format of testing plan document:
	\begin{itemize}
	\item Title
	\item Introduction with short description of each type of test.
	\item References for how you plan to implement each tests.
	\end{itemize}
\item Main Project Testing Plan
\end{itemize}
\subsection{Feasibility Report}
	Brief piece that goes with material along with final report and for presentation of project.
\begin{itemize}
\item Title
\item Introduction
\item Background Section\\
	This is information on the client and the problem they want solved. For example what services do they have, where are they located, etc. The goal is to show the solution and how it fits into their environment. It also explains their problem (task they need accomplished) and their workflow.
	\begin{itemize}
	\item Issues with existing process (if applicable)
	\item Brief possible solutions.
	\end{itemize}
\item Design decisions\\
	List multiple decisions made for the program and compare them. These might not all have been implemented so explain why that idea may have been trashed. Explain reasoning behind certain ideas.
\item Stories/Spikes\\
	For stories split it up into each release.
\item Resources\\
	When will you have the final specs? Explain resources needed to complete project. when will these resources become available? If NDAs are required, when will they be here? etc.
\item Legal
	\begin{itemize}
	\item NDA
	\item Ownership
	\item License
	\item etc.
	\end{itemize}
\end{itemize}
	
	
	
	
\mychapter{3}{2018-03-21}
\section{Quad Evaluation Grading}
\begin{itemize}
\item Version Control
\item Documentation
	\begin{itemize}
	\item README
	\item Use of Version Control [readme, makefile, etc]
	\item Programming Standards
	\item Unit Testing Standards
	\item Automation Standards
	\item References to IEEEfp or Quad Solvers
	\item Stories
	\item TPS reports
	\item Testing plan/results
	\item ...
	\end{itemize}
\item Spikes
	\begin{itemize}
	\item README [Name of directories and what is in directory]
	\item IEEEfp
	\item Precision and Range
	\item IO
	\item Quadratic Equation (initial code trying things out with it)
	\item Testing sqrt function
	\item Relative/Absolute Error
	\item unit tests (documentation and code)
	\item system (functional) testing
	\end{itemize}
\item src
	\begin{itemize}
	\item directory containing headers
	\item main
	\item io
	\item qsolve
	\item directory containing method that generates string to output
	\end{itemize}
\item distribution
	\begin{itemize}
	\item contains something that generates tar file
	\end{itemize}
\item bin 
	\begin{itemize}
	\item contains something that generates a binary file
	\end{itemize}
\item If possible, add logging so that a line is written to a log file every time that a function is executed. If this were a parallel program then we'd want to log thread IDs as well.
\item If possible, add a help feature to help the user operate the program. Also give a version number.
\end{itemize}
At meeting we show him the program running and the file structure and everything else.\\
Final exam will have questions about security, stack on web server, etc. Possibly take home.\\
Need to attend two senior design presentations and write a report on one of them.


\mychapter{4}{2018-03-26}
\section{Senior Design Presentations}
\subsection{Senior Design Presentation Review Papers}
\begin{itemize}
\item What is the group trying to accomplish?
\item Who is the audience for their product?
\item Is it maintainable/secure/tested?
\item Introduction: Where/When/What was the talk about
\item Explain content
\item Explain style
\item Pick three things that could make the talk better
\end{itemize}

\subsection{Senior Design Presentation Paper Outline}
\begin{itemize}
\item Introduction
	\begin{itemize}
	\item Date
	\item Place
	\item Title of Talk
	\item Explain what we are going to do
	\end{itemize}
\item Content
	\begin{itemize}
	\item Give background about history of company, size of company, what type of hardware/software service do they have?
		\begin{itemize}
		\item Define actors
		\item Define other terminology as well
		\end{itemize}
	\item Give information about the problem. Information about other software options and so on.
	\end{itemize}
\item Style
	\begin{itemize}
	\item Design Decisions (framework/os/language). This section can have a substructure. The goal is to convince the audience that this implementation is maintainable and reasonable.
	\item Implementation. Define actors and how they will interact with the product.
	\item Maintenance
	\item Security
	\item Types of testing done.
	\end{itemize}
\item Summary: Define the problem, solution to the problem, importance of maintenance, security, the fact that it was testable and tested. Reinforce the fact that these are important pieces and the product given to the client meets their needs.
\end{itemize}
You can give technical details but keep in mind the audience isn't only programmers so these have to be explained in a way that the general public can understand why it is necessary.\\
Try to avoid putting large amounts of code on the screen at one time.\\
\subsection{Presentation Style}
\begin{itemize}
\item Be confident
\item Good Posture
\item Project Voice
\item Be honest about failed pieces but dont dwell on it.
\item Speak in third person
\item Mind speaking speed
\item When switching from one presenter to another, give a verbal handoff (introduce the new speaker)
\item Presentation Content
	\begin{itemize}
	\item Color
	\item Text Size
	\item Dont use too many bullet points
	\item Good to have graphs and diagrams
	\item DONT USE LASER POINTER TO POINT AT STUFF
	\item DONT POINT AT THE SLIDES WITH YOUR FINGER
	\item If demonstrating project, use a recording of it working, don't do a live demo.
	\end{itemize}
\end{itemize}

\section{Testing Plan}
\subsection{Introduction}
	Introduce the testing plan and its contents. Standard intro.
\subsection{Unit Testing}
	Explain who does the testing, that it is done on every module and there are multiple done. Explain the types of tests done. Indicate that this will be automated and describe the unit test framework. Explain that it is expected that all unit tests will pass on the system. Explain how thorough these will be. Coverage testing is included in the UNIT testing section.
\subsection{System Testing}
	Similar to unit testing explanation in describing the framework and the details of how tests will be performed. Indicate how many will be done. They are expected to all pass. Address how we will handle relative/absolute error. You could add other flags to program that prints out numbers to six decimal places in order to compare values to better precision. All tests expected to pass except unusual rounding conditions and these errors will be looked into.
\subsection{Usability Testing}
	Need to test the expected users. "We will give them some use cases. We will watch them and ask questions indicating whether or not they understood the output/input." Do not ask leading questions. Try to be as neutral as possible. Ask about formatting. Ask if errors returned are well defined. Can also add information about logging or help commands and if they are understandable.
\subsection{Items for each Section}
Each section should have information for who will do the testing, how it will be tested, what is to be tested.
\end{document}
