\documentclass[12pt]{article}
\usepackage[margin=1in]{geometry} 
\usepackage{amsmath,amsthm,amssymb,amsfonts}
\usepackage{tabto}
\usepackage{hyperref}

\usepackage{arydshln} % gives hdashline and cdashline
\newcommand*{\tempb}{\multicolumn{1}{:c}{}} % Used for block matrices

% Spacers:
% BEGIN BLOCK------------------------------------------
% END BLOCK============================================




\newcommand{\N}{\mathbb{N}}
\newcommand{\Z}{\mathbb{Z}}

% CUSTOM SETTINGS
% BEGIN BLOCK------------------------------------------
% For equation system alignment
\usepackage{systeme,mathtools}
% Usage:
%	\[
%	\sysdelim.\}\systeme{
%	3z +y = 10,
%	x + y +  z = 6,
%	3y - z = 13}


% For definitions
\newtheorem{defn}{Definition}[section]
\newtheorem{thrm}{Theorem}[section]

% For circled text
\usepackage{tikz}
\usetikzlibrary{matrix}
\newcommand*\circled[1]{\tikz[baseline=(char.base)]{
            \node[shape=circle,draw,inner sep=0.8pt] (char) {#1};}}

\newenvironment{problem}[2][Problem]{\begin{trivlist}
\item[\hskip \labelsep {\bfseries #1}\hskip \labelsep {\bfseries #2.}]}{\end{trivlist}}
%If you want to title your bold things something different just make another thing exactly like this but replace "problem" with the name of the thing you want, like theorem or lemma or whatever
 
%used for matrix vertical line
\makeatletter
\renewcommand*\env@matrix[1][*\c@MaxMatrixCols c]{%
  \hskip -\arraycolsep
  \let\@ifnextchar\new@ifnextchar
  \array{#1}}
\makeatother 

% END BLOCK============================================

\newtheorem*{lemma}{Lemma} %added
\newtheorem*{result}{Result} %added
\newtheorem*{theorem}{Theorem} %added
\theoremstyle{definition}
\newtheorem*{solution}{Solution} %added
\theoremstyle{plain}


\setlength{\textwidth}{6in}
\setlength{\textheight}{10truein}
\setlength{\evensidemargin}{.5in}
\setlength{\oddsidemargin}{.25in}
\setlength{\topmargin}{-1 in}
\setlength{\parskip} {0.125 in}
\setlength{\parindent} {0 in}



\begin{document}
\large


{\bf Name(s) \hrulefill}

{\bf
\begin{centering}
Elementary Linear Algebra (MATH 2300) \\
Quiz \#10  \\
\end{centering}
}


\vskip .125 in

{\bf Instructions.} Show or explain all of your work.  You are allowed to work on this
quiz with a colleague and/or your notes.

1. [10] Suppose a matrix is given as
$$M=\left[\begin{array}{rrr}
1 & -1 & 2 \\
-1 & 1 & 2 \\
2 & 2 & -2 \\
\end{array}\right].$$

(a) Find an orthogonal set of eigenvectors for $M$.
\[ M-\lambda I = \begin{bmatrix}[rrr]1-\lambda&-1&2\\-1&1-\lambda&2\\2&2&-2-\lambda\\\end{bmatrix} \]
\begin{align*}
M-\lambda I &= \mathrm{det}\begin{bmatrix}[rrr]1-\lambda&-1&2\\-1&1-\lambda&2\\2&2&-2-\lambda\\\end{bmatrix}\\
&= (1-\lambda)(-2+\lambda+\lambda^2-4) +1(2+\lambda - 4) + 2(-2-2+2\lambda)\\
&= (\lambda-2)^2(\lambda+4)
\end{align*}
Thus our eigenvalues are $\lambda=2$ with multiplicity of 2 and $\lambda=-4$.\\
Let $\lambda=2$, then
\begin{align*}
M - \lambda I_2 &= \begin{bmatrix}[rrr]-1&-1&2\\-1&-1&2\\2&2&-4\\\end{bmatrix}\\
&\xrightarrow[]{\mathrm{rref}} \begin{bmatrix}[rrr]1&1&-2\\0&0&0\\0&0&0\\\end{bmatrix}
\end{align*}
Since we have two free variables, we will get two different eigenvectors which also agrees with the fact that our eigenvalue of 2 has a multiplicity of 2.\\
So $a+b-2c=0$ and we let $b=1,c=0$ then we get $a=-1$. Next we let $b=0,c=1$ then $a=2$. Thus our eigenvectors for $\lambda=2$ are
\[ \left \{ \begin{bmatrix}[r]-1\\1\\0\\\end{bmatrix},\begin{bmatrix}[r]2\\0\\1\\\end{bmatrix} \right \} \]
Next let $\lambda=-4$, then
\begin{align*}
M - \lambda I_{-4} &= \begin{bmatrix}[rrr]5&-1&2\\-1&5&2\\2&2&2\\\end{bmatrix}\\
&\xrightarrow[]{\mathrm{rref}} \begin{bmatrix}[rrr]1&0&\frac{1}{2}\\0&1&\frac{1}{2}\\0&0&0\\\end{bmatrix}
\end{align*}
By parameterizing this solution, we get the eigenvector
\[ \left\{\begin{bmatrix}[r]-\frac{1}{2}\\-\frac{1}{2}\\1\\\end{bmatrix}\right\} \]
To make our lives easier we multiply this eigenvector by $2$ to get a simpler eigenvector
\[ \left\{\begin{bmatrix}[r]-1\\-1\\2\\\end{bmatrix}\right\} \]
Next we normalize these results to obtain our orthonormal basis. For each eigenvector we normalize it by performing the following operation
\[ \dfrac{1}{\sqrt{a^2+b^2+c^2}}(a,b,c) \]
Putting these normalized vectors as columns in a matrix gives us our $P$ value. Thus
\[ D = P^TMP =\begin{bmatrix}[rrr]2&-2\sqrt{\frac{2}{5}}&0\\-2\sqrt{\frac{2}{5}}&2&0\\0&0&-4\\\end{bmatrix} \]



\newpage
(b) For each eigenvalue $\lambda$, determine the matrix of the orthogonal projection
$\mathbb{R}^3\xrightarrow{\ E_\lambda\ }V_\lambda$, where $V_\lambda$ denotes the eigenspace
corresponding to $\lambda$.  (A formula for orthogonal projection appears on the last page of Exam 3.)

(c) Check your answers.  You should confirm that: \\
The column space of $E_\lambda$ coincides with $V_\lambda$ for each eigenvalue, \\
the multiplicity of each eigenvalue coincides with the dimension of $V_\lambda$, \\
$E_\lambda^2=E_\lambda$ for each eigenvalue, \\
if $\lambda\neq\mu$, then $E_\lambda\circ E_\mu=0$, \\
$\sum_\lambda E_\lambda$ is an identity matrix, and \\
$\sum_\lambda \lambda E_\lambda=M$. \\

\end{document}



