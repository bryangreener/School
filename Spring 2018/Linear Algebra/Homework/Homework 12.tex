\documentclass[12pt]{article}
\usepackage[margin=1in]{geometry} 
\usepackage{amsmath,amsthm,amssymb,amsfonts}
\usepackage{tabto}
\usepackage{hyperref}

% Spacers:
% BEGIN BLOCK------------------------------------------
% END BLOCK============================================




\newcommand{\N}{\mathbb{N}}
\newcommand{\Z}{\mathbb{Z}}

% CUSTOM SETTINGS
% BEGIN BLOCK------------------------------------------
% For equation system alignment
\usepackage{systeme,mathtools}
% Usage:
%	\[
%	\sysdelim.\}\systeme{
%	3z +y = 10,
%	x + y +  z = 6,
%	3y - z = 13}


% For definitions
\newtheorem{defn}{Definition}[section]
\newtheorem{thrm}{Theorem}[section]

% For circled text
\usepackage{tikz}
\newcommand*\circled[1]{\tikz[baseline=(char.base)]{
            \node[shape=circle,draw,inner sep=0.8pt] (char) {#1};}}

\newenvironment{problem}[2][Problem]{\begin{trivlist}
\item[\hskip \labelsep {\bfseries #1}\hskip \labelsep {\bfseries #2.}]}{\end{trivlist}}
%If you want to title your bold things something different just make another thing exactly like this but replace "problem" with the name of the thing you want, like theorem or lemma or whatever
 
%used for matrix vertical line
\makeatletter
\renewcommand*\env@matrix[1][*\c@MaxMatrixCols c]{%
  \hskip -\arraycolsep
  \let\@ifnextchar\new@ifnextchar
  \array{#1}}
\makeatother 

% END BLOCK============================================

\newtheorem*{lemma}{Lemma} %added
\newtheorem*{result}{Result} %added
\newtheorem*{theorem}{Theorem} %added
\theoremstyle{definition}
\newtheorem*{solution}{Solution} %added
\theoremstyle{plain}

% HEADER
% BEGIN BLOCK------------------------------------------
\usepackage{fancyhdr}
 
\pagestyle{fancy}
\fancyhf{}
\lhead{Homework \#11}
\rhead{Bryan Greener}
\cfoot{\thepage}
% END BLOCK============================================

% TITLE
% BEGIN BLOCK------------------------------------------
\title{Bryan Greener}
\author{MATH 2300 CRN:15163}
\date{2018-03-19}
\begin{document}
\maketitle
% END BLOCK============================================

\TabPositions{4cm}

\begin{enumerate}
\item[10.57]Consider the matrix $A=\begin{bmatrix}[rrrr]1&2&-2&3\\3&-1&5&0\\4&0&2&1\\1&7&2&-3\\\end{bmatrix}$. Find the cofactor: (a)$A_{3,1}$;(b)$A_{2,3}$;(c)$A_{4,2}$.
	\begin{enumerate}
	\item We delete row 3 and column 1 from the matrix to get
	\[ A_{3,1}=\mathrm{det}\begin{bmatrix}[rrr]2&-2&3\\-1&5&0\\7&2&-3\\\end{bmatrix} = -135 \]
	\item We delete row 2 and column 3 from the matrix to get
	\[ A_{2,3}=\mathrm{det}\begin{bmatrix}[rrr]1&2&3\\4&0&1\\1&7&-3\\\end{bmatrix} = 103 \]
	\end{enumerate}
	
\item[10.58]Find $\mathrm{det}(A)$, $\mathrm{adj}(A)$, and $A^{-1}$, where:\\
(a) $A=\begin{bmatrix}[rrr]1&1&0\\1&1&1\\0&2&1\\\end{bmatrix}$; (b) $A=\begin{bmatrix}[rrr]1&2&2\\3&1&0\\1&1&1\\\end{bmatrix}$
	\begin{enumerate}
	\item
	\begin{align*}
	\mathrm{det}(A)&= -2\\
	A^{-1}&=\frac{1}{|A|}(\mathrm{adj}(A))=-\frac{1}{2}\begin{bmatrix}[rrr]-1&-1&1\\-1&1&-1\\2&-2&0\\\end{bmatrix} = \begin{bmatrix}[rrr]\frac{1}{2}&\frac{1}{2}&-\frac{1}{2}\\\frac{1}{2}&-\frac{1}{2}&\frac{1}{2}\\-1&1&0\\\end{bmatrix}
	\end{align*}
	\end{enumerate}		
	
\item[10.59]Find the classical adjoint of each matrix in Problem 10.54.
	\begin{enumerate}
	\item
	\begin{align*}
	A&=\begin{bmatrix}[rrrr]1&2&2&3\\1&0&-2&0\\3&-1&1&-2\\4&-3&0&2\\\end{bmatrix}\\
	\mathrm{det}(A)A^{-1} &= \begin{bmatrix}[rrrr]-16&-29&-26&-2\\-30&-38&-16&29\\-8&51&-13&-1\\-13&1&28&-18\\\end{bmatrix}
	\end{align*}
	Since $A$ is invertible we are able to use the formula above instead of writing out each determinant per index.
	\end{enumerate}		

\item[10.63]For $k=1,2,3$ find the sum $S_k$ of all principal minors of order $k$ for:\\
(a)$A=\begin{bmatrix}[rrr]1&3&2\\2&-4&3\\5&-2&1\\\end{bmatrix}$; (b) $B=\begin{bmatrix}[rrr]1&5&-4\\2&6&1\\3&-2&0\\\end{bmatrix}$; (c) $C=\begin{bmatrix}[rrr]1&-4&3\\2&1&5\\4&-7&11\\\end{bmatrix}$.
	\begin{enumerate}
	\item We start by finding $S_1$ which is just the trace of $A$.
	\[ S_1 = \mathrm{tr}(A)=1-4+1=-2 \]
	Next to find $S_2$ we add the cofactors $A_{1,1}, A_{2,2}, A_{3,3}$. 
	\begin{align*}
	A_{1,1}&=\mathrm{det}\begin{bmatrix}[rr]-4&3\\-2&1\\\end{bmatrix} = 2\\
	A_{2,2}&=\mathrm{det}\begin{bmatrix}[rr]1&2\\5&1\\\end{bmatrix} = -9\\
	A_{3,3}&=\mathrm{det}\begin{bmatrix}[rr]1&3\\2&-4\\\end{bmatrix} = -10\\
	S_2 &= 2-9-10 = -17
	\end{align*}
	Finally we find $S_3$ by taking the determinant of $A$ itself.
	\[ \mathrm{det}(A)= 73 \]
	\end{enumerate}		
	
\item[10.65]Solve by determinants: (a) $\sysdelim{.}{.}\systeme[xy]{3x+5y=8,4x-2y=1}$; (b) $\sysdelim{.}{.}\systeme[xy]{2x-3y=-1,4x+7y=-1}$.
	\begin{enumerate}
	\item We start by finding the determinant of the system
	\[ \mathrm{det}\begin{bmatrix}[rr]3&5\\4&-2\\\end{bmatrix} = -26 \]
	Since the determinant is not zero, this system has a unique solution. Next we find $N_x$ and $N_y$ by replacing $x$ and $y$ (respective to the $N$ value we are computing) with the coefficients to the right of the equal signs in the system.
	\[ N_x = \mathrm{det}\begin{bmatrix}[rr]8&5\\1&-2\\\end{bmatrix} = -21, \quad N_y = \mathrm{det}\begin{bmatrix}[rr]3&8\\4&1\\\end{bmatrix} = -29 \]
	Thus the unique solution to the system is $x=\frac{N_x}{D} = \frac{21}{26}$, $y=\frac{N_y}{D} = \frac{29}{26}$.
	\end{enumerate}		
	
\item[10.68]Find the parity of the permutations $\sigma = 32154$, $\tau = 13524$, and $\pi = 42531$ in $S_5$.\\
	\begin{solution}
	We find the number of inversions per set then if the number of inversions is even, then $\mathrm{sgn}(\sigma) = 1$, otherwise $\mathrm{sgn}(\sigma)=-1$. This rule will apply for $\sigma$, $\tau$, and $\pi$.\\
	Below are the numbers producing inversions and their corresponding inversions to their right:
	\begin{align*}
	\sigma&\\
	3&: (3,2),(3,1)\\
	2&: (2,1)\\
	1&: \mathrm{none}\\
	5&: (5,4)\\
	4&: \mathrm{none}\\
	=& 1\\
	{}\\
	\tau&\\
	1&: \mathrm{none}\\
	3&: (3,2)\\
	5&: (5,2),(5,4)\\
	2&: \mathrm{none}\\
	4&: \mathrm{none}\\
	=& -1\\
	{}\\
	\pi&\\
	4&: (4,2),(4,3),(4,1)\\
	2&: (2,1)\\
	5&: (5,3),(5,1)\\
	3&: (3,1)\\
	1&: \mathrm{none}\\
	=& -1
	\end{align*}
	\end{solution}
\item[10.69]For the permutations in Problem 10.68, find: (a) $\tau\circ\sigma$; (b) $\pi\circ\sigma$; (c) $\sigma^{-1}$; (b) $\tau^{-1}$.
	\begin{enumerate}
	\item We start by writing the mappings of $\tau$ and $\sigma$.
	\[ \tau = \begin{matrix}1&2&3&4&5\\\downarrow&\downarrow&\downarrow&\downarrow&\downarrow\\1&3&5&2&4\\\end{matrix} \qquad \sigma = \begin{matrix}1&2&3&4&5\\\downarrow&\downarrow&\downarrow&\downarrow&\downarrow\\3&2&1&5&4\\\end{matrix} \]
	Next we plug in our values from $\sigma$ into $\tau$ to get $\tau \circ \sigma = 53142$.
	\item[(c)] \[ \sigma^{-1}=\begin{matrix}[rrrrr]3&2&1&5&4\\\downarrow&\downarrow&\downarrow&\downarrow&\downarrow\\1&2&3&4&5\\\end{matrix} \]
	Thus $\sigma^{-1}=32154$.
	\end{enumerate}		
	
\item[10.74]Let $D: V\rightarrow V$ be the differential operator, that is, $D(f(t))=df/dt$. Find $\mathrm{det}(T)$ if $V$ is the vector space of functions with bases: (a) $\{1,t,...,t^5\}$; (b) $\{e^t,e^{2t},e^{3t}\}$; (c) $\{\sin t, \cos t\}$.
	\begin{enumerate}
	\item
	\begin{align*}
	D(t^5)&=5t^4=0t^5+5t^4+0t^3+0t^2+-t+0\\
	D(t^4)&=4t^3=0t^5+0t^4+4t^3+0t^2+0t+0\\
	D(t^3)&=3t^2=0t^5+0t^4+0t^3+3t^2+0t+0\\
	D(t^2)&=2t  =0t^5+0t^4+0t^3+0t^2+2t+0\\
	D(t)  &=1   =0t^5+0t^4+0t^3+0t^2+0t+1\\
	D(1)  &=0   =0t^5+0t^4+0t^3+0t^2+0t+0\\
	D&= \begin{bmatrix}[rrrrrr]0&5&0&0&0&0\\0&0&4&0&0&0\\0&0&0&3&0&0\\0&0&0&0&2&0\\0&0&0&0&0&1\\0&0&0&0&0&0\\\end{bmatrix}\\
	\mathrm{det}(D) &= 0
	\end{align*}
	
	\item
	\begin{align*}
	D(e^t)&=e^t=1e^t+0e^{2t}+0e^{3t}\\
	D(e^{2t})&=2e^{2t}=0e^t+2e^{2t}+0e^{3t}\\
	D(e^{3t})&=3e^{3t}=0e^t+0e^{2t}+3e^{3t}\\
	D&=\begin{bmatrix}[rrr]1&0&0\\0&2&0\\0&0&3\\\end{bmatrix}\\
	\mathrm{det}(D)&=6
	\end{align*}
	\item
	\begin{align*}
	D(\sin t) &= 0\sin t + 1\cos t\\
	D(\cos t) &= -1\sin t + 0\cos t\\
	D&=\begin{bmatrix}[rr]0&1\\-1&0\\\end{bmatrix}\\
	\mathrm{det}(D)&= 1
	\end{align*}
	\end{enumerate}
	
\item[10.76]Prove: (a) $\mathrm{det}(1_V)=1$, where $1_V$ is the identity operator; (b) $\mathrm{det}(T^{-1})=\mathrm{det}(T)^{-1}$ when $T$ is invertible.
	\begin{enumerate}
	\item By the definition of an identity mapping, $\mathrm{det}(1_V) = \mathrm{det}(I)$ and since $I$ is diagonal then $\mathrm{det}(I)=1$. Thus $\mathrm{det}(1_V)=1$.
	
	\item Note that $T^{-1}T=I$. So $1=\mathrm{det}(I)=\mathrm{det}(T^{-1}T) = \mathrm{det}(T^{-1})\mathrm{det}(T)$ and so $\mathrm{det}(T^{-1}) = \mathrm{det}(T)^{-1}$.
	\end{enumerate}
	
\item[10.77]Find the volume $V(S)$ of the parallelepiped $S$ in $\mathbb{R}^3$ determined by the vectores: (a) $u_1=(1,2,-3),u_2=(3,4,-1),u_3=(2,-1,5)$; (b) $u_1=(1,1,3),u_2=(1,-2,-4),u_3=(4,1,2)$.
	\begin{enumerate}
	\item We set the given vectors as rows in a matrix and find its determinant:
	\[ \mathrm{det}\begin{bmatrix}[rrr]1&2&-3\\3&4&-1\\2&-1&5\\\end{bmatrix} = 18 \]
	Thus $V(S) = |-18| = 18$.
	\end{enumerate}
	
\item[10.79]Let $V$ be the space of $2 \times 2$ matrices $M=\begin{bmatrix}[rr]a&b\\c&d\\\end{bmatrix}$ over $\mathbb{R}$. Determine whether or not $D: V\rightarrow R$ is 2-linear (with respect to the rows), where: (a) $D(M)=ac-bd$; (b) $D(M)=ab-cd$; (c) $D(M)=0$; (d) $D(M)=1$.
	\begin{enumerate}
	\item Let $A=(a,b)$ and $B=(c,d)$. Then $D(A,B)=\begin{bmatrix}[rr]a&b\\c&d\\\end{bmatrix}=ac-db$ and $D(B,A)=\begin{bmatrix}[rr]c&d\\a&b\\\end{bmatrix}=ca-bd$. So $ac-db = ca-bd$.\\
	Next let $kD(A,B)=\begin{bmatrix}[rr]ka&kb\\kc&kd\\\end{bmatrix}=k(ac-bd)$ and $D(kA,B)=\begin{bmatrix}[rr]ka&kb\\c&d\\\end{bmatrix}=kac-kbd=k(ac-bd)$.\\
	Since $M$ holds under addition and multiplication, then $D(M)$ is 2-linear.
	\item Let $A=(a,b)$ and $B=(c,d)$. Then $D(A,B)=\begin{bmatrix}[rr]a&b\\c&d\\\end{bmatrix}=ab-cd$ and $D(B,A)=\begin{bmatrix}[rr]c&d\\a&b\\\end{bmatrix}=cd-ab$. So $ -ab+cd \neq ab-cd$.\\
	Since $M$ does not hold under addition then $M$ is not 2-linear.
	\item 
	\item
	\end{enumerate}
	
\item[10.82]Suppose $A$ is orthogonal, that is $A^TA=1$. Show that $\mathrm{det}(A)=\pm 1$.
	\begin{proof}
	By theorem 10.1, $\mathrm{det}(A) = \mathrm{det}(A^T)$.\\
		
	\end{proof}
	
\item TWO OF 10.67, 10.83, 10.85, 10.86	
\item[10.67]Prove theorem 10.11: The homogeneous system $AX=0$ has a nonzero solution if and only if $D=|A|=0$.
	\begin{proof}
	First we prove that if $AX=0$ has a nonzero solution, then $D=|A|=0$.\\
	Suppose that $AX=0$, then $A^{-1}(AX)=A^{-1}0$. So $1X=0$.\\
	Next we prove that if $D=|A|=0$, then $AX=0$. FINISH ME
	\end{proof}
\end{enumerate}
\end{document}