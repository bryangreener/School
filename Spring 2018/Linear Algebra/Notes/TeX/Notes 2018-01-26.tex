\documentclass[12pt]{article}
\usepackage[margin=1in]{geometry} 
\usepackage{amsmath,amsthm,amssymb,amsfonts}
\usepackage{tabto}
\usepackage[yyyymmdd]{datetime}
\renewcommand{\dateseparator}{--}
\newcommand{\N}{\mathbb{N}}
\newcommand{\Z}{\mathbb{Z}}

% For definitions
%\newtheorem{defn}{Definition}[section]
%\newtheorem{thrm}{Theorem}[section]
%\newtheorem{ex}[Example}[section]
\newtheorem*{ex}{Example}
\newtheorem*{defn}{Definition}
\newtheorem*{thrm}{Theorem}
\newtheorem*{lemma}{Lemma}
\newtheorem*{result}{Result}

% For circled text
\usepackage{tikz}
\newcommand*\circled[1]{\tikz[baseline=(char.base)]{
            \node[shape=circle,draw,inner sep=0.8pt] (char) {#1};}}

% For equation system alignment
\usepackage{systeme,mathtools}
% Usage:
%	\[
%	\sysdelim.\}\systeme{
%	3z +y = 10,
%	x + y +  z = 6,
%	3y - z = 13}

\newenvironment{problem}[2][Problem]{\begin{trivlist}
\item[\hskip \labelsep {\bfseries #1}\hskip \labelsep {\bfseries #2.}]}{\end{trivlist}}
%If you want to title your bold things something different just make another thing exactly like this but replace "problem" with the name of the thing you want, like theorem or lemma or whatever
 
%used for matrix vertical line
\makeatletter
\renewcommand*\env@matrix[1][*\c@MaxMatrixCols c]{%
  \hskip -\arraycolsep
  \let\@ifnextchar\new@ifnextchar
  \array{#1}}
\makeatother 
 
 




\begin{document}
\TabPositions{4cm}
%\renewcommand{\qedsymbol}{\filledbox}
%Good resources for looking up how to do stuff:
%Binary operators: http://www.access2science.com/latex/Binary.html
%General help: http://en.wikibooks.org/wiki/LaTeX/Mathematics
%Or just google stuff
 
\today {} Linear Algebra Notes\\
\begin{defn}
Let $A$ be an m x n matrix. The nullspace of $A$ is the set of all vectors $\vec{v}$ such that $A\vec{v} = \vec{0}$.\\
Notation: $N(A) = null(A) = nulspace(A) = \{\vec{v}: A\vec{v} = \vec{0}\}$
\end{defn}

\begin{defn}
$M$ is orthogonal is $MM^T = I$.
\end{defn}

Five Concepts Required for Linear Algebra:
\begin{enumerate}
\item Linear Combinations
\item Span of Set (Spanning Set)
\item Linear Independence
\item Basis of a Vector Space
\item Dimension
\end{enumerate}

\begin{defn}
Let $V$ be a vector space and $S \subset V$. A linear combination of elements of $S$ is any expression having the form $C_1\vec{v}_1+C_2\vec{v}_2+\dots+C_n\vec{v}_n$ where $C_1,C_2,...,C_n$ are scalars and $\vec{v}_1,\vec{v}_2,...,\vec{v}_n$ are in S.
\end{defn}

\begin{ex}
Write $(12,4)$ as a linear combination of $(1,0)$ and $(0,1)$.\\
Solution: $(12,4) = 12(1,0)+4(0,1)$
\end{ex}
\begin{ex}
Write $3+4x+x^2$ as a linear combination of $1$, $x$, and $x^2$.\\
Solution: $3+4x+x^2=3(1)+4(x)+1(x^2)$
\end{ex}
\begin{ex}
Write $(12,4)$ as a linear combination of $(0,1)$ and $(1,1)$.\\
Solution: $(12,4) = x(0,1) + y(1,1)$.\\
$12 = 0x+1y$, $4=1x+1y$\\
$y=12$, $x=-8$\\
$(12,4) = -8(0,1)+12(1,1)$
\end{ex}














\end{document}