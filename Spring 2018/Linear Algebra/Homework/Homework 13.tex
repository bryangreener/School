\documentclass[12pt]{article}
\usepackage[margin=1in]{geometry} 
\usepackage{amsmath,amsthm,amssymb,amsfonts}
\usepackage{tabto}
\usepackage{hyperref}

% Spacers:
% BEGIN BLOCK------------------------------------------
% END BLOCK============================================




\newcommand{\N}{\mathbb{N}}
\newcommand{\Z}{\mathbb{Z}}

% CUSTOM SETTINGS
% BEGIN BLOCK------------------------------------------
% For equation system alignment
\usepackage{systeme,mathtools}
% Usage:
%	\[
%	\sysdelim.\}\systeme{
%	3z +y = 10,
%	x + y +  z = 6,
%	3y - z = 13}


% For definitions
\newtheorem{defn}{Definition}[section]
\newtheorem{thrm}{Theorem}[section]

% For circled text
\usepackage{tikz}
\newcommand*\circled[1]{\tikz[baseline=(char.base)]{
            \node[shape=circle,draw,inner sep=0.8pt] (char) {#1};}}

\newenvironment{problem}[2][Problem]{\begin{trivlist}
\item[\hskip \labelsep {\bfseries #1}\hskip \labelsep {\bfseries #2.}]}{\end{trivlist}}
%If you want to title your bold things something different just make another thing exactly like this but replace "problem" with the name of the thing you want, like theorem or lemma or whatever
 
%used for matrix vertical line
\makeatletter
\renewcommand*\env@matrix[1][*\c@MaxMatrixCols c]{%
  \hskip -\arraycolsep
  \let\@ifnextchar\new@ifnextchar
  \array{#1}}
\makeatother 

% END BLOCK============================================

\newtheorem*{lemma}{Lemma} %added
\newtheorem*{result}{Result} %added
\newtheorem*{theorem}{Theorem} %added
\theoremstyle{definition}
\newtheorem*{solution}{Solution} %added
\theoremstyle{plain}

% HEADER
% BEGIN BLOCK------------------------------------------
\usepackage{fancyhdr}
 
\pagestyle{fancy}
\fancyhf{}
\lhead{Homework \#13}
\rhead{Bryan Greener}
\cfoot{\thepage}
% END BLOCK============================================

% TITLE
% BEGIN BLOCK------------------------------------------
\title{Bryan Greener}
\author{MATH 2300 CRN:15163}
\date{2018-04-01}
\begin{document}
\maketitle
% END BLOCK============================================

\TabPositions{4cm}

\begin{enumerate}
\item[11.53]Let $A=\begin{bmatrix}[rr]2&-3\\5&1\\\end{bmatrix}$ and $B=\begin{bmatrix}[rr]1&2\\0&3\\\end{bmatrix}$. Find $f(A)$, $g(A)$, $f(B)$, and $g(B)$ where $f(t)=2t^2-5t+6$ and $g(t)=t^3-2t^2+t+3$.
	\begin{enumerate}
	\item[f(A)]$=2\begin{bmatrix}[rr]2&-3\\5&1\\\end{bmatrix}^2-5\begin{bmatrix}[rr]2&-3\\5&1\\\end{bmatrix}+6\begin{bmatrix}[rr]1&0\\0&1\\\end{bmatrix} = \begin{bmatrix}[rr]-26&-3\\5&-27\\\end{bmatrix}$
	\item[g(A)]$=\begin{bmatrix}[rr]1&2\\0&3\\\end{bmatrix}^3-2\begin{bmatrix}[rr]1&2\\0&3\\\end{bmatrix}^2+\begin{bmatrix}[rr]1&2\\0&3\\\end{bmatrix}+3\begin{bmatrix}[rr]1&0\\0&1\\\end{bmatrix} = \begin{bmatrix}[rr]-40&39\\-65&-27\\\end{bmatrix}$
	\item[f(B)]$=\begin{bmatrix}[rr]3&6\\0&9\\\end{bmatrix}$
	\item[g(B)]$=\begin{bmatrix}[rr]3&12\\0&15\\\end{bmatrix}$
	\end{enumerate}
\item[11.54]Let $B=\begin{bmatrix}[rrr]8&12&0\\0&8&12\\0&0&8\\\end{bmatrix}$. Find a real matrix $A$ such that $B=A^3$.\\
	Let $B = \begin{bmatrix}[rrr]a&b&c\\d&e&f\\g&h&i\\\end{bmatrix}^3$.
	Expanding the cubed matrix then solving the system gives us the following matrix for $A$: $\begin{bmatrix}[rrr]2&1&-\frac{1}{2}\\0&2&1\\0&0&2\\\end{bmatrix}$

\item[11.55]For each matrix, find a polynomial having the matrix as a root:
	\begin{enumerate}
	\item $A=\begin{bmatrix}[rr]2&5\\1&-3\\\end{bmatrix}$\\
	First we find the trace of $A$. $\mathrm{tr}(A)=-1$. Next we find the determinant of $A$. $\mathrm{det}(A)=-11$. Next we plug this into $\Delta (t) = t^2+t-11$.
	\end{enumerate}
	
\item[11.56]Let $A$ be any square matrix and let $f(t)$ be any polynomial. Prove:
	\begin{enumerate}
	\item[(b)] $f(P^{-1}AP)=P^{-1}f(A)P$
	\begin{proof}
	Let $f(x)=\sum_{k=0}^nC_kx^k$ be a polynomial.\\
	Notice that $f(P^{-1}AP=\sum_{k=0}^nC_k(P^{-1}AP)^k$. So
	\[ \sum_{k=0}^nC_k(P^{-1}AP)^k=\sum_{k=0}^nP^{-1}A^kP=P^{-1}(\sum_{k=0}^nC_kA^k)P=P^{-1}f(A)P \]
	\end{proof}
	\end{enumerate}
			
\item[11.58]Let $A=\begin{bmatrix}[rr]2&-1\\-2&3\\\end{bmatrix}$
	\begin{enumerate}
	\item Find all eigenvalues and corresponding linearly independent eigenvectors.
	\[ \chi_A (\lambda) = \mathrm{det}(\lambda I_2-A) = \mathrm{det}\begin{bmatrix}[rr]\lambda - 2&-(-1)\\-(-2) & \lambda - 3\\\end{bmatrix} = (\lambda - 2)(\lambda - 3) - 2 = \lambda^2-5\lambda + 4=(\lambda - 1)(\lambda - 4) \]
	So our eigenvalues are 1 and 4.\\
	Let $\lambda=1$
	\[ \lambda I_2 - A = \begin{bmatrix}[rr]-1&1\\2&-2\\\end{bmatrix} \Rightarrow v_1=\begin{bmatrix}[r]1\\1\\\end{bmatrix} \]
	Let $\lambda=4$
	\[ \lambda I_2 - A=\begin{bmatrix}[rr]2&1\\2&1\\\end{bmatrix} \Rightarrow v_4=\begin{bmatrix}[r]1\\-2\\\end{bmatrix} \]
	\item Find a nonsingular matrix $P$ such that $D=P^{-1}AP$ is diagonal. \\
	Let $P=\begin{bmatrix}[rr]v_1&v_4\\\end{bmatrix}=\begin{bmatrix}[rr]1&1\\1&-2\\\end{bmatrix}$.\\
	Notice $P^{-1}AP=-\frac{1}{3}\begin{bmatrix}[rr]-2&-1\\-1&1\\\end{bmatrix}\begin{bmatrix}[rr]2&-1\\-2&3\\\end{bmatrix}\begin{bmatrix}[rr]1&1\\1&-2\\\end{bmatrix}=\begin{bmatrix}[rr]1&0\\0&4\\\end{bmatrix}$
	\item Find $A^8$ and $f(A)$ where $f(t)=t^4-5t^3+7t^2-2t+5$.\\
	From $P^{-1}AP=D$, we get $AP=PD$ and so $A=PDP^{-1}$. Then $A^8=(PDP^{-1})^8$. Expanding this formula, we can cancel all inner $P$ and $P^{-1}$ since multiplying those two together gives the identity matrix. Thus
	\[ A^8=PD^8P^{-1}=\begin{bmatrix}[rr]1&1\\1&-2\\\end{bmatrix}\begin{bmatrix}[rr]1&0\\0&4^8\\\end{bmatrix}\begin{bmatrix}[rr]2&1\\1&-1\\\end{bmatrix}\frac{1}{3} \]
	\[ f(A) = \begin{bmatrix}[rr]2&-1\\-2&3\\\end{bmatrix}^4-5\begin{bmatrix}[rr]2&-1\\-2&3\\\end{bmatrix}^3+7\begin{bmatrix}[rr]2&-1\\-2&3\\\end{bmatrix}^2-2\begin{bmatrix}[rr]2&-1\\-2&3\\\end{bmatrix}+5\begin{bmatrix}[rr]1&0\\0&1\\\end{bmatrix} \]
	\[ = \begin{bmatrix}[rr]2&-6\\2&9\\\end{bmatrix} \]
	\item Find a matrix $B$ such that $B^2=A$.\\
	Let $B=\begin{bmatrix}[rr]a&b\\c&d\\\end{bmatrix}$. Then expanding $B^2$ and setting it equal to matrix $A$, then solving the system gives us the matrix $B=\begin{bmatrix}[rr]\frac{4}{3}&-\frac{1}{3}\\-\frac{2}{3}&\frac{5}{3}\\\end{bmatrix}$.
	
	\end{enumerate}
	
\item[11.60]For each matrix, find all eigenvalues and maximum set $S$ of linearly independent eigenvectors:
	\[ A=\begin{bmatrix}[rrr]1&-3&3\\3&-5&3\\6&-6&4\\\end{bmatrix} \quad B=\begin{bmatrix}[rrr]3&-1&1\\7&-5&1\\6&-6&4\\\end{bmatrix} \quad  C=\begin{bmatrix}[rrr]1&2&2\\1&2&-1\\-1&1&4\\\end{bmatrix} \]
	Which matrices can be diagonalized and why?
	\begin{enumerate}
	\item $\chi_A (\lambda) = \mathrm{det}(\lambda I_3 - A) = \mathrm{det}\begin{bmatrix}[rrr]\lambda-1&3&-3\\-3&\lambda+5 & -3\\-6&6&\lambda - 4\\\end{bmatrix} = (\lambda+2)(\lambda-4)(\lambda+2)$\\
	Thus our eigenvalues are 4 and -2 with a multiplicity of 2.\\
	Let $\lambda=4$
	\[ \lambda I_3-A = \begin{bmatrix}[rrr]3&3&-3\\-3&9&-3\\-6&6&0\\\end{bmatrix} \]
	\[ v_4 = \begin{bmatrix}[r]\frac{1}{2}\\\frac{1}{2}\\1\\\end{bmatrix} \]
	Let $\lambda=-2$	
	\[ \lambda I_3 - A = \begin{bmatrix}[rrr]-3&3&-3\\-3&3&-3\\-6&6&-6\\\end{bmatrix} \]
	\[ u_{2} = \begin{bmatrix}[r]1\\1\\0\\\end{bmatrix}, v_2=\begin{bmatrix}[r]1\\0\\-1\\\end{bmatrix} \]
	\item We repeat the steps above to find our eigenvalues $\chi_B(\lambda)=(\lambda-4)(\lambda-2)(\lambda+4)$. Thus our eigenvalues are 4, -4, and 2.\\
	Let $\lambda=4$
	\[ \lambda I_3-A = \begin{bmatrix}[rrr]1&1&-1\\-7&9&-1\\-6&6&0\\\end{bmatrix} \xrightarrow[]{rref} \begin{bmatrix}[rrr]1&0&-\frac{1}{2}\\0&1&-\frac{1}{2}\\0&0&0\\\end{bmatrix} \]
	Thus our eigenvector is $\begin{bmatrix}[r]\frac{1}{2}\\\frac{1}{2}\\1\\\end{bmatrix}$.\\
	Let $\lambda=-4$
	\[ \lambda I_3-A = \begin{bmatrix}[rrr]-7&1&-1\\-7&1&-1\\-6&6&-8\\\end{bmatrix} \xrightarrow[]{rref} \begin{bmatrix}[rrr]1&0&-\frac{1}{18}\\0&1&-\frac{25}{18}\\0&0&0\\\end{bmatrix} \]
	Thus our eigenvector is $\begin{bmatrix}[r]\frac{1}{18}\\\frac{25}{18}\\1\\\end{bmatrix}$.\\
	Let $\lambda=2$
	\[ \lambda I_3-A=\begin{bmatrix}[rrr]-1&1&-1\\-7&7&-1\\-6&6&-2\\\end{bmatrix} \xrightarrow[]{rref} \begin{bmatrix}[rrr]1&-1&0\\0&0&1\\0&0&0\\\end{bmatrix} \]
	Thus our eigenvector is $\begin{bmatrix}[r]-1\\-1\\1\\\end{bmatrix}$.
	\item We repeat the steps above to find our eigenvalues $\chi_C(\lambda)=(\lambda-3)^2(\lambda-1)$.\\
	Thus our eigenvalues are 3 with a multiplicity of 2 and 1.
	Let $\lambda=3$
	\[ \lambda I_3-A=\begin{bmatrix}[rrr]2&-2&-2\\-1&1&1\\1&-1&-1\\\end{bmatrix} \xrightarrow[]{rref} \begin{bmatrix}[rrr]1&-1&-1\\0&0&0\\0&0&0\\\end{bmatrix} \]
	Thus our eigenvectors are $\begin{bmatrix}[r]1\\1\\0\\\end{bmatrix},\begin{bmatrix}[r]1\\0\\1\\\end{bmatrix}$.\\
	Let $\lambda=1$
	\[ \lambda I_3-A=\begin{bmatrix}[rrr]0&-2&-2\\-1&-1&1\\1&-1&-3\\\end{bmatrix}\xrightarrow[]{rref}\begin{bmatrix}[rrr]1&0&-2\\0&1&1\\0&0&0\\\end{bmatrix} \]
	Thus our eigenvector is $\begin{bmatrix}[r]2\\-1\\1\\\end{bmatrix}$.
	\end{enumerate}
\item[11.62]Let $A=\begin{bmatrix}[rr]a&b\\c&d\\\end{bmatrix}$ be a real matrix. Find necessary and sufficient conditions on $a,b,c,d$ so that $A$ is diagonalizable, that is, that $A$ has two (real) linearly independent eigenvectors.\\
	\begin{solution}
	We use the quadratic formula and $b^2-4ac$ along with $\Delta (t)=t^2-tr(A)t_det(A)$ to get $(-tr(A))^2-4(det(A)) > 0$. Thus since this is a $2\times 2$ matrix, we must have two eigenvalues which are linearly independent. Otherwise if they were linearly dependent, then this matrix would not be diagonalizable.
	\end{solution}
\item[11.63]Show that matrices $A$ and $A^T$ have the same eigenvalues. Give an example of a $2 \times 2$ matrix $A$ where $A$ and $A^T$ have different eigenvectors.\\
	\begin{solution}
	Assuming that our matrix has eigenvalues, then our matrix is diagonal. Thus since it is diagonal then taking the transpose of the matrix will not change the order of the values in its diagonal. So we have the formula $\Delta(t)=t^2-tr(A)+det(A)$ and our matrices and theorem 10.1 in the book stating that $det(A)=det(A)^T$, then our equations $t^2-tr(A)+det(A)$ and $t^2-tr(A^T)+det(A^T)$ are equal.\\
	Let $A=\begin{bmatrix}[rr]1&1\\0&1\\\end{bmatrix}$. The values in the diagonal can be any number so long as they are equal. The value in the top right can be any value as long as it is not zero and the value at the bottom left corner must be equal to zero.\\
	Let $\lambda=1$:
	\[ \lambda I-A = \begin{bmatrix}[rr]1&0\\0&1\\\end{bmatrix}-\begin{bmatrix}[rr]1&1\\0&1\\\end{bmatrix} = \begin{bmatrix}[rr]0&-1\\0&0\\\end{bmatrix} \]
	So
	\[ \begin{bmatrix}[rr]0&-1\\0&0\\\end{bmatrix}\begin{bmatrix}[r]v_1\\v_2\\\end{bmatrix} = \begin{bmatrix}[r]0\\0\\\end{bmatrix} \]
	Thus our vector $v$ can be $\begin{bmatrix}[r]1\\0\\\end{bmatrix}$ which satisfies the condition above.\\
	Next we find the eigenvector of $A^T=\begin{bmatrix}[rr]1&0\\1&1\\\end{bmatrix}$.\\
	let $\lambda=1$:
	\[ \lambda I-A=\begin{bmatrix}[rr]0&0\\-1&0\\\end{bmatrix} \]
	Thus our vector $v$ can be $\begin{bmatrix}[r]0\\1\\\end{bmatrix}$.\\
	Therefore our $2\times 2$ matrix $A$ where $A$ and $A^T$ have the same eigenvalue has a different eigenvector from its transpose.
	\end{solution}
	
\item[11.68]For each symmetric matrix $B$, find its eigenvalues, a maximal orthogonal set $S$ of eigenvectors, and an orthogonal matrix $P$ such that $D=P^{-1}BP$ is diagonal:
	\begin{enumerate}
	\item $B=\begin{bmatrix}[rrr]0&1&1\\1&0&1\\1&1&0\\\end{bmatrix}$\\
	\[ X_B=\mathrm{det}\begin{bmatrix}[rrr]\lambda&-1&-1\\-1&\lambda&-1\\-1&-1&\lambda\\\end{bmatrix} = (\lambda+1)^2(\lambda-2) \]
	Thus our eigenvalues are -1 with a multiplicity of 2 and 2 with a multiplicity of 1.\\
	Let $\lambda=-1$, then
	\[ \lambda I-A=\begin{bmatrix}[rrr]-1&-1&-1\\-1&-1&-1\\-1&-1&-1\\\end{bmatrix}\xrightarrow[]{rref} \begin{bmatrix}[rrr]-1&-1&-1\\0&0&0\\0&0&0\\\end{bmatrix} \]
	Thus our eigenvectors are $v_1=\begin{bmatrix}[r]1\\-1\\0\\\end{bmatrix},v_2=\begin{bmatrix}[r]1\\1\\-2\\\end{bmatrix}$.\\
	Let $\lambda=2$, then
	\[ \lambda I-A = \begin{bmatrix}[rrr]2&-1&-1\\-1&2&-1\\-1&-1&2\\\end{bmatrix} \xrightarrow[]{rref}\begin{bmatrix}[rrr]1&0&-1\\0&1&-1\\0&0&0\\\end{bmatrix} \]
	Thus our eigenvector is $v_3=\begin{bmatrix}[r]1\\1\\1\\\end{bmatrix}$.\\
	Next we normalize our eigenvectors $v_1,v_2,v_3$ to get our orthonormal basis $U$.
	\[ u_1=(\frac{1}{\sqrt{2}},-\frac{1}{\sqrt{2}},0),\quad u_2=(\frac{1}{\sqrt{6}},\frac{1}{\sqrt{6}},\-\frac{2}{\sqrt{6}}),\quad u_3=(\frac{1}{\sqrt{3}},\frac{1}{\sqrt{3}},\frac{1}{\sqrt{3}}) \]
	Let $P$ be a matrix whose columns are $u_1,u_2,u_3$, then
	\[ P^{-1}BP = \begin{bmatrix}[rrr]0&-\frac{3}{2}&0\\\frac{1}{6}&0&\frac{1}{2}\\0&4&0\\\end{bmatrix} \]
	
	\item $B=\begin{bmatrix}[rrr]2&2&4\\2&5&8\\4&8&17\\\end{bmatrix}$
	\[ \chi_B=\mathrm{det}\begin{bmatrix}[rrr]\lambda-2&-2&-4\\-2&\lambda-5&-8\\-4&-8&\lambda-17\\\end{bmatrix} = (\lambda-1)(\lambda-1)(\lambda-22) \]
	Thus our eigenvalues are 1 with a multiplicity of 2, and 22.\\
	Let $\lambda=1$, then
	\[ \lambda I-A=\begin{bmatrix}[rrr]-1&-2&-4\\-2&-4&-8\\-4&-8&-16\\\end{bmatrix}\xrightarrow[]{rref}\begin{bmatrix}[rrr]1&2&4\\0&0&0\\0&0&0\\\end{bmatrix} \]
	Thus our eigenvectors are $v_1=\begin{bmatrix}[r]-2\\1\\0\\\end{bmatrix},v_2=\begin{bmatrix}[r]-4\\0\\1\\\end{bmatrix}$.\\
	Let $\lambda=22$, then
	\[ \lambda I-B = \begin{bmatrix}[rrr]20&-2&-4\\-2&17&-8\\-4&-8&5\\\end{bmatrix}\xrightarrow[]{rref}\begin{bmatrix}[rrr]1&0&-\frac{1}{4}\\0&1&-\frac{1}{2}\\0&0&0\\\end{bmatrix} \]
	Thus our eigenvector is $v_3=\begin{bmatrix}[r]\frac{1}{4}\\\frac{1}{2}\\1\\\end{bmatrix}$.\\
	Next we normalize our eigenvectors $v_1,v_2,v_3$ to get our orthonormal basis $U$.
	\[ u_1=(-\frac{2}{\sqrt{5}},\frac{1}{\sqrt{5}},0),\quad u_2=(-\frac{4}{\sqrt{17}},0,\frac{1}{\sqrt{17}}),\quad u_3=(\frac{\sqrt{\frac{21}{16}}}{4},\frac{\sqrt{\frac{21}{16}}}{2},\frac{1}{\sqrt{\frac{21}{16}}}) \]
	\end{enumerate}
\item[11.69]Find a real $2 \times 2$ symmetric matrix $A$ with eigenvalues $\lambda = 1$ and $\lambda = 4$ such that $u=(1,1)$ is an eigenvector belonging to $\lambda=1$. Also find a matrix $B$ for which $B^2=A$.
	\begin{solution}
	Since $A$ is symmetric, then a vector $v$ must be orthogonal to $u$. Set $v=(-1,1)^T$.\\
	Let $P$ be the matrix whose columns are the eigenvectors $u$ and $v$.
	\[ A=PDP^{-1}=\begin{bmatrix}[rr]1&-1\\1&1\\\end{bmatrix}\begin{bmatrix}[rr]1&0\\0&4\\\end{bmatrix}\begin{bmatrix}[rr]\frac{1}{2}&\frac{1}{2}\\-\frac{1}{2}&\frac{1}{2}\\\end{bmatrix} =  \begin{bmatrix}[rr]\frac{5}{2}&-\frac{3}{2}\\-\frac{3}{2}&\frac{5}{2}\\\end{bmatrix} \]
	\end{solution}
\item[11.65]Suppose $v$ is an eigenvector of a linear operator $T$ belonging to the eigenvalue $\lambda$. Prove:
	\begin{enumerate}
	\item For $n>0$, $v$ is an eigenvector of $T^n$ belonging to $\lambda^n$.
	\begin{proof}
Assume $v\neq 0$ and $Tv=\lambda v$.
Let $n=0$ then $T^0v=Iv=v$ and $\lambda^0v=1v=v$. Now assume $n$ is fixes and $T^nv=\lambda^nv$. Apply $T$, then 
\[T^{n+1}v=T(T^nv)=T(\lambda^nv)=\lambda^nT(v) = \lambda^n(\lambda v)=\lambda^{n+1}v \]
Thus by mathematical induction, $T^nv=\lambda^nv$ for all $n > 0$.
\end{proof}
	\item $f(\lambda)$ is an eigenvalue of $f(T)$ for any polynomial $f(t)$.
	\begin{proof}
Let $f$ be a polynomial. Then there exist scalars $c_0,c_1,...,c_n$ such that $f(x)=\sum_{k=0}^nc_kx^k$.\\
So by using our proof from part (a),
\[ f(x)=\sum_{k=0}^nc_k\lambda^kv \]
Then by the definition of $f$, this sum is equal to $f(\lambda)v$.\\
Thus $v$ is an eigenvector for the $f(t)$ with the eigenvalue $f(\lambda)$.
\end{proof}
	\end{enumerate}
	
\end{enumerate}	
\end{document}