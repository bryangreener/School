\documentclass[12pt]{article}
\usepackage[margin=1in]{geometry} 
\usepackage{amsmath,amsthm,amssymb,amsfonts}
\usepackage{tabto}
\usepackage{hyperref}

% Spacers:
% BEGIN BLOCK------------------------------------------
% END BLOCK============================================




\newcommand{\N}{\mathbb{N}}
\newcommand{\Z}{\mathbb{Z}}

% CUSTOM SETTINGS
% BEGIN BLOCK------------------------------------------
% For equation system alignment
\usepackage{systeme,mathtools}
% Usage:
%	\[
%	\sysdelim.\}\systeme{
%	3z +y = 10,
%	x + y +  z = 6,
%	3y - z = 13}


% For definitions
\newtheorem{defn}{Definition}[section]
\newtheorem{thrm}{Theorem}[section]

% For circled text
\usepackage{tikz}
\newcommand*\circled[1]{\tikz[baseline=(char.base)]{
            \node[shape=circle,draw,inner sep=0.8pt] (char) {#1};}}

\newenvironment{problem}[2][Problem]{\begin{trivlist}
\item[\hskip \labelsep {\bfseries #1}\hskip \labelsep {\bfseries #2.}]}{\end{trivlist}}
%If you want to title your bold things something different just make another thing exactly like this but replace "problem" with the name of the thing you want, like theorem or lemma or whatever
 
%used for matrix vertical line
\makeatletter
\renewcommand*\env@matrix[1][*\c@MaxMatrixCols c]{%
  \hskip -\arraycolsep
  \let\@ifnextchar\new@ifnextchar
  \array{#1}}
\makeatother 

% END BLOCK============================================

\newtheorem*{lemma}{Lemma} %added
\newtheorem*{result}{Result} %added
\newtheorem*{theorem}{Theorem} %added


% HEADER
% BEGIN BLOCK------------------------------------------
\usepackage{fancyhdr}
 
\pagestyle{fancy}
\fancyhf{}
\lhead{Homework \#7}
\rhead{Bryan Greener}
\cfoot{\thepage}
% END BLOCK============================================

% TITLE
% BEGIN BLOCK------------------------------------------
\title{Bryan Greener}
\author{MATH 2300 CRN:15163}
\date{2018-02-14}
\begin{document}
\maketitle
% END BLOCK============================================

\TabPositions{4cm}

\begin{enumerate}
\item[5.68] Find the rank of each matrix:
	\begin{enumerate}
	%5.68.a
	\item $\begin{bmatrix}[rrrrr]1&3&-2&5&4\\1&4&1&3&5\\1&4&2&4&3\\2&7&-3&6&13\\\end{bmatrix}$
	\[ \mathrm{rref}\begin{bmatrix}[rrrrr]1&3&-2&5&4\\1&4&1&3&5\\1&4&2&4&3\\2&7&-3&6&13\\\end{bmatrix} = \begin{bmatrix}[rrrrr]1&0&0&22&-21\\0&1&0&-5&7\\0&0&1&1&-2\\0&0&0&0&0\\\end{bmatrix} \]
	Since we get 1 zero row, this shows linear independence and dim $\mathrm{rowspace}(A)=3$. Thus our rank is 3.
	%5.68.b	
	\item $\begin{bmatrix}[rrrrr]1&2&-3&-2&-3\\1&3&-2&0&-4\\3&8&-7&-2&-11\\2&1&-9&-10&-3\\\end{bmatrix}$
	\[ \mathrm{rref}\begin{bmatrix}[rrrrr]1&2&-3&-2&-3\\1&3&-2&0&-4\\3&8&-7&-2&-11\\2&1&-9&-10&-3\\\end{bmatrix} = \begin{bmatrix}[rrrrr]1&0&-5&-6&-1\\0&1&1&2&-1\\0&0&0&0&0\\0&0&0&0&0\\\end{bmatrix} \]
	Since we get 2 zero rows, this shows linear independence and dim $\mathrm{rowspace}(A)=2$. Thus our rank is 2.
	%5.68.c	
	\item $\begin{bmatrix}[rrr]1&1&2\\4&5&5\\5&8&1\\-1&-2&-2\\\end{bmatrix}$
	\[ \mathrm{rref}\begin{bmatrix}[rrr]1&1&2\\4&5&5\\5&8&1\\-1&-2&-2\\\end{bmatrix} = \begin{bmatrix}[rrr]1&0&0\\0&1&0\\0&0&1\\0&0&0\\\end{bmatrix} \]
	Since we get 1 zero row, this shows linearly independence and dim $\mathrm{rowspace}(A)=3$. Thus our rank is 3.
	%5.68.d	
	\item $\begin{bmatrix}[rr]2&1\\3&-7\\-6&1\\5&-8\\\end{bmatrix}$
	\[ \mathrm{rref}\begin{bmatrix}[rr]2&1\\3&-7\\-6&1\\5&-8\\\end{bmatrix} = \begin{bmatrix}[rr]1&\frac{1}{2}\\0&1\\0&0\\0&0\\\end{bmatrix} \]
	Since we get 2 zero rows, this shows linear independence and dim $\mathrm{rowspace}(A)=2$. Thus our rank is 2.
	\end{enumerate}

\item[5.70] For each of the following matrices, find (i) columns that are linear combinations of preceding columns, and (ii) columns that form a basis for the column space:
	\begin{enumerate}
	%5.70.a	
	\item $A=\begin{bmatrix}[rrrrrr]1&1&2&2&3&3\\2&2&5&6&8&7\\1&1&4&6&7&7\\\end{bmatrix}$
		\begin{enumerate}
		%5.70.a.i		
		\item
		\[ \mathrm{rref}(A^T) = \begin{bmatrix}[rrr]1&0&0\\0&1&0\\0&0&1\\0&0&0\\0&0&0\\0&0&0\\\end{bmatrix} \]
		This leaves us with the zero rows originally being rows $R_2$, $R_4$, and $R_5$. Thus in our original matrix, we get the columns $\{C_2,C_4,C_5\}$ that are linear combinations of preceding columns. 
		%5.70.a.ii		
		\item From the previous step, we know that columns $C_1$, $C_3$, and $C_6$ are linearly independent. Therefore we get the basis $\{C_1,C_3,C_6\}$.
		\end{enumerate}			
	%5.70.b	
	\item $B=\begin{bmatrix}[rrrrrr]1&1&2&2&3&3\\2&3&5&8&10&7\\1&3&4&11&11&6\\\end{bmatrix}$
		\begin{enumerate}
		%5.70.b.i		
		\item
		\[ \mathrm{rref}(B_T)=\begin{bmatrix}[rrr]1&0&0\\0&1&0\\0&0&1\\0&0&0\\0&0&0\\0&0&0\\\end{bmatrix}\]
		This leaves us with the zero rows originally being rows $R_3$, $R_5$, and $R_6$. Thus in our original matrix, we get the columns $\{C_3,C_5,C_6\}$ that are linear combinations of preceding columns.
		%5.70.b.ii
		\item From the previous step, we know that columns $C_1$, $C_2$, and $C_4$ are linearly independent. Therefore we get the basis $\{C_1,C_2,C_4\}$.
		\end{enumerate}			
	\end{enumerate}

\item[5.72] Find a basis for (i) the row space and (ii) the column space of each matrix $M$:
	\begin{enumerate}
	%5.72.a	
	\item $M=\begin{bmatrix}[rrrrr]0&0&3&1&4\\1&3&1&2&1\\3&9&4&5&2\\4&12&8&8&7\\\end{bmatrix}$
		\begin{enumerate}
		%5.72.a.i
		\item
		\[ \mathrm{rref}(M^T) = \begin{bmatrix}[rrrr]1&0&0&1\\0&1&0&1\\0&0&1&1\\0&0&0&0\\0&0&0&0\\\end{bmatrix} \]
		Since this is the transpose of $M$, we need to take the transpose of the result which gives us our basis for the row space:
		\[ \{\begin{bmatrix}[r]1\\0\\0\\1\\\end{bmatrix},\begin{bmatrix}[r]0\\1\\0\\1\\\end{bmatrix},\begin{bmatrix}[r]0\\0\\1\\1\\\end{bmatrix}\} \]
		%5.72.a.ii
		\item Using our resultant matrix from 5.72.a.i, we get the basis for the column space:
		\[ \{C_1,C_3,C_4\} \] 
		\end{enumerate}
	%5.72.b
	\item $M=\begin{bmatrix}[rrrrr]1&2&1&0&1\\1&2&2&1&3\\3&6&5&2&7\\2&4&1&-1&0\\\end{bmatrix}$
		\begin{enumerate}
		%5.72.b.i		
		\item
		\[ \mathrm{rref}(M^T)=\begin{bmatrix}[rrrr]1&0&1&3\\0&1&2&-1\\0&0&0&0\\0&0&0&0\\0&0&0&0\\\end{bmatrix} \]
		Since this is the transpose of $M$, we need to take the transpose of the result which gives us our basis for the row space:
		\[ \{\begin{bmatrix}[r]1\\0\\1\\3\\\end{bmatrix},\begin{bmatrix}[r]0\\1\\2\\-1\\\end{bmatrix}\} \]
		%5.72.b.ii
		\item Using our resultant matrix from 5.72.b.i, we get the basis for the column space:
		\[ \{C_1,C_3\} \]
		\end{enumerate}
	\end{enumerate}

\item[5.74] Let $r=rank(A+B)$. Find 2 x 2 matrices $A$ and $B$ such that: (a) $r<rank(A),rank(B)$; (b) $r=rank(A)=rank(B)$; (c) $r>rank(A),rank(B)$.
	\begin{enumerate}
	%5.74.a	
	\item Let $A=\begin{bmatrix}[rr]1&0\\0&1\\\end{bmatrix}$ and $B=\begin{bmatrix}[rr]-1&0\\0&-1\\\end{bmatrix}$. Then $\mathrm{rank}(A)=2$ and $\mathrm{rank}(B) = 2$. So
	\[ A+B = \begin{bmatrix}[rr]1&0\\0&1\\\end{bmatrix}+\begin{bmatrix}[rr]-1&0\\0&-1\\\end{bmatrix} = \begin{bmatrix}[rr]0&0\\0&0\\\end{bmatrix} \]
	Therefore $r = \mathrm{rank}(A+B) = 0$ and so $r<\mathrm{rank}(A),\mathrm{rank}(B)$.
	%5.74.b
	\item Let $A=\begin{bmatrix}[rr]1&0\\0&0\\\end{bmatrix}$ and $B=\begin{bmatrix}[rr]0&1\\0&0\\\end{bmatrix}$. Then $\mathrm{rank}(A) = 1$ and $\mathrm{rank}(B) = 1$. So
	\[ A+B = \begin{bmatrix}[rr]1&0\\0&0\\\end{bmatrix}+\begin{bmatrix}[rr]0&1\\0&0\\\end{bmatrix} = \begin{bmatrix}[rr]1&1\\0&0\\\end{bmatrix} \]
	Therefore $r=\mathrm{rank}(A+B)=1$ and so $r=\mathrm{rank}(A),\mathrm{rank}(B)$.
	%5.74.c
	\item Let $A=\begin{bmatrix}[rr]1&0\\0&0\\\end{bmatrix}$ and $B=\begin{bmatrix}[rr]0&0\\0&1\\\end{bmatrix}$. Then $\mathrm{rank}(A) = 1$ and $\mathrm{rank}(B) = 1$. So
	\[ A+B = \begin{bmatrix}[rr]1&0\\0&0\\\end{bmatrix}+\begin{bmatrix}[rr]0&0\\0&1\\\end{bmatrix}=\begin{bmatrix}[rr]1&0\\0&1\\\end{bmatrix} \]
	Therefore $r=\mathrm{rank}(A+B)=2$ and so $r>\mathrm{rank}(A),\mathrm{rank}(B)$.
	\end{enumerate}

\item[5.75] Let $U$ and $W$ be subspaces of $\mathbb{R}^3$ for which $\mathrm{dim}U=4$, $\mathrm{dim}W=5$, and $\mathrm{dim}V=7$. Find the possible dimensions of $U \cap W$.\\
	By the given dimensions, $5\leq (U+W) \leq 7$. Thus there are three possible solutions: 5, 6, or 7. By theorem 5.9 in our book,
	\[ \mathrm{dim}(U\cap W) = \mathrm{dim}U+\mathrm{dim}W-\mathrm{dim}(U+W)=9-\mathrm{dim}(U+W) \]
	So
	\begin{align*}
	&\mathrm{dim}(U\cap W) = 9 - 5 = 4\\
	\mathrm{or} \quad &\mathrm{dim}(U\cap W) = 9 - 6 = 3\\
	\mathrm{or} \quad &\mathrm{dim}(U\cap W) = 9 - 7 = 2\\
	\end{align*}
	Thus our possible solutions are 2, 3, or 4.

\item[5.77] Consider the following subspaces of $\mathbb{R}^5$:
	\begin{align*}
		U &= span[(1,3,-3,-1,-4),(1,4,-1,-2,-2),(2,9,0,-5,-2)]\\
		W &= span[(1,6,2,-2,3),(2,8,-1,-6,-5),(1,3,-1,-5,-6)]
	\end{align*}
	Find: (a) $\mathrm{dim}(U+W)$; (b) $\mathrm{dim}(U\cap W)$.
	\begin{enumerate}
	%5.77.a	
	\item 
	\[ \mathrm{rref}(\mathrm{rowspace}(U)) = \begin{bmatrix}[rrrrr]1&0&-9&2&10\\0&1&2&-1&2\\0&0&0&0&0\\\end{bmatrix} \]
	\[ \mathrm{rref}(\mathrm{rowspace}(W)) = \begin{bmatrix}[rrrrr]1&0&0&-16&-19\\0&1&0&3&4\\0&0&1&-1&-1\\\end{bmatrix} \]
	Thus we get dim$U = 2$ and dim$W=3$. We need to find dim$(U+W)$. To do this, we combine the two matrices.
	\[ \mathrm{rref}\begin{bmatrix}[rrrrr]1&3&-3&-1&-4\\1&4&-1&-2&-2\\2&9&0&-5&-2\\1&6&2&-2&3\\2&8&-1&-6&-5\\1&3&-1&-5&-6\\\end{bmatrix} = \begin{bmatrix}[rrrrr]1&0&0&-16&-19\\0&1&0&3&4\\0&0&1&-2&-1\\0&0&0&0&0\\0&0&0&0&0\\0&0&0&0&0\\\end{bmatrix} \]
	Thus we get dim$(U+W)=3$.
	%5.77.b
	\item dim$(U\cap W) = \mathrm{dim}(U)+\mathrm{dim}(W)-\mathrm{dim}(U+W) = 2+3-3=2$.\\
	Thus dim$(U\cap W) = 2$.
	\end{enumerate}

\item[5.80] Answer true or false. If false, prove it with a counterexample.
	\begin{enumerate}
	%5.80.a
	\item If $u_1,u_2,u_3$ span $V$, then $\mathrm{dim}V = 3$.\\
	False. Any time that any of $u_1$, $u_2$, or $u_3$ are a linear combination of another, then the dimension could vary.
	%5.80.b
	\item If $A$ is a 4 x 8 matrix, then any six columns are linearly dependent.\\
	Book says true but I feel like it isn't right.
	%5.80.c	
	\item If $u_1,u_2,u_3$ are linearly independent, then $u_1,u_2,u_3,w$ are linearly dependent.\\
	False. If we set these variables to be a 4 x 4 identity matrix then we get linear independence.
	%5.80.d
	\item If $u_1,u_2,u_3,u_4$ are linearly independent, then $\mathrm{dim}V \geq 4$.\\
	True.
	%5.80.e	
	\item If $u_1,u_2,u_3$ span $V$, then $w,u_1,u_2,u_3$ span $V$.\\
	True.
	\end{enumerate}

\item[5.81] Answer true or false. If false, prove it with a counterexample.
	\begin{enumerate}
	%5.81.a	
	\item If any row is deleted from a matrix in echelon form, then the resulting matrix is still in echelon form.\\
	True.
	%5.81.b	
	\item If any column is deleted from a matrix in echelon form, then the resulting matrix is still in echelon form.\\
	True.
	%5.81.c	
	\item If any row is deleted from a matrix in row canonical form, then the resulting matrix is still in row canonical form.\\
	True.
	%5.81.d	
	\item If any column is deleted from a matrix in row canonical form, then the resulting matrix is still in row canonical form.\\
	False. If deleting a column results in a column where it is not the only nonzero entry, then the matrix is no longer in row canonical form.
	%5.81.e	
	\item If any column without a pivot is deleted from a matrix in row canonical form, then the resulting matrix is in row canonical form.\\
	True.
	\end{enumerate}

\item[5.82] Determine the dimension of the vector space $W$ of the following n-square matrices: (a) symmetric matrices; (b) antisymmetric matrices; (c) diagonal matrices; (d) scalar matrices.
	\begin{enumerate}
	%5.82.a
	\item $n(n+1)$
	%5.82.b
	\item $n(n-1)$
	%5.82.c
	\item $n$
	%5.82.d
	\item $1$
	\end{enumerate}
	
\item[5.79] Suppose $V=U\oplus W$. Show that $\mathrm{dim}V=\mathrm{dim}U+\mathrm{dim}W$.
	\begin{theorem}[5.79.1]
	Let $V$ be a finite dimensional vector space such that $U_1,U_2,...,U_m$ are subspaces of $V$ and $V=U_1\oplus U_2\oplus \dots \oplus U_m$. Then $\mathrm{dim}(V) = \mathrm{dim}(U_1)+\mathrm{dim}(U_2)+\dots+\mathrm{dim}(U_m)$.
	\end{theorem}
	\begin{proof}
	Let $B_1,B_2,...,B_m$ be bases of $U_1,U_2,...,U_m$. Let $B=U_1\cup U_2\cup \dots \cup U_m$. Note for every vector $v\in V$ that $v=u_1+u_2+\dots u_m$ where $u_1\in U_1,u_2\in U_2,...,u_m\in U_m$. So for each $u_i$ where $i=1,2,..,m$, $u_i$ is a linear combination of the vectors in $B_i$. Thus $B $ is a spanning set of $V$.\\
	Next we show that $B$ is linearly independent. Assume that a linear combination of $B$ is zero. Since $V=U_1\oplus U_2 \oplus \dots \oplus U_m$, then each group of bases equals zero. Thus $B_1,B_2,...,B_m$ are bases of each subspace and the coefficients all equal zero. Therefore $B$ is linearly independent.\\
	Since $B$ is linearly independent and has spans $V$ then $B$ is a basis of $V$. So
	\[ \mathrm{dim}(V) = \mathrm{dim}(U_1)+\mathrm{dim}(U_2)+\dots +\mathrm{dim}(U_m) \]
	Thus for our case, $\mathrm{dim}(V) = \mathrm{dim}(U) + \mathrm{dim}(W)$.
	\end{proof}
	Source:\\
	\url{http://mathonline.wikidot.com/the-dimension-of-a-direct-sum-of-subspaces}
	
	NOTE:\\
	We can get this solution by taking the theorem that $(U\cap W) = \{\vec{0}\}$. From this, we get $dim(U\oplus W) + dim(U\cap W) = dim(U) + dim(W)$. So then $dim(U\oplus W) + dim(\{\vec{0}\} = dim(U\oplus W) + 0 = dim(U)+dim(W)$.\\
	See class notes for 2018-02-16 for further detail.
\end{enumerate}
\end{document}