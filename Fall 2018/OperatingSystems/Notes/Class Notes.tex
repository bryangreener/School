\documentclass{report}
\usepackage[margin=1in]{geometry} 
\usepackage{amsmath,amsthm,amssymb,amsfonts}
\usepackage{tabto}
\usepackage[yyyymmdd]{datetime}
\renewcommand{\dateseparator}{--}
\newcommand{\N}{\mathbb{N}}
\newcommand{\Z}{\mathbb{Z}}
\usepackage[super]{nth}
\setlength{\parindent}{0pt}

% For definitions
%\newtheorem{defn}{Definition}[section]
%\newtheorem{thrm}{Theorem}[section]
%\newtheorem{ex}[Example}[section]
\newtheorem*{ex}{Example}
\newtheorem*{defn}{Definition}
\newtheorem*{thrm}{Theorem}
\newtheorem*{lemma}{Lemma}
\newtheorem*{result}{Result}

% For circled text
\usepackage{tikz}
\newcommand*\circled[1]{\tikz[baseline=(char.base)]{
            \node[shape=circle,draw,inner sep=0.8pt] (char) {#1};}}

\usepackage{pgf}
\usetikzlibrary{arrows, automata}

% For equation system alignment
\usepackage{systeme,mathtools}
% Usage:
%	\[
%	\sysdelim.\}\systeme{
%	3z +y = 10,
%	x + y +  z = 6,
%	3y - z = 13}

\newenvironment{problem}[2][Problem]{\begin{trivlist}
\item[\hskip \labelsep {\bfseries #1}\hskip \labelsep {\bfseries #2.}]}{\end{trivlist}}
%If you want to title your bold things something different just make another thing exactly like this but replace "problem" with the name of the thing you want, like theorem or lemma or whatever
 
%used for matrix vertical line
\makeatletter
\renewcommand*\env@matrix[1][*\c@MaxMatrixCols c]{%
  \hskip -\arraycolsep
  \let\@ifnextchar\new@ifnextchar
  \array{#1}}
\makeatother 
 
% Change chapter numbering
\newcommand{\mychapter}[2]{
	\setcounter{chapter}{#1}
	\setcounter{section}{0}
	\chapter*{#2}
	\addcontentsline{toc}{chapter}{#2}
}

\usepackage{graphicx}
\graphicspath{ {images/} }

\begin{document}
 
\tableofcontents{}
\mychapter{1}{2018-09-13}
\section{Interrupts}
\subsection{Scheduling}
Assume we have only one core to work with. The kernel is keeping track of its processes. The process can be running in the CPU or it can be waiting for an interrupt or it can be in a queue of processes that are ready to be sent to the CPU.

\begin{defn}[Time Sharing]
Every process gets a set amount of time allowed in the CPU before it is removed from the CPU and lets the next item in the ready queue use the CPU. 70\% of the time, processes leave the CPU voluntarily to go into the 'waiting for interrupt' queue.
\end{defn}
\noindent
The timer used by the CPU to determine the allowed time is called the \textit{Time Quantum}. Items kicked out by the time quantum get sent back to the end of the ready queue. 

\mychapter{2}{2018-09-25}
\section{Exam 1 Review}

\begin{enumerate}
\item[1.]Discuss the differences between user space and kernel space including security/rights as well as memory access.\\
	The kernel is a different mode in the CPU and can access anywhere in the computer. Thus the kernel has no security at all.

\item[2.]Where in the OS system levels does a device driver usually sit? User or Kernel and why?\\
	Device drivers sit in the kernel since they have to be fast and the kernel calls them.

\item[3.]Why are device drivers a threat to the security of a computer?\\
	Since drivers run in the kernel, and the kernel is not secure, then it has unrestricted access to the whole computer.

\item[4.]What is the role of a device driver in terms of the device's hardware/software and the OS code? Does the kernel call different functions for each device of the same type?\\
	


\item[5.]Describe the way each OS deals with device drivers as far as not having viruses and doing their job correctly.\\
	\begin{enumerate}
	\item[a] Linux\\
			The Linux user base tests and certifies drivers. The driver is the put into a trusted repository where users can download drivers checked out by other users. When you download a driver, you're asked to go to a different source and run a CRC (Cyclical Redundancy Check) check which is similar to comparing checksums.
	\item[b] Apple\\
			Only allows device drivers approved by Apple software developers (which is a title that you can get for little money).
	\item[c] MS-Windows\\
			Drivers must be certified by MS people and they must have an SSH Certificate in order to be loaded.
	\end{enumerate}

\item[6.]Explain why a program cannot be optimized when using a debugger to single step through the source code.\\
	Programs cannot be optimized when debugging since the debugger will change the machine code that is translated from the program.

\item[7]
	\begin{enumerate}
	\item What does the static key word do to x?\\
		Static stores the variable in the data section of memory.
	\item Where is x stored?\\
		x is stored in the data section as mentioned in the previous part of this problem.
	\item What value is returned the second time func is called?\\
		25 is returned. Even though static int x is declared inside the function every time func is called, the value is stored elsewhere and is not re-initialized every function call.
	\end{enumerate}

\item[9]Explain the difference between the memory types shared by processes and memory shared by threads.\\
	When a thread is spawned, the only thing not shared by other things is the stack, thus that section of stack used by the process is unique to that process. All processes share the heap, data, and text.
	
\item[10]The CPU's MMU translates virtual memory addresses to real memory addresses.
	\begin{enumerate}
	\item The MMU uses a table to manage the translation, how does this table get its information?\\
		The MMU gets its information from the Kernel and tells the MMU where each segment is.
	\item[c.] What happens if a process asks for memory addresses that are not loaded into real memory and the real memory is full?\\
		The Kernel will take segments out of real memory and saves them to the hard disk then loads what is requested. This also throws a page fault.
	\end{enumerate}

\item[13.]Why do you need to use a remote debugger to single step debug a kernel?\\
If you stop code while single stepping then the debugger won't be running since the kernel won't be running.

\item[14.]Write the body for myStrcpy using array notation like dest[i] and pointer arithmetic like src++, one form on each size of =\\
	\begin{verbatim}
	void myStrcpy(char *dest, char *src[]){
	  while(*dest++ = *src++);
	}
	\end{verbatim}

\item[15.]The parameters for myStrcpy are dest and src. What is the difference in their data types?\\
	There is no difference between src and dest in terms of data types in myStrcpy.

\item[16.]If the file q1.c is in the current working directory and no makefile show the make command to compile and create q1.\\
	Tricky question wording. It just wanted the command to make the program given a makefile is provided so the answer would be
	\begin{verbatim}
	user@comp:~/$ make program_name
	\end{verbatim}

\item[20.]
	\begin{verbatim}
	for(int i = 0; i < c; i++){ printf("%s\n", f[i]); }
	\end{verbatim}


\end{enumerate}

\mychapter{3}{2018-09-27}
\section{Schedulers}
\subsection{Scheduler Systems and Queues}
Types of Scheduler Systems:
\begin{itemize}
	\item Servers
	\item Users
	\item Small Devices and Embedded Systems
	\item Real Time
\end{itemize}

\noindent
Normally processes leave the CPU and go into the ready queue because they are preempted or they time out. The scheduler determines which items from the ready queue go into the CPU next.\\

\noindent
Types of structures for the ready queue:
\begin{itemize}
	\item FIFO Queue.
	\item Take shortest time task first. Problem is that you need to be able to predict ahead of time how long a process will take.
	\item Priority Queue. Every process has priority and highest priority runs first. Low priority processes rarely get run with these types of structures since they never get to the front of the queue.
\end{itemize}

\subsection{Priority Queues}
Some priority queues use \textit{aging} to allow low priority processes to make their way out of the queue. Most operating systems use this system along with the time quantum in the CPU to process the ready queue.\\

\begin{defn}[Thrashing] Thrashing is when switching processes is done too often due to too small of a time quantum. This ends up hindering overall performance.
\end{defn}

\subsection{Multilevel Queues}
Many modern CPUs use multilevel queues where process priority queues are based on the process type in order from highest to lowest priority:
\begin{enumerate}
	\item Real-time processes
	\item System processes
	\item Interactive processes
	\item Batch processes
\end{enumerate}

\subsection{Real-Time Scheduling}
Latency has to be accounted for with real time systems. There are two types of latencies that affect performance:

\begin{defn}[Interrupt Latency]
	Time from arrival of interrupt to start of routine that services interrupt.
\end{defn}

\begin{defn}[Dispatch Latency]
	Time for schedule to take current process off CPU and switch to another.
\end{defn}

\noindent
Real-time scheduling must support preemptive priority based scheduling.

\mychapter{4}{2018-10-16}
\section{Exam Review}
\begin{enumerate}
	\item[1.]If you double the number of processors working on a problem, does the speed double?\\
	No because there is a delay in processing information.
	
	\item[2.] What is the biggest challenge for Computer Scientists with concurrent processing?\\
Designing the algorithms that handle concurrent processing.

	\item[7.]What does it mean to preempt a process in the CPU?\\
	Preempt means to take a process out because there is a more important process that came up.
	
	\item[8.b.1]What type of OS uses shortest future run time/shortest job first?\\
	In real-time systems you know ahead of time how long each process will take.

	\item[11.]How does hyper-threading work and how should the scheduler use them?\\
	Hyper-threading is where one core can have multiple threads. Hyper-threading only works when both threads are from the same process and are sharing everything except the stack. When one thread is waiting for memory access the other thread will run. The scheduler should make sure that threads from the same process are in the same processor. The term ''Hyper-threading'' is a term created by Intel.
	
	\item[13.]Describe L1, L2, and L3 caches and why they are important.\\
	Caches pre-fetch instructions and data from processes to allow the CPU to operate faster. This is as opposed to the CPU going to main memory to fetch information. Caches are more important than raw CPU speed for performance of a CPU.
	
	\item[14.]What is processor affinity and why is it important with levels of caches?\\
	Processor affinity is when the scheduler tries to keep processes together on the same CPU so it can take advantage of what has already been pre-fetched into the cache. So if running a process with 5 threads, you'd want to keep it all in one CPU core. But without processor affinity, the CPU will try to split this process up between other cores if they are available.
	
	\item[15.]When designing scheduling for an OS, should they be fixed or dynamic and why?\\
	They should be dynamic in order to be flexible in different situations.
\end{enumerate}

\section{Concurrent Processing}
See chapter 4 in \nth{10} edition and chapter 5 in \nth{9} edition of textbook.\\

\subsection{Process Synchronization}
\begin{defn}[Concurrent Processing] Concurrent Processing is when there are multiple processes that need to communicate with each other or multiple threads that need to share data.
\end{defn}
\noindent
One of the primary goals of process synchronization is protecting shared resources from all the other processes that are trying to use that resource at the same time.\\

Java for example has both thread-safe and normal threading classes where thread-safe classes have larger overhear to lock objects and their methods when threads are using those resources. Thread-safe classes run much slower due to this overhead.\\


\subsection{Shared Memory}
Most OSs have ''shared memory'' which allows multiple processes to share the same memory.\\

\subsection{Sockets}
Sockets are a way of opening a two directional communication path with another process. Can be secure but isn't guaranteed. Sockets can be used with internet protocols such as Novell or TCP/IP.\\

When opening connection to devices on a network, a socket is opened for data to transmit through. The socket is then written to and read from.\\

\subsection{Signals}
\begin{defn}[Signals]Signals are like software interrupts. For example, Ctrl+C sends a signal to the current process that tells the process to process that signal. If the signal is not set in the system then the system uses the default which in this case is the 'Kill' command.
\end{defn}

\noindent
Signals can be programmed into software to be whatever the developer wants. The following command (Linux) is the only command that cannot be blocked by software.
\begin{verbatim}
kill -9 [process#]
\end{verbatim}

\noindent
Signaling is a way of talking between processes on the same computer as a software interrupt. It is also a way of telling your process what to do.

\subsection{Concurrent Processing Summary}
Those are the ways of allowing processes to communicate with each other.

\section{Semaphores}
A semaphore is used to control access to shared resources between processes.\\

Semaphores were originally used to control trains on railroad tracks to prevent collisions.\\

When a process is given the semaphore, it is able to use a requested shared resource, otherwise it is not able to use it.\\

Semaphores are handled by a special bit in the CPU. If the bit is a 0 it means the object is available otherwise it isn't. Usually in low level programming, test if the bit is 0 then set the bit. This operation must be atomic (done in 1 operation) otherwise there could be collisions.\\

The kernel has a table for process IDs and semaphores and all the threads in the kernel can only get access to this table. The semaphore bit controls who gets access to this table. This table is protected by the semaphore bit in the CPU. This is what allows multiple processes to access this single semaphore bit at the same time. You can create system functions to ask for semaphores. When done with the semaphore, you give it back.\\

Processes mount a semaphore, then either set/get/unset, then when done the semaphore is unmounted (disconnected from process).\\

\begin{ex}
Below are the steps taken by a process when using semaphores:
\begin{enumerate}
\item Mount semaphore 'sm'
\item Set 'sm'
\item Process
\item Unset 'sm'
\item Unmount semaphore 'sm'
\end{enumerate}
The setting and unsetting steps are usually in a loop where the process sets the semaphore then unsets the semaphore.
\end{ex}

\subsection{Deadlocks}
\begin{defn}[Deadlock]
A deadlock is when two or more processes are waiting for another process to unlock a semaphore in order to continue while the process that is being waited on is waiting for the other process to unlock a semaphore. This results in both processes being in a stalemate where each cannot continue and unlock its semaphore until the other process unlocks its semaphore.
\end{defn}
\begin{ex}
Let's say two processes each need 2 semaphores to process something. So process 1 gets semaphore B and A and process 2 gets semaphore A, C, then B.\\
First off, P1 locks B, P2 locks A. Then P1 is waiting for P2 to unlock A in order to continue while P2 is waiting for P1 to unlock B to continue. This is a deadlock.
\end{ex}

A way to avoid deadlocks is to give back the locked semaphore then wait for a random time and try to get the next semaphore again.\\

A way to tell if a process is in a deadlock is to set a timer on the semaphore requests (after initial request) then if request fails then try again.\\

In POSIX/UNIX you can only ask for one semaphore at a time. In System5 and Non-POSIX you can ask the kernel for multiple semaphores all in one function call. The kernel then handles the deadlock detection and returns either all the requested semaphores or none of them.

\mychapter{5}{2018-10-23}
\section{Semaphores Continued}
\textit{printf} uses a FILE* and a buffer of about 2k and printf writes to the buffer. The buffer doesn't get flushed to the OS. At a lower level the system uses the write() system call and sends the buffer to this system call. In the OS there is at least one buffer, one of which is associated with the HDD called the ''Disk Controller''. What is written to the file is written to one of these buffers and if the system crashes then that written info will be lost. Only after the info if successfully written is the transaction actually ended. If the system crashes then the database will be restores to the last backup and it will use the log file to bring it back to where the system was before the crash. Without the log file everything done since the last backup is gone.\\

Due to these problems, many commercial databases bypass the OS and take over the HDD to avoid all these buffers that leave room for error.\\ 

\begin{defn}[Dirty Read]
The option to allow reading locked records in a database. This can result in reading information that is in the middle of a transaction and what is read will actually be old data that was changed. However dirty reads are much faster, though this is the trade-off that must be considered. This is an option that can be configured for a database in most database systems.
\end{defn}

Thus, buffers are a problem if you need guaranteed data transfer. This is why semaphores are needed.

\section{Fork/Execute/Redirection/Pipes}
stdio contains the following:
\begin{itemize}
\item FILE *stdin
	Normal programs read from this. File number 0.
\item FILE *stdout
	Normal programs write to this. File number 1.
\item FILE *stderr
	stderr has no buffer. File number 2.
\end{itemize}

\subsection{Redirection}
Redirection is when stdin or stdout is changed.\\

The following line redirects stdout to point to the value at the right. So instead of writing to the normal stdout, the output of the program is a.out. The program still sees stdout as normal but the pointer for stdout is changed.
\begin{verbatim}
a > a.out
\end{verbatim}

The following redirects stdin to point to the value at the right. So instead of reading from normal stdin, the input for the program is coming from a2.out. As before, the program still sees stdin as normal but the pointer for stdin is changed.
\begin{verbatim}
a < a.out
\end{verbatim}

You can also redirect stderr.\\
Redirect stdout to one file and stderr to another:
\begin{verbatim}
a > a.out 2>error
\end{verbatim}
Redirect stderr to stdout (\&1) then redirect stdout to file:
\begin{verbatim}
a > a.out 2>&1
\end{verbatim}
Redirect both to a file:
\begin{verbatim}
a &> a.out
\end{verbatim}

\subsection{Pipes}
Pipes are another form of redirection. Pipes take the stdout from one program and sends it as stdin to another. Below, the output from program a is being sent as input to program b. This is not limited to a single pipe, so you can pipe over and over again. Unknown at time of class if you can use the same program in a series of pipes multiple times.
\begin{verbatim}
a | b
\end{verbatim}

\subsection{Forks}
When a process forks itself, an exact duplicate of that process is created. So if a program 'a' forks itself, there will now be two processes: 'a' and its child.\\

fork() returns either 0 or the process id (pid). fork() returns a 0 to the child process created and a pid of the child process to the parent. Otherwise if fork returns a negative value then the creation of the child process failed.\\

The process flow of fork() is as follows
\begin{verbatim}
fork()
redirect -> open -> dupe
execute()
\end{verbatim}

During the second step, argc, argv, and an array of environment variables are created.\\

The return of execute is what comes back from main.\\

Summary:\\
Fork creates an exact copy of the parent, the variables are all in the same places. Fork returns the code for the parent or child. Anything done in child will not affect the child. Anything done in the parent before the fork will affect the child.\\

\pagebreak
Simple command line interpreter. Most code omitted to prevent confusion:
\begin{verbatim}
int main(int argc, char **argv){
  int line_count, pid;
  char buf[1000];
  int fileOne, fileTwo, fileThree;
  if(argv<4){
  	// not enough args
  }else{
    if((pid = fork()) < 0){
  		
    }else if(pid == 0{
      //child
      // Open files in here and save to file vars
      execlp(); // executes new program
    }else{
      //parent
      wait_pid(pid);
    }
  }
}
\end{verbatim}
There are different versions of the exec system call. The execlp() version uses 'l' to pass a list of command-line arguments and 'p' (stands for PATH) which allows passing in of environment variables. This tells the program to search through all the PATH directories for the file specified; otherwise you need to use a specific path. This is a Unix/POSIX function. Any lists passed in in any version must be null terminated. The other versions of the exec function can be found by looking up the Exec system call.\\

When execute is called in a process, the process that called that execute command disappears.

\mychapter{6}{2018-10-25}
\section{Assignment 3 Notes}
For Assignment 3, fork() returns 2 values: either process id or 0. If 0 then you're in child.\\

This assignment can be completed with a computer with a single core/thread.\\

Don't forget to unlock the semaphores.\\

This assignment returns the number of deadlocks during the program. If you don't get any, then right after running the first semaphore sleep the thread for a second. This will force a deadlock.\\

sem\textunderscore wait stalls until receiving a semaphore, sem\textunderscore timedwait waits until it either gets the semaphore or the timer runs out in which case it will return an error. sem\textunderscore post gives the semphore back. sem\textunderscore open opens the sempahore, sem\textunderscore unlink destroys the named semaphore. We'll be using a version of sem\textunderscore wait called sem\textunderscore timedwait in this assignment. We'll also be using both named and unnamed semaphores. Read the manpage for sem() to see more functions.\\

Each process in this assignment will count the number of deadlocks that it gets.\\

wait() and waitpid() can be used to wait for the child process to die before continuing with parent process.\\

Look up pthread library and look up what is needed for named ones.\\

\section{Semaphores Continued}

Semaphores are all system calls provided by the kernel. In user processes in higher level languages, there are concepts like mutexes and shared code segments and thread safe classes (like in java). All of these are in user space (higher level function) used underneath system calls used by semaphores. These higher level languages only use 1 semaphore. You can also tell a semaphore to allow multiple people to use it at once but no more than that specified number. However Hardin hasn't ever seen anyone use this feature.\\

System 5 semaphores allow the user to ask for multiple semaphores at once from the same system call. The kernel detects the deadlocks. It will either give you all or none of the semaphores. System 5 was depreciated for ease of use over functionality. POSIX doesn't use System 5 semaphores.\\

\section{Links and inodes}
You can view inode information for files by running the command below

\begin{verbatim}
root@comp:~/$ ls -i
\end{verbatim}

In any given system there is a series of unique inode numbers. The inode number is how the OS refers to any given file. So copying a file in multiple locations will have the same inode number for each of these files. By running the following command, you can create a link to a file and link the two.

\begin{verbatim}
root@comp:~/$ ln err.out errln
\end{verbatim}

Note that both the newly created 'errln' and the old 'err.out' have the same inode number. These two files can be in any directories in the OS and they will still share the same inode number. This is an example of a hard link. However going across different file systems you will need a soft link in order to link files together. Then by running the following command, the third column shows the number of links to the file.

\begin{verbatim}
root@comp:~/$ ls -il
\end{verbatim}

So in our example, 'err.out' and 'errln' both have a 2 in the third column of the command results.\\

Directories will have 2 links by default because they have the link to the directory in its containing directory and the '.' link which links to the name in the containing directory inside of the directory itself. Thus the inode number in the containing folder and the inode number of the '.' inside that folder will be the same number.\\

So unlinking will remove the inode link between the two. The 'rm' command unlinks during the deletion process. There is also an unlink command which is possibly a depreciated version of 'rm' which deletes the link between two items.\\



\mychapter{7}{2018-10-30}
\section{Message Queues}
Message queues are controlled by the kernel. Message queues provide inter-process communication on any given computer (as are Pipes). These are not needed if using threads. If using threads, you need some way of controlling the messages sent to the queue controller. Threads can however use message queues and could benefit from using them.\\

Look up how to open/read/write using message queues.\\

The advantage of message queues is you don't need separate queues for both processes on both ends.\\

\section{Sockets}
All operating systems support sockets.\\

Secure Socket Handler (SSH)\\

Secure Sockets require double key encryption systems on both ends whereas Unsecure Sockets don't require this.\\

\mychapter{8}{2018-11-01}
\section{Events/Signals}
Event handlers are just a more sophisticated form of signals. Signals are like interrupts but interrupts are mediated and handled by the kernel whereas signals are mediated by the kernel and handled by the process.\\

When the CPU received an interrupt, it stops whatever it is doing and jumps to the proper location in the interrupt table where it finds a jump to the code it needs to run to handle that interrupt.\\

\section{printf/scanf}


\mychapter{9}{2018-11-08}
\section{Virtual-address Space}
Between the stack and heap in virtual memory is a hole. A hole means that the location on the physical storage at the end of the stack is somewhere other than just before the start of the heap's address space on physical storage.\\

These holes in memory are left there to allow for growth of the stack/heap. Holes also occur in files.\\

When you put a hole in a file, the filesize includes the size of the hole. However when you get the size of the file on the disk, the size of the hole isn't included.

\section{Paging}
Most OS's bring page in when needed and have a minimum size of the page.\\

\subsection{Basic Page Replacement}
\begin{enumerate}
\item Find location of desired page on disk.
\item Find a free frame:
	\begin{itemize}
	\item if there is free frame, use it 
	\item otherwise use page replacement algorithm to select a victim frame then write victim frame to disk if dirty
	\end{itemize}
\item Bring desired page into newly free frame; update page and frame tables.
\item Continue the process by restarting the instruction that caused the trap.
\end{enumerate}

One way of determining next page to load, you can use a stack to record the most recent page references. Otherwise there are ways to keep track of what should come next but those come at high overhead cost.\\

\subsection{Thrashing}
If a process doesn't have enough pages, the page-fault rate is very high. This can lead to thrashing.
\begin{defn}[Thrashing]
A process is busy swapping pages in and out.
\end{defn}

One thing to watch out for with 2D arrays is traversing column-by-column vs row-by-row. In C, rows of 2D arrays are all contiguous memory so it is faster to traverse row-by-row since it will have a lot less traversal time hopping between memory addresses. In an example where we have a 2D array of size $128\times 128$ where each row is stored in one page, traversing by columns causes over 16,000 page faults whereas traversing by rows causes only 128 page faults.

\section{Array Indexing}
Assuming we have
\begin{verbatim}
char a[20];
\end{verbatim}
and this array starts at 1024 in memory. If we index to a[i] then to index to that place in memory the computer does $1024+(i*sizeof(char))$. So the computer calculates the address offset from the start of the array.\\

Next if we have a 2D array such as
\begin{verbatim}
char a[3,4];
\end{verbatim}
then we get 3 chunks of memory that are 4 bytes. So our array in memory is 3 pointers that point to their chunk of 4 bytes in memory. The address of a is the start of these 3 pointers in memory; in this example the address is 1024. To index to a[x,y], the computer does $1024+(1024+sizeof(address)) + (y*sizeof(char))$.\\

This is how C works however some languages such as fortran do not function this way but since fortran is dead we won't explain that.\\




\mychapter{10}{2018-11-13}
\section{Exam 3 Review}
\begin{enumerate}
\item[2.] What is the main difference between multiple threads and multiple processes as far as memory from malloc?\\
Threads all share the heap whereas processes don't.

\item[3.]Why does the following correct code work as far as the value stored in n? What part of the run time system allows recursion?
\begin{verbatim}
int f(int n){ if(n < 2) return 1; return n * f(n-1); }
\end{verbatim}

The stack allows for recursion.

\item[8.]Rewrite the loop as a while loop with no if, break, or continue.\\
Do not add the \textit{n++} line.

\item[9.]What is the difference between named and unnamed semaphores as far as where the semaphore is stored?\\
Named semaphores are stored in the file system by the kernel. Unnamed semaphores are stored in memory. Both however are a variable with a pointer to them.

\item[10.]The exec() function has a version that allows passing environmental variables. What are these and how are they used?\\
Environmental variables are global variables. These can be set in the cshrc files.

\item[11.]The exec() function has a version that uses an environmental variable to search for the program you want to execute. What is the name of this variable and how is it used?\\
The path variable is the list of directories that a program searches within when executed.

\item[13.]What does the following command line to do stdin and stdout of P1 and P2?
\begin{verbatim}
$ P1 | P2
\end{verbatim}

P1 stdin stays the same. P1 stdout goes to input of the pipe. P2 stdin goes to the output of the pipe. P2 stdout is the screen.

\item[15.]If your process is waiting for a system call like sleep() or read() and it receives a signal you catch, what does the system call do after your code returns?\\

The system call returns with the program with some sort of error code explaining why it was interrupted.

\item[17.]Who/what enforces the use of Semaphores; for example, to share the console screen?\\

You as a programmer have to decide to use semaphores.

\item[18.]There is only one bit in the CPU that can be used as a semaphore. The kernel maintains many semaphores for the processes that are running. How does this work?\\

Only one kernel thread at any time can check or change a semaphore in the semaphore table.

\item[21.]C, Java, and .NET all use different methods to manage dynamically allocated memory.
	\begin{enumerate}
	\item How does the garbage collection differ for Java and .NET?\\
	Java does a copy into new memory during defragmentation whereas .NET defragments memory in place.
	
	\item When a program asks for new memory, how does each language find it?\\
	C follows a linked list of free spaces in the malloc buffer. Java just gives memory at the end of the allocated memory and uses garbage collection to squish memory together.

	\item What performance penalty does Java's allocation cause?\\
	Variables in java have a handle that must be looked up in a table to find a memory address. This lookup and overhear causes a slowdown.
	\end{enumerate}
\end{enumerate}



\end{document}
