\documentclass[12pt]{article}
\usepackage[margin=1in]{geometry} 
\usepackage{amsmath,amsthm,amssymb,amsfonts}
\usepackage{tabto}
\usepackage[yyyymmdd]{datetime}
\renewcommand{\dateseparator}{--}
\newcommand{\N}{\mathbb{N}}
\newcommand{\Z}{\mathbb{Z}}

% For definitions
\newtheorem{defn}{Definition}[section]
\newtheorem{thrm}{Theorem}[section]

% For circled text
\usepackage{tikz}
\newcommand*\circled[1]{\tikz[baseline=(char.base)]{
            \node[shape=circle,draw,inner sep=0.8pt] (char) {#1};}}

% For equation system alignment
\usepackage{systeme,mathtools}
% Usage:
%	\[
%	\sysdelim.\}\systeme{
%	3z +y = 10,
%	x + y +  z = 6,
%	3y - z = 13}

\newenvironment{problem}[2][Problem]{\begin{trivlist}
\item[\hskip \labelsep {\bfseries #1}\hskip \labelsep {\bfseries #2.}]}{\end{trivlist}}
%If you want to title your bold things something different just make another thing exactly like this but replace "problem" with the name of the thing you want, like theorem or lemma or whatever
 
%used for matrix vertical line
\makeatletter
\renewcommand*\env@matrix[1][*\c@MaxMatrixCols c]{%
  \hskip -\arraycolsep
  \let\@ifnextchar\new@ifnextchar
  \array{#1}}
\makeatother 
 
 

\newtheorem*{lemma}{Lemma} %added
\newtheorem*{result}{Result} %added

\begin{document}
\TabPositions{4cm}
%\renewcommand{\qedsymbol}{\filledbox}
%Good resources for looking up how to do stuff:
%Binary operators: http://www.access2science.com/latex/Binary.html
%General help: http://en.wikibooks.org/wiki/LaTeX/Mathematics
%Or just google stuff
 
\today {} Linear Algebra Notes\\

For 1.76 in homework, you can write each of the equations in the 2x3 matrix as an augmented matrix 6x2 and solve as normal for the 12 unknowns.\\
2.80(b) on homework:\\
z and t become free variables and x and y are dependent variables.\\
When writing out solution set, you want to assign z and t (or any free variables) to a new set of variables.\\
Ex:\\
Parameterization of solution set:
\begin{align*}
t &= q\\
z &= p\\
y &= 2p-2q+1\\
x &= -p+2q
\end{align*}

\section*{Invertibility}
\begin{defn} A square matrix A is invertible if there is another square matrix B such that AB is the identity matrix.
\end{defn}

Ex:
\[
A = 
\begin{bmatrix}[rr]
5 & 17\\
2 & 7\\
\end{bmatrix}
\]
Inverse of A is
\[
\begin{bmatrix}[rr]
7 & -17\\
-2 & 5\\
\end{bmatrix}
\]

\[
\begin{bmatrix}[rr]
5 & 17\\ 2 & 7\\
\end{bmatrix}
\begin{bmatrix}[rr]
7 & -17\\ -2 & 5\\
\end{bmatrix}
=
\begin{bmatrix}[rr] 1 & 0\\ 0 & 1\\ \end{bmatrix}
\]

Notation:
\[ 
A^{-1} = \begin{bmatrix}[rr]
5 & 17\\
2 & 7\\
\end{bmatrix} ^{-1}
\]

Alternate method of writing matrices in 1.64\\
Find $\begin{bmatrix}[r] x\\ y\\ z\\ \end{bmatrix}$ such that
\[
\begin{bmatrix}[rrr] 1 & 2 & 4\\ 3 & 5 & -2\\ 3 & -1 & 3\\ \end{bmatrix}
\begin{bmatrix}[r] x\\ y\\ z\\ \end{bmatrix}
=
\begin{bmatrix} 9\\ -3\\ 16\\ \end{bmatrix}
\]
Solution
\[
\begin{bmatrix}[r] x\\ y\\ z\\ \end{bmatrix}
=
\begin{bmatrix}[rrr] 1 & 2 & 4\\ 3 & 5 & -2\\ 3 & -1 & 3\\ \end{bmatrix}^{-1}
\begin{bmatrix} 9\\ -3\\ 16\\ \end{bmatrix}
=
\begin{bmatrix}[r] 3\\ -1\\ 2\\ \end{bmatrix}
\]

\begin{thrm} Suppose A is an invertible n x m matrix. Then
\[ rref\begin{bmatrix}[r|r] A & I_n\\ \end{bmatrix}
=
\begin{bmatrix}[r|r] I_n & A_{-1}\\ \end{bmatrix}
\]
\end{thrm}

Fact: Every elementary row operation can be written as an invertible matrix.\\
\[
\begin{bmatrix}[rrr]
1 & 0 & 0\\ 0 & 4 & 0\\ 0 & 0 & 1\\
\end{bmatrix} ^{-1}
=
\begin{bmatrix}[rrr]
1 & 0 & 0\\ 0 & \frac{1}{4} & 0\\ 0 & 0 & 1\\
\end{bmatrix}
\]


\end{document}