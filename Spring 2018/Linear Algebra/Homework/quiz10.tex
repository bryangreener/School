\documentclass{article}
\usepackage{amsmath,amsgen,amstext,amsbsy,amsopn,amsfonts}
\pagestyle{empty}


\setlength{\textwidth}{6in}
\setlength{\textheight}{10truein}
\setlength{\evensidemargin}{.5in}
\setlength{\oddsidemargin}{.25in}
\setlength{\topmargin}{-1 in}
\setlength{\parskip} {0.125 in}
\setlength{\parindent} {0 in}

\newcommand{\R}{\mathbb{R}}


\begin{document}
\large


{\bf Name(s) \hrulefill}

{\bf
\begin{centering}
Elementary Linear Algebra (MATH 2300) \\
Quiz \#10  \\
\end{centering}
}


\vskip .125 in

{\bf Instructions.} Show or explain all of your work.  You are allowed to work on this
quiz with a colleague and/or your notes.

1. [10] Suppose a matrix is given as
$$M=\left[\begin{array}{rrr}
1 & -1 & 2 \\
-1 & 1 & 2 \\
2 & 2 & -2 \\
\end{array}\right].$$

(a) Find an orthogonal set of eigenvectors for $M$.

(b) For each eigenvalue $\lambda$, determine the matrix of the orthogonal projection
$\mathbb{R}^3\xrightarrow{\ E_\lambda\ }V_\lambda$, where $V_\lambda$ denotes the eigenspace
corresponding to $\lambda$.  (A formula for orthogonal projection appears on the last page of Exam 3.)

(c) Check your answers.  You should confirm that: \\
The column space of $E_\lambda$ coincides with $V_\lambda$ for each eigenvalue, \\
the multiplicity of each eigenvalue coincides with the dimension of $V_\lambda$, \\
$E_\lambda^2=E_\lambda$ for each eigenvalue, \\
if $\lambda\neq\mu$, then $E_\lambda\circ E_\mu=0$, \\
$\sum_\lambda E_\lambda$ is an identity matrix, and \\
$\sum_\lambda \lambda E_\lambda=M$. \\

\end{document}



