\documentclass[12pt]{article}
\usepackage[margin=1in]{geometry} 
\usepackage{amsmath,amsthm,amssymb,amsfonts}
\usepackage{tabto}
\usepackage{hyperref}

\usepackage{arydshln} % gives hdashline and cdashline
\newcommand*{\tempb}{\multicolumn{1}{:c}{}} % Used for block matrices

% Spacers:
% BEGIN BLOCK------------------------------------------
% END BLOCK============================================




\newcommand{\N}{\mathbb{N}}
\newcommand{\Z}{\mathbb{Z}}

% CUSTOM SETTINGS
% BEGIN BLOCK------------------------------------------
% For equation system alignment
\usepackage{systeme,mathtools}
% Usage:
%	\[
%	\sysdelim.\}\systeme{
%	3z +y = 10,
%	x + y +  z = 6,
%	3y - z = 13}


% For definitions
\newtheorem{defn}{Definition}[section]
\newtheorem{thrm}{Theorem}[section]

% For circled text
\usepackage{tikz}
\newcommand*\circled[1]{\tikz[baseline=(char.base)]{
            \node[shape=circle,draw,inner sep=0.8pt] (char) {#1};}}

\newenvironment{problem}[2][Problem]{\begin{trivlist}
\item[\hskip \labelsep {\bfseries #1}\hskip \labelsep {\bfseries #2.}]}{\end{trivlist}}
%If you want to title your bold things something different just make another thing exactly like this but replace "problem" with the name of the thing you want, like theorem or lemma or whatever
 
%used for matrix vertical line
\makeatletter
\renewcommand*\env@matrix[1][*\c@MaxMatrixCols c]{%
  \hskip -\arraycolsep
  \let\@ifnextchar\new@ifnextchar
  \array{#1}}
\makeatother 

% END BLOCK============================================

\newtheorem*{lemma}{Lemma} %added
\newtheorem*{result}{Result} %added
\newtheorem*{theorem}{Theorem} %added
\theoremstyle{definition}
\newtheorem*{solution}{Solution} %added
\theoremstyle{plain}

% HEADER
% BEGIN BLOCK------------------------------------------
\usepackage{fancyhdr}
 
\pagestyle{fancy}
\fancyhf{}
\lhead{Homework \#14}
\rhead{Bryan Greener}
\cfoot{\thepage}
% END BLOCK============================================

% TITLE
% BEGIN BLOCK------------------------------------------
\title{Bryan Greener}
\author{MATH 2300 CRN:15163}
\date{2018-04-10}
\begin{document}
\maketitle
% END BLOCK============================================

\TabPositions{4cm}

\begin{enumerate}
\item[11.71]Find the characteristic and minimum polynomials of each matrix:
	\begin{enumerate}
	\item $A=\begin{bmatrix}[rrr]3&1&-1\\2&4&-2\\-1&-1&3\\\end{bmatrix}$\\
	\[ A-\lambda I = \begin{bmatrix}[rrr]3-\lambda & 1 & -1\\2&4-\lambda &-2\\-1&-1&3-\lambda\\\end{bmatrix} \]
	Next we compute the determinant
	\[ \mathrm{det}(A) = -\lambda^3+10\lambda^2-28\lambda+24 = (\lambda-6)(\lambda-2)^2 \]
	Thus $m(t)$ is either $f(t)=(t-6)(t-2)$ or $g(t)=(t-6)(t-2)^2$.\\
	By the Cayley-Hamilton theorem, $g(A)=\Delta(A)=0$. Hence we only need to test $f(t)$.
	\[ f(A)=(A-6I)(A-2I) = \begin{bmatrix}[rrr]-3&1&-1\\2&-2&-2\\-1&-1&-3\\\end{bmatrix}\begin{bmatrix}[rrr]1&1&-1\\2&2&-2\\-1&-1&1\\\end{bmatrix} = \begin{bmatrix}[rrr]0&0&0\\0&0&0\\0&0&0\\\end{bmatrix} \]
	Thus $m(t)=f(t)=(t-6)(t-2)=t^2-12t+12$ is the minimal polynomial of $A$.
	
	\item $B=\begin{bmatrix}[rrr]3&2&-1\\3&8&-3\\3&6&-1\\\end{bmatrix}$\\
	Repeating the process from part (a), we get $\mathrm{det}(B)=-\lambda^3+10\lambda^2-28\lambda+24$. Since this is the same characteristic polynomial as part (a), then we get the same set of possibilities for $m(t)$. Thus we only need to test
	\[ \begin{bmatrix}[rrr]3&2&-1\\3&8&-3\\3&6&-1\\\end{bmatrix}\begin{bmatrix}[rrr]1&2&-1\\3&6&-3\\3&6&-3\\\end{bmatrix}\neq\begin{bmatrix}[rrr]0&0&0\\0&0&0\\0&0&0\\\end{bmatrix} \]
	Since $m(t)\neq f(t)$, then $m(t)=g(t)=(t-6)(t-2)^2$.
	\end{enumerate}
\item[11.72]Find the characteristic and minimum polynomials of each matrix:
	\begin{enumerate}
	\item $A=\begin{bmatrix}[rrrrr]2&5&0&0&0\\0&2&0&0&0\\0&0&4&2&0\\0&0&3&5&0\\0&0&0&0&7\\\end{bmatrix}$\\
	This matrix is block diagonal with blocks
	\[ A=\begin{bmatrix}[rr]2&5\\0&2\\\end{bmatrix}, \quad B=\begin{bmatrix}[rrr]4&2\\3&5\\\end{bmatrix}, \quad C=\begin{bmatrix}[r]7\\\end{bmatrix} \]
	The matrix $A$ has minimum polynomial of $f(t)=(t-2)^2$ and $B$ has $g(t)=(t-2)(t-7)$ and $C$ has $h(t)=(t-7)$. Thus $\Delta(t) = f(t)g(t)h(t) = (t-2)^3(t-7)^2$. So $m(t)=LCM[f(t)g(t)h(t)] = (t-2)^2(t-7)$.
	\item $B=\begin{bmatrix}[rrrrr]4&-1&0&0&0\\1&2&0&0&0\\0&0&3&1&0\\0&0&0&3&1\\0&0&0&0&3\\\end{bmatrix}$\\
	This matrix has the block matrices
	\[ A=\begin{bmatrix}[rr]4&-1\\1&2\\\end{bmatrix}, \quad B=\begin{bmatrix}[rrr]3&1&0\\0&3&1\\0&0&3\\\end{bmatrix} \]
	$A$ has minimum polynomial $f(t)=(t-3)^2$ and $B$ has $g(t)=(t-3)^3$. Thus $\Delta(t)=(t-3)^5$ but $m(t)=(t-3)^3$.
	\item $C=\begin{bmatrix}[rrrrr]3&2&0&0&0\\1&4&0&0&0\\0&0&3&1&0\\0&0&1&3&0\\0&0&0&0&4\\\end{bmatrix}$\\
	This matrix has the block matrices
	\[ A=\begin{bmatrix}[rr]3&2\\1&4\\\end{bmatrix}, \quad B=\begin{bmatrix}[rrr]3&1&0\\1&3&0\\0&0&4\\\end{bmatrix}, \quad C=\begin{bmatrix}[r]4\\\end{bmatrix} \]
	Matrix $A$ has the minimum polynomial $f(t)=(t-2)(t-5)$ and $B$ has $g(t)=(t-4)(t-2)$ and $C$ has $h(t)=(t-4)$. Thus $\Delta(t)=(t-2)^2(t-5)(t-4)^2$ but $m(t)=(t-2)(t-4)(t-5)$.
	\end{enumerate}
\item[11.75]Show that a matrix $A$ and its transpose $A^T$ have the same minimum polynomial.

\item[11.77]Show that $A$ is a scalar matrix $kI$ if and only if the minimum polynomial of $A$ is $m(t)=t-k$.
\item[11.78]Find a matrix $A$ whose minimum polynomial is: (a) $t^3-5t^2+6t+8$; (b) $t^4-5t^3-2t+7t+4$.
	\begin{enumerate}
	\item $A=\begin{bmatrix}[rrr]0&0&-8\\1&0&-6\\0&1&5\\\end{bmatrix}$
	\item $A=\begin{bmatrix}[rrrr]0&0&0&-4\\1&0&0&-7\\0&1&0&2\\0&0&1&5\\\end{bmatrix}$
	\end{enumerate}
\item[11.79]Consider the following matrices in Jordan canonical form:
\[ A=\begin{bmatrix}[ccccccccccc]
	4&1&0&\tempb &{}&{}&{}&{}&{}&{}&{}\\
	0&4&1&\tempb &{}&{}&{}&{}&{}&{}&{}\\
	0&0&4&\tempb &{}&{}&{}&{}&{}&{}&{}\\
	\cdashline{1-6}
	{}&{}&{}&\tempb &4&1&\tempb &{}&{}&{}&{}\\
	{}&{}&{}&\tempb &0&4&\tempb &{}&{}&{}&{}\\
	\cdashline{4-9}
	{}&{}&{}&{}&{}&{}&\tempb &2&1&\tempb &{}\\
	{}&{}&{}&{}&{}&{}&\tempb &0&2&\tempb &{}\\
	\cdashline{7-11}
	{}&{}&{}&{}&{}&{}&{}&{}&{}&\tempb &2\\\end{bmatrix} 
	\quad B=\begin{bmatrix}[ccccccccccc]
	4&1&\tempb &{}&{}&{}&{}&{}&{}&{}&{}\\
	0&4&\tempb &{}&{}&{}&{}&{}&{}&{}&{}\\
	\cdashline{1-5}
	{}&{}&\tempb &4&1&\tempb &{}&{}&{}&{}&{}\\
	{}&{}&\tempb &0&4&\tempb &{}&{}&{}&{}&{}\\
	\cdashline{3-7}
	{}&{}&{}&{}&{}&\tempb &4 &\tempb &{}&{}&{}\\
	\cdashline{6-11}	
	{}&{}&{}&{}&{}&{}&{}&\tempb &2&1&0\\	
	{}&{}&{}&{}&{}&{}&{}&\tempb &0&2&1\\	
	{}&{}&{}&{}&{}&{}&{}&\tempb &0&0&2\\\end{bmatrix}	
	\]
	\begin{enumerate}
	\item Find the characteristic polynomials of $A$ and $B$.\\
	Since both matrices have five 4s and three 2s on the diagonal, then $\Delta(t)=(t-4)^5(t-2)^3$ is the characteristic polynomial of both $A$ and $B$ and their eigenvalues are 2 and 4.
	\item Find the minimum polynomials of $A$ and $B$.\\
	The minimum polynomial for $A$ is $(t-4)^3(t-2)^2$ since the largest block with 4s in the diagonal is of order 3 and the largest block with 2s is of order 2. The minimum polynomial for $B$ is $(t-4)^2(t-2)^3$ for the opposite reason of A's minimum polynomial.
	\item Find a maximal set $S$ of linearly independent eigenvectors of $A$.\\
	Since each block contributes one eigenvector to $S$, then a maximal set can be $v_1=(1,0,0,0,0,0,0,0),v_2=(0,0,0,1,0,0,0,0),v_3=(0,0,0,0,0,1,0,0),v_4=(0,0,0,0,0,0,0,1)$.
	\item Find a maximal set $S$ of linearly independent eigenvectors of $B$.\\
	Since each block contributes one eigenvector to $S$, then a maximal set can be $v_1=(1,0,0,0,0,0,0,0),v_2=(0,0,1,0,0,0,0,0),v_3=(0,0,0,0,1,0,0,0),v_4=(0,0,0,0,0,1,0,0)$.
	\end{enumerate}
	
\item[11.80]Find all possible Jordan canonical forms whose characteristic polynomials $\Delta (t)$ and minimum polynomials $m(t)$ are as follows: (a) $\Delta (t)=(t-2)^4(t-3)^2,m(t)=(t-2)^2(t-3)^2$; (b) $\Delta (t)=(t-7)^5,m(t)=(t-7)^2$; (c) $\Delta (t) =(t-8)^4(t-1)^3,m(t)=(t-8)^3(t-1)^2$.
	\begin{enumerate}
	\item
	\item
	\item
	\end{enumerate}		
\item[11.81]The six possible Jordan canonical forms with characteristic polynomial $\Delta (t)=(t-2)^3(t-5)^2$ follow:\\
I'm not writing these out. Block graphs are a nightmare in \LaTeX
\item[12.28]For each matrix $A$, find a nonsingular matrix $P$ and a diagonal matrix $D$ such that $D=P^TAP$:
	\begin{enumerate}
	\item $A=\begin{bmatrix}[rrr]1&0&2\\0&3&6\\2&6&7\\\end{bmatrix}$
	\item $A=\begin{bmatrix}[rrr]1&-2&1\\-2&5&3\\1&3&-2\\\end{bmatrix}$
	\end{enumerate}		
\item[12.29]For each matrix $B$, find a nonsingular matrix $P$ and a diagonal matrix $D$ such that $D=P^TAP$:
	\begin{enumerate}
	\item $B=\begin{bmatrix}[rrrr]1&1&-2&-3\\1&2&-5&-1\\-2&-5&10&9\\-3&-1&9&11\\\end{bmatrix}$
	\end{enumerate}		
\item[12.30]Find the symmetric matrix belonging to the following quadratic forms: (a) $q(x,y)=3x^2+8xy-7y^2$; (b) $q(x,y)=2x^2-xy+5y^2$.
	\begin{enumerate}
	\item
	\item
	\end{enumerate}		
\item[12.31]Find the symmetric matrix, belonging to the following quadratic forms:\\
(a) $q(x,y,z)=2x^2-8xy+y^2-16xz+14yz+5z^2$; (b) $q(x,y,z)=x^2-xz+y^2$; (c) $q(x,y,z)=xy+y^2+4xz+z^2$; (d) $q(x,y,z)=xy+yz$.
	\begin{enumerate}
	\item
	\item
	\item
	\item
	\end{enumerate}		
\item[12.32]Let $q(x,y)=2x^2-6xy-3y^2$ and $x=s+2t$, $y=3s-t$.
	\begin{enumerate}
	\item Rewrite $q(x,y)$ in matrix notation, and find the matrix $A$ representing the quadratic form.
	\item Rewrite the linear substitution using matrix notation, and find the matrix $P$ corresponding to the substitution.
	\item Find $q(s,t)$ using: (i) direct substitution; (ii) matrix notation.
	\end{enumerate}		
	
\end{enumerate}	
\end{document}