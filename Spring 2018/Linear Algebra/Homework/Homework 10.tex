\documentclass[12pt]{article}
\usepackage[margin=1in]{geometry} 
\usepackage{amsmath,amsthm,amssymb,amsfonts}
\usepackage{tabto}
\usepackage{hyperref}

% Spacers:
% BEGIN BLOCK------------------------------------------
% END BLOCK============================================




\newcommand{\N}{\mathbb{N}}
\newcommand{\Z}{\mathbb{Z}}

% CUSTOM SETTINGS
% BEGIN BLOCK------------------------------------------
% For equation system alignment
\usepackage{systeme,mathtools}
% Usage:
%	\[
%	\sysdelim.\}\systeme{
%	3z +y = 10,
%	x + y +  z = 6,
%	3y - z = 13}


% For definitions
\newtheorem{defn}{Definition}[section]
\newtheorem{thrm}{Theorem}[section]

% For circled text
\usepackage{tikz}
\newcommand*\circled[1]{\tikz[baseline=(char.base)]{
            \node[shape=circle,draw,inner sep=0.8pt] (char) {#1};}}

\newenvironment{problem}[2][Problem]{\begin{trivlist}
\item[\hskip \labelsep {\bfseries #1}\hskip \labelsep {\bfseries #2.}]}{\end{trivlist}}
%If you want to title your bold things something different just make another thing exactly like this but replace "problem" with the name of the thing you want, like theorem or lemma or whatever
 
%used for matrix vertical line
\makeatletter
\renewcommand*\env@matrix[1][*\c@MaxMatrixCols c]{%
  \hskip -\arraycolsep
  \let\@ifnextchar\new@ifnextchar
  \array{#1}}
\makeatother 

% END BLOCK============================================

\newtheorem*{lemma}{Lemma} %added
\newtheorem*{result}{Result} %added
\newtheorem*{theorem}{Theorem} %added
\theoremstyle{definition}
\newtheorem*{solution}{Solution} %added
\theoremstyle{plain}

% HEADER
% BEGIN BLOCK------------------------------------------
\usepackage{fancyhdr}
 
\pagestyle{fancy}
\fancyhf{}
\lhead{Homework \#10}
\rhead{Bryan Greener}
\cfoot{\thepage}
% END BLOCK============================================

% TITLE
% BEGIN BLOCK------------------------------------------
\title{Bryan Greener}
\author{MATH 2300 CRN:15163}
\date{2018-03-16}
\begin{document}
\maketitle
% END BLOCK============================================

\TabPositions{4cm}

\begin{enumerate}
\item[9.29]For each of the following linear transformations $L$ on $\mathbb{R}^2$, find the matrix $A$ representing $L$ (relative to the usual basis of $\mathbb{R}^2$):
	\begin{enumerate}
	\item $L$ is the rotation in $\mathbb{R}^2$ counterclockwise by $45^\circ$.
		\begin{align*}
		\begin{bmatrix}[rr]\cos(45^\circ)&-\sin(45^\circ)\\\sin(45^\circ)&\cos(45^\circ)\\\end{bmatrix}\begin{bmatrix}[rr]1&0\\0&1\\\end{bmatrix} &= \begin{bmatrix}[rr]\frac{\sqrt{2}}{2}&-\frac{\sqrt{2}}{2}\\\frac{\sqrt{2}}{2}&\frac{\sqrt{2}}{2}\\\end{bmatrix}
		\end{align*}
	\item $L$ is the reflection in $\mathbb{R}^2$ about the line $y=x$.
		\[ L\begin{bmatrix}[r]1\\0\\\end{bmatrix} = \begin{bmatrix}[r]0\\1\\\end{bmatrix} \quad L\begin{bmatrix}[r]0\\1\\\end{bmatrix} = \begin{bmatrix}[r]1\\0\\\end{bmatrix} \]
		\[ L\begin{bmatrix}[rr]1&0\\0&1\\\end{bmatrix} = \begin{bmatrix}[rr]0&1\\1&0\\\end{bmatrix} \]
	\item $L$ is defined by $L(1,0)=(3,5)$ and $L(0,1)=(7,-2)$.\\
		We write each transformation as a column vector to get our end result.
		\[ \begin{bmatrix}[rr]3&7\\5&-2\\\end{bmatrix} \]
	\item $L$ is defined by $L(1,1)=(3,7)$ and $L(1,2)=(5,-4)$.\\
		
	\end{enumerate}
\item[9.30]Find the matrix representing each of the following linear transformations $T$ on $\mathbb{R}^3$ relative to the usual basis of $\mathbb{R}^3$:
	\begin{enumerate}
	\item $T(x,y,z)=(x,y,0)$
		\[ \begin{bmatrix}[rrr]1&0&0\\0&1&0\\0&0&0\\\end{bmatrix} \]
	\item $T(x,y,z)=(2x-7y-4z,3x+y+4z,6x-8y+z)$
		\[ \begin{bmatrix}[rrr]2&-7&-4\\3&1&4\\6&-8&1\\\end{bmatrix} \]
	\end{enumerate}
\item[9.33]Let $D$ denote the differential operator, that is $F(f(t))=\frac{df}{dt}$. Each of the following sets is a basis of a vector space $V$ of functions. Find the matrix representing $D$ in each basis:
	\begin{enumerate}
	\item $\{a^t,e^{2t},te^{2t}\}$
		\begin{align*}
		D(a^t)&=t\\
		D(e^{2t})&=2e^{2t}\\
		D(te^{2t})&=e^{2t}+2te^{2t}\\
		[D] &= \begin{bmatrix}[rrr]1&0&0\\0&2&1\\0&0&2\\\end{bmatrix}
		\end{align*}
	\item $\{1,t,\sin(3t),\cos(3t)\}$
		\begin{align*}
		D(1)&=0\\
		D(t)&=1\\
		D(\sin(3t))&=3\cos(3t)\\
		D(\cos(3t))&=3-\sin(3t)\\
		[D] &= \begin{bmatrix}[rrrr]0&1&0&0\\0&0&0&0\\0&0&0&-3\\0&0&3&0\\\end{bmatrix}
		\end{align*}
	\end{enumerate}
\item[9.34]Let $V$ be the vector space of real $2\times2$ matrices and let $M=\begin{bmatrix}[rr]a&b\\c&d\\\end{bmatrix}$. Find the matrix representing each of the following linear operators $T$ on $V$ relative to the usual basis of $V$:
\[ \left\{ E_1=\begin{bmatrix}[rr]1&0\\0&0\\\end{bmatrix}, E_2=\begin{bmatrix}[rr]0&1\\0&0\\\end{bmatrix}, E_3=\begin{bmatrix}[rr]0&0\\1&0\\\end{bmatrix},E_4=\begin{bmatrix}[rr]0&0\\0&1\\\end{bmatrix} \right\} \]
	\begin{enumerate}
	\item $T(A)=MA$
		\begin{align*}
		T(E_1)&=ME_1 = \begin{bmatrix}[rr]a&b\\c&d\\\end{bmatrix}\begin{bmatrix}[rr]1&0\\0&0\\\end{bmatrix} = \begin{bmatrix}[rr]a&0\\c&0\\\end{bmatrix} = aE_1+0E_2+cE_3+0E_4\\
		T(E_2)&=ME_2 = 0E_1+aE_2+0E_3+cE_4\\
		T(E_3)&=ME_3 = bE_1+0E_2+dE_3+0E_4\\
		T(E_4)&=ME_4 = 0E_1+bE_2+0E_3+dE_4\\
		[T]&= \begin{bmatrix}[rrrr]a&0&b&0\\0&a&0&b\\c&0&d&0\\0&c&0&d\\\end{bmatrix}
		\end{align*}
	\item $T(A)=AM$
		\begin{align*}
		E_1M&= aE_1+bE_2+0E_3+0E_4\\
		E_2M&= cE_1+dE_2+0E_3+0E_4\\
		E_3M&= 0E_1+0E_2+aE_3+bE_4\\
		E_4M&= 0E_1+0E_2+cE_3+dE_4\\
		[T]&=\begin{bmatrix}[rrrr]a&c&0&0\\b&d&0&0\\0&0&a&c\\0&0&b&d\\\end{bmatrix}
		\end{align*}
	\item $T(A)=MA-AM$
		\[ MA-AM=\begin{bmatrix}[rrrr]a&0&b&0\\0&a&0&b\\c&0&d&0\\0&c&0&d\\\end{bmatrix}-\begin{bmatrix}[rrrr]a&c&0&0\\b&d&0&0\\0&0&a&c\\0&0&b&d\\\end{bmatrix} = \begin{bmatrix}[cccc]0&-c&b&0\\-b&a-d&0&b\\c&0&d-a&-c\\0&c&-b&0\\\end{bmatrix} \]
	\end{enumerate}
\item[9.38]Find the trace of each linear map on $\mathbb{R}^3$:
	\begin{enumerate}
	\item $F(x,y,z)=(x+3y,3x-2z,x-4y-3z)$\\
		Setting each equation as a row in a $3\times 3$ matrix then adding the values in the diagonal from top left to bottom right gives us the trace of $-2$.
	\end{enumerate}
\item[9.43]Let $G:\mathbb{R}^3\rightarrow\mathbb{R}^2$ be defined by $G(x,y,z)=(2x+3y-z,4x-y+2x)$.
	\begin{enumerate}
	\item Find the matrix $A$ representing $G$ relative to the bases
	\[ S=\{(1,1,0),(1,2,3),(1,3,5)\} \quad \mathrm{and} \quad S^\prime = \{(1,2),(2,3)\} \]
	\item For any $v=(a,b,c)$ in $\mathbb{R}^3$, find $[v]_S$ and $[G(v)]_S$.
	\item Verify that $A[v]_S=[G(v)]_{S^\prime}$
	\end{enumerate}
\item[7.60]Let $V$ be the vector space of polynomials over $\mathbb{R}$ of degree $\leq 2$ with inner product defined by $<f,g>=\int_0^1f(t)g(t)dt$. Find a basis of the subspace $W$ orthogonal to $h(t)=2t+1$.
\item[7.61]Find a basis of the subspace $W$ of $\mathbb{R}^4$ orthogonal to $u_1=(1,-2,3,4)$ and $u_2=(3,-5,7,8)$.
\item[7.63]Let $w=(1,-2,-1,3)$ be a vector in $\mathbb{R}^4$. Find:
	\begin{enumerate}
	\item an orthogonal basis for $w^\perp$.
	\item an orthonormal basis for $w^\perp$.
	\end{enumerate}
\item[7.65]Let $S$ consist of the following vectors in $\mathbb{R}^4$:
\[ u_1=(1,1,1,1), \quad u_2=(1,1,-1,-1), \quad u_3=(1,-1,1,-1), \quad u_4=(1,-1,-1,1) \]
	\begin{enumerate}
	\item Show that $S$ is orthogonal and a basis of $\mathbb{R}^4$.
	\item Write $v=(1,3,-5,6)$ as a linear combination of $u_1,u_2,u_3,u_4$.
	\item Find the coordinates of an arbitrary vector $v=(a,b,c,d)$ in $\mathbb{R}^4$ relative to the basis $S$.
	\item Normalize $S$ to obtain an orthonormal basis of $\mathbb{R}^4$.
	\end{enumerate}


two of 9.41,9.47,9.48,7.57






\end{enumerate}

\end{document}