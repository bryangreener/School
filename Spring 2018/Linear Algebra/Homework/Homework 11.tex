\documentclass[12pt]{article}
\usepackage[margin=1in]{geometry} 
\usepackage{amsmath,amsthm,amssymb,amsfonts}
\usepackage{tabto}
\usepackage{hyperref}

% Spacers:
% BEGIN BLOCK------------------------------------------
% END BLOCK============================================




\newcommand{\N}{\mathbb{N}}
\newcommand{\Z}{\mathbb{Z}}

% CUSTOM SETTINGS
% BEGIN BLOCK------------------------------------------
% For equation system alignment
\usepackage{systeme,mathtools}
% Usage:
%	\[
%	\sysdelim.\}\systeme{
%	3z +y = 10,
%	x + y +  z = 6,
%	3y - z = 13}


% For definitions
\newtheorem{defn}{Definition}[section]
\newtheorem{thrm}{Theorem}[section]

% For circled text
\usepackage{tikz}
\newcommand*\circled[1]{\tikz[baseline=(char.base)]{
            \node[shape=circle,draw,inner sep=0.8pt] (char) {#1};}}

\newenvironment{problem}[2][Problem]{\begin{trivlist}
\item[\hskip \labelsep {\bfseries #1}\hskip \labelsep {\bfseries #2.}]}{\end{trivlist}}
%If you want to title your bold things something different just make another thing exactly like this but replace "problem" with the name of the thing you want, like theorem or lemma or whatever
 
%used for matrix vertical line
\makeatletter
\renewcommand*\env@matrix[1][*\c@MaxMatrixCols c]{%
  \hskip -\arraycolsep
  \let\@ifnextchar\new@ifnextchar
  \array{#1}}
\makeatother 

% END BLOCK============================================

\newtheorem*{lemma}{Lemma} %added
\newtheorem*{result}{Result} %added
\newtheorem*{theorem}{Theorem} %added
\theoremstyle{definition}
\newtheorem*{solution}{Solution} %added
\theoremstyle{plain}

% HEADER
% BEGIN BLOCK------------------------------------------
\usepackage{fancyhdr}
 
\pagestyle{fancy}
\fancyhf{}
\lhead{Homework \#11}
\rhead{Bryan Greener}
\cfoot{\thepage}
% END BLOCK============================================

% TITLE
% BEGIN BLOCK------------------------------------------
\title{Bryan Greener}
\author{MATH 2300 CRN:15163}
\date{2018-03-19}
\begin{document}
\maketitle
% END BLOCK============================================

\TabPositions{4cm}

\begin{enumerate}
\item[2.73]Find the line $L$ of best fit for each set of points:
	\begin{enumerate}
	\item $(1,4),(2,6),(3,10),(4,15),(5,20)$
		\begin{align*}
		(1,4): 4&=a+1b\\
		(2,6): 6&=a+2b\\
		(3,10): 10&=a+3b\\
		(4,15): 15&=a+4b\\
		(5,20): 20&=a+5b
		\end{align*}
		\[ \begin{bmatrix}[rr]4&1\\6&2\\10&3\\15&4\\20&5\\\end{bmatrix}\begin{bmatrix}[r]a\\b\\\end{bmatrix}=\begin{bmatrix}[r]4\\6\\10\\15\\20\\\end{bmatrix} \]
		\begin{align*}
		M^TM&=\begin{bmatrix}[rrrrr]1&1&1&1&1\\1&2&3&4&5\\\end{bmatrix}\begin{bmatrix}[rr]1&1\\1&2\\1&3\\1&4\\1&5\\\end{bmatrix} = \begin{bmatrix}[rr]5&15\\15&55\\\end{bmatrix}\\
		M^T\vec{w}&=\begin{bmatrix}[rrrrr]1&1&1&1&1\\1&2&3&4&5\\\end{bmatrix}\begin{bmatrix}[r]4\\6\\10\\15\\20\\\end{bmatrix} = \begin{bmatrix}[r]55\\206\\\end{bmatrix}\\
		\vec{v}&\approx \begin{bmatrix}[rr]5&15\\15&55\\\end{bmatrix}^{-1}\begin{bmatrix}[r]55\\206\\\end{bmatrix} \approx \begin{bmatrix}[r]-1.3\\4.1\\\end{bmatrix}
		\end{align*}
		Thus our y-intercept is $\approx -1.3$ and our slope is $\approx 4.1$. Therefore the line of best fit is $y \approx 4.1x-1.3$.
	\end{enumerate}
\pagebreak
\item[7.69]Let $U$ and $W$ be subspaces of a finite-dimensional inner product space $V$. Show that:
	\begin{enumerate}
	\item $(U+W)^\perp = U^\perp \cap W^\perp$
	\begin{proof}
	Suppose that $\vec{v}\in(U+W)^\perp$, then
	\[ \vec{v}\cdot (\vec{u}+\vec{w})=0\]
	for all $\vec{w}\in W$. Then $\vec{v}\cdot (\vec{0}+\vec{w})=0$ for all $\vec{w}\in W$. So $\vec{v}\cdot \vec{w} = 0$. Thus $\vec{v}\in W^\perp$. The same argument also shows that $\vec{v}\in U^\perp$. Since $\vec{v}$ is in both $W^\perp$ and $U^\perp$, then $\vec{v}\in (U \cap W)$. Thus $(U+W)^\perp \subset U^\perp \cap W^\perp$.\\
	Next let $\vec{v}\in (U^\perp \cap W^\perp )$. Then $\vec{v}\cdot (\vec{a}) = 0$ for all $\vec{a}\in (U\cap W)$, $\vec{v}\in U^\perp$, and $\vec{v}\in W^\perp$. Notice that $(\vec{0}+W)^\perp = U^\perp$ and $(U+\vec{0})^\perp = W^\perp$. So $\vec{v}\in U^\perp$ and $\vec{v}\in W^\perp$ and so $\vec{v}\in (U+W)^\perp$. Thus $U^\perp \cap W^\perp \subset (U+W)^\perp$.\\
	Since $(U+W)^\perp \subset U^\perp \cap W^\perp$ and $U^\perp \cap W^\perp \subset (U+W)^\perp$, then $(U+W)^\perp = U^\perp \cap W^\perp$.
	\end{proof}
	\end{enumerate}

\item[7.70]Find the Fourier coefficient $c$ and the project $cw$ of $v$ along $w$, where:
	\begin{enumerate}
	\item[(b)] $v=(1,3,1,2)$ and $w=(1,-2,7,4)$ in $\mathbb{R}^4$.
		\[ \langle v,w \rangle=1-6+7+8=10 \quad \mathrm{and} \quad \langle w,w \rangle=1+4+49+16=70 \]
		So $c=\frac{10}{70}=\frac{1}{7}$ and $\mathrm{proj}(v,w)=cw=(\frac{1}{7},-\frac{2}{7},1,\frac{4}{7})$.
	\end{enumerate}

\item[7.71]Let $U$ be the subspace of $\mathbb{R}^4$ spanned by:
\[ v_1=(1,1,1,1), \quad v_2=(1,-1,2,2), \quad v_3=(1,2,-3,-4) \]
	\begin{enumerate}
	\item Apply the Gram-Schmidt algorithm to find an orthogonal and an orthonormal basis for $U$.
		\begin{align*}
		w_2&=v_2-\frac{\langle v_2,w_1\rangle}{||w_1||^2}w_1 = (1,-1,2,2)-\frac{4}{4}(1,1,1,1) = (0,-2,1,1)\\
		w_3&=v_3-\frac{\langle v_3,w_1 \rangle}{||w_1||^2}w_1-\frac{ \langle v_3,w_2 \rangle}{||w_2||^2}w_2\\
		&= (1,2,-3,-4)-\frac{-4}{4}(1,1,1,1)-\frac{-11}{6}(0,-2,1,1) = (12,-4,-1,-7)
		\end{align*}
		Next we normalize the orthogonal basis: $||w_1||^2 = 4$, $||w_2||^2=6$, and $||w_3||^2=\frac{121}{6}$. Thus the following form an orthogonal basis of $U$:
		\[ u_1=\frac{1}{2}(1,1,1,1), \quad u_2=\frac{1}{\sqrt{6}}(0,-2,1,1), \quad u_3 = \frac{\sqrt{6}}{11}(12,-4,-1,-7) \]
	\item Find the projection of $v=(1,2,-3,4)$ onto $U$.\\
	Since $U$ is orthogonal, we need to compute the Fourier coefficients.
	\begin{align*}
	c_1 &= \frac{\langle v,u_1 \rangle}{||u_1||^2} = 1\\
	c_2 &= \frac{\langle v,u_2 \rangle}{||u_2||^2} = -\frac{1}{2}\\
	c_3 &= \frac{\langle v,u_3 \rangle}{||u_3||^2} = -\frac{1}{10}\\
	u&=\mathrm{proj}(v,U)=c_1u_1+c_2u_2+c_3u_3\\
	&= 1(1,1,1,1)-\frac{1}{2}(0,-2,1,1)-\frac{1}{10}(12,-4,-1,-7)\\
	&= \frac{1}{5}(-1,12,3,6)
	\end{align*}
	\end{enumerate}

\item[7.73]Consider the subspace $W=P_2(t)$ of $P(t)$ with inner product $<f,g>=\int_0^1f(t)g(t)dt$. Find the projection of $f(t)=t^3$ onto $W$. (Hint: Use the orthogonal polynomials $1,2t-1,6t^2-6t+1$ obtained in Problem 7.21.)\\
	Since we have an orthogonal basis, we need to compute the Fourier coefficients. We will let $v$ be $f(t)=t^3$ and $u_1, u_2, u_3$ be the orthogonal basis given.
	\begin{align*}
	\frac{\langle t^3,1 \rangle}{\langle 1,1 \rangle} &+ \frac{\langle t^3,2t-1 \rangle}{\langle 2t-1,2t-1 \rangle}(2t-1)\\
	&+ \frac{\langle t^3,6t^2-6t+1 \rangle}{\langle 6t^2-6t+1,6t^2-6t+1 \rangle}(6t^2-6t+1)\\
	&= \frac{3t^2}{2}-\frac{3t}{5} + \frac{1}{20}
	\end{align*}

\item[7.76]Find a $3\times 3$ orthogonal matrix $P$ whose first two rows are multiples of $u=(1,1,1)$ and $v=(1,-2,3)$, respectively.\\
Correction: Let $u_1=(1,2,1)$, $u_2=(1,-2,3)$. Original problem vectors were not orthogonal.
	\[ \begin{bmatrix}[rrr]\frac{1}{\sqrt{3}}&\frac{1}{\sqrt{3}}&\frac{1}{\sqrt{3}}\\\frac{1}{\sqrt{14}}&-\frac{2}{\sqrt{14}}&\frac{3}{\sqrt{14}}\\\frac{5}{\sqrt{38}}&-\frac{2}{\sqrt{38}}&-\frac{3}{\sqrt{38}}\\\end{bmatrix} \]
\pagebreak
\item[7.81]Find the matrix $C$ which represents the usual inner product on $\mathbb{R}^3$ relative to the basis $S$ of $\mathbb{R}^3$ consisting of the vectors $u_1=(1,1,1),u_2=(1,2,1),$ and $u_3=(1,-1,3)$.\\
	\begin{align*}
	\langle u_1, u_1 \rangle &= 3\\
	\langle u_1, u_2 \rangle &= 4\\
	\langle u_1, u_3 \rangle &= 3\\
	\langle u_2, u_3 \rangle &= 2\\
	\langle u_2, u_2 \rangle &= 6\\
	\langle u_3, u_3 \rangle &= 11\\
	&\Rightarrow \begin{bmatrix}[rrr]3&4&3\\4&6&2\\3&2&11\\\end{bmatrix}
	\end{align*}

\item[7.85]Suppose $B$ is a real nonsingular matrix. Show that: (a) $B^TB$ is symmetric; (b) $B^TB$ is positive definite.
	\begin{enumerate}
	\item
	\begin{proof}
		Notice that $u^TB^TBu = (u^TB^T)(Bu) = (Bu)^T(Bu)$. Thus $B^TB$ fits the definition of a symmetric matrix since its transpose is multiplied by itself.
	\end{proof}
	\item
	\begin{proof}
		Notice that $u^TB^TBu = (u^TB^T)(Bu) = (Bu)^T(Bu) = (Bu)\cdot (Bu)$. 	
	\end{proof}
	\end{enumerate}

\item[10.51]Evaluate (a) $\begin{vmatrix}2&6\\4&1\\\end{vmatrix}$; (b) $\begin{vmatrix}5&1\\3&-2\\\end{vmatrix}$; (c) $\begin{vmatrix}-2&8\\-5&-3\\\end{vmatrix}$; (d) $\begin{vmatrix}4&9\\1&-3\\\end{vmatrix}$; (e) $\begin{vmatrix}a+b&a\\b&a+b\\\end{vmatrix}$
	\begin{enumerate}
	\item $2(1)-6(4) = -22$
	\item $5(-2)-1(3) = -13$
	\item $-2(-3)-8(-5) = 46$
	\item[(e)] $(a+b)(a+b)-a(b) = a^2+b^2+ab$
	\end{enumerate}

\item[10.52]Find all $t$ such that: (a) $\begin{vmatrix}t-4&3\\2&t-9\\\end{vmatrix} = 0$; (b) $\begin{vmatrix}t-1&4\\3&t-2\\\end{vmatrix} = 0$.
	\begin{enumerate}
	\item $(t-4)(t-9)-3(2) = t^2-13t+30 = 0$ Thus $t=3$ or $t=10$.
	\item $(t-1)(t-2)-4(3) = t^2-3t-10 = 0$ Thus $t=-2$ or $t=5$.
	\end{enumerate}

\item[10.53]Compute the determinant of each matrix:
	\begin{enumerate}
	\item $\begin{bmatrix}[rrr]2&1&1\\0&5&-2\\1&-3&4\\\end{bmatrix}$
	\[ (2*5*4)+(1*-2*1)+(1*0*-3)-(2*-2*-3)-(1*0*4)-(1*5*1) = 21 \]
	\item $\begin{bmatrix}[rrr]3&-2&-4\\2&5&-1\\0&6&1\\\end{bmatrix}$
	\[ (3*5*1)+(-2*-1*0)+(-4*2*6)-(3*-1*6)-(-2*2*1)-(-4*5*0*)=-11 \]
	\item $\begin{bmatrix}[rrr]-2&-1&4\\6&-3&-2\\4&1&2\\\end{bmatrix}$
	\[ (-2*-3*2)+(-1*-2*4)+(4*6*1)-(-2*-2*1)-(-1*6*2)-(4*-3*4) = 100 \]
	\end{enumerate}

\item[10.54]Evaluate the determinant of each matrix: (a) $\begin{bmatrix}[rrrr]1&2&2&3\\1&0&-2&0\\3&-1&1&-2\\4&-3&0&2\\\end{bmatrix}$; (b) $\begin{bmatrix}[rrrr]2&1&3&2\\3&0&1&-2\\1&-1&4&3\\2&2&-1&1\\\end{bmatrix}$
	\begin{enumerate}
	\item $-131$
	\end{enumerate}

\item[7.74]Consider $P(t)$ with inner product $\langle f,g \rangle = \int_{-1}^1f(t)g(t)dt$.
	\begin{enumerate}
	\item Find an orthogonal basis for the subspace $W=P_3(t)$ by applying the Gram-Schmidt algorithm to $\{1,t,t^2,t^3\}$.\\
	We follow the gram-schmidt formula.
	\begin{align*}
		w_1&=v_1\\
		w_2&=v_2-\frac{\langle v_2,w_1\rangle}{\langle 1,1 \rangle}w_1 = t\\
		w_3&=v_3-\frac{\langle v_3,w_1 \rangle}{\langle 1,1 \rangle}w_1-\frac{\langle v_3,w_2\rangle}{\langle t,t \rangle}w_2 = 3t^2-1\\
		w_4&=v_4-\frac{\langle v_4,w_1\rangle}{\langle 1,1 \rangle}w_1-\frac{\langle v_4,w_2\rangle}{\langle t,t \rangle}w_2-\frac{\langle v_4,w_3\rangle}{\langle t^2,t^2 \rangle}w_3 = 5t^3-3t
	\end{align*}
	Thus we form the basis for v: $[1,t,3t^2-1,5t^3-3t]$
	\item Find the projection of $f(t)=t^5$ onto $W$.
	\begin{align*}
	\frac{\langle t^5,1 \rangle}{\langle 1,1 \rangle} &+ \frac{\langle t^5,t \rangle}{\langle t,t \rangle}t\\
	&+ \frac{\langle t^5,3t^2-1 \rangle}{\langle 3t^3-1,3t^2-1 \rangle}(3t^2-1)\\
	&+ \frac{\langle t^5,5t^3-3t \rangle}{\langle 5t^3-3t,5t^3-3t \rangle}(5t^3-3t)\\
	&= \frac{10t^3}{9}-\frac{5t}{21}\\
	\end{align*}
	\end{enumerate}
	
\item[7.84] Suppose $A$ and $B$ are positive definite matrices. Show that: (a) $A+B$ is positive definite; (b) $kA$ is positive definite for $k>0$.
	\begin{enumerate}
	\item
	\begin{proof}
	Since $A$ and $B$ are positive definite, then $u^T(A+B)u \geq 0$ for all columns $u$ and $u^T(A+B)u=0$ only when $u=\vec{0}$.\\
	So $u^T(A+B)u=a^TAu+u^TBu$. Since both $A$ and $B$ are positive definite, then neither $u^TAu$ nor $u^TBu$ can be less than zero. Thus, their addition cannot result in any negative values. Therefore $u^T(A+B)u$ observes all the conditions for a positive-definite matrix.
	\end{proof}
	\end{enumerate}
\end{enumerate}
\end{document}