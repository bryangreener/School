\documentclass[12pt]{article}
\usepackage[margin=1in]{geometry} 
\usepackage{amsmath,amsthm,amssymb,amsfonts}
\usepackage{tabto}

\newcommand{\N}{\mathbb{N}}
\newcommand{\Z}{\mathbb{Z}}

% For definitions
\newtheorem{defn}{Definition}[section]
\newtheorem{thrm}{Theorem}[section]

% For circled text
\usepackage{tikz}
\newcommand*\circled[1]{\tikz[baseline=(char.base)]{
            \node[shape=circle,draw,inner sep=0.8pt] (char) {#1};}}

\newenvironment{problem}[2][Problem]{\begin{trivlist}
\item[\hskip \labelsep {\bfseries #1}\hskip \labelsep {\bfseries #2.}]}{\end{trivlist}}
%If you want to title your bold things something different just make another thing exactly like this but replace "problem" with the name of the thing you want, like theorem or lemma or whatever
 
%used for matrix vertical line
\makeatletter
\renewcommand*\env@matrix[1][*\c@MaxMatrixCols c]{%
  \hskip -\arraycolsep
  \let\@ifnextchar\new@ifnextchar
  \array{#1}}
\makeatother 
 
 

\newtheorem*{lemma}{Lemma} %added
\newtheorem*{result}{Result} %added

\begin{document}
\TabPositions{4cm}
%\renewcommand{\qedsymbol}{\filledbox}
%Good resources for looking up how to do stuff:
%Binary operators: http://www.access2science.com/latex/Binary.html
%General help: http://en.wikibooks.org/wiki/LaTeX/Mathematics
%Or just google stuff
 
2018-01-11 Linear Algebra Notes\\

\section*{Gauss-Jordan Elimination}
\begin{align*}
IF \quad M=
\begin{bmatrix}[cccc|c]
1 & 2 & -1 & 2 & 1\\
2 & 4 & 1 & -2 & 3\\
3 & 6 & 2 & -6 & 5\\
\end{bmatrix}
\begin{bmatrix}
R_1\\ R_2\\ R_3\\
\end{bmatrix}\\
%========================================
\begin{bmatrix}
R_1\\
R_2 - 2R_1\\
R_3-3R_1\\
\end{bmatrix}
\begin{bmatrix}[cccc|c]
\circled{1} & 2 & -1 & 2 & 1\\
0 & 0 & \circled{3} & -6 & 1\\
0 & 0 & \circled{5} & -12 & 2\\
\end{bmatrix}
\begin{bmatrix}
R_4\\ R_5\\ R_6\\
\end{bmatrix}\\
%========================================
\begin{bmatrix}
R_4\\
(\frac{1}{3})R_5\\
R_6\\
\end{bmatrix}
\begin{bmatrix}[cccc|c]
1 & 2 & -1 & 2 & 1\\
0 & 0 & 1 & -2 & \frac{1}{3}\\
0 & 0 & 5 & -12 & 2\\
\end{bmatrix}
\begin{bmatrix}
R_7\\ R_8\\ R_9
\end{bmatrix}\\
%========================================
\begin{bmatrix}
R_7\\
R_8\\
R_9 - 5R_8\\
\end{bmatrix}
\begin{bmatrix}[cccc|c]
\circled{1} & 2 & -1 & 2 & 1\\
0 & 0 & \circled{1} & -2 & \frac{1}{3}\\
0 & 0 & 0 & \circled{-2} & \frac{1}{3}\\
\end{bmatrix}
\begin{bmatrix}
R_{10}\\ R_{11}\\ R_{12}\\
\end{bmatrix}\\
%========================================
\begin{bmatrix}
R_{10}\\
R_{11}\\
(-\frac{1}{2})R_{12}\\
\end{bmatrix}
\begin{bmatrix}[cccc|c]
1 & 2 & -1 & 2 & 1\\
0 & 0 & 1 & -2 & \frac{1}{3}\\
0 & 0 & 0 & 1 & -\frac{1}{6}\\
\end{bmatrix}
\begin{bmatrix}
S_1\\ S_2\\ S_3\\
\end{bmatrix}\\
%========================================
\begin{bmatrix}
S_1 - 2S_3\\
S_2 + 2S_3\\
S_3\\
\end{bmatrix}
\begin{bmatrix}[cccc|c]
\circled{1} & 2 & -1 & 0 & \frac{4}{3}\\
0 & 0 & \circled{1} & 0 & 0\\
0 & 0 & 0 & \circled{1} & -\frac{1}{6}\\
\end{bmatrix}
\begin{bmatrix}
S_4\\ S_5\\ S_6\\
\end{bmatrix}\\
%========================================
\begin{bmatrix}
S_4 + S_5\\
S_5\\
S_6\\
\end{bmatrix}
\begin{bmatrix}[cccc|c]
1 & 2 & 0 & 0 & \frac{4}{3}\\
0 & 0 & 1 & 0 & 0\\
0 & 0 & 0 & 1 & -\frac{1}{6}\\
\end{bmatrix}
\begin{bmatrix}
S_7\\
S_8\\
S_9\\
\end{bmatrix}
\end{align*}
This is now in reduced row-echelon form.\\
\\
\section*{Matrix/Vector Algebra}
Vector:
\begin{align*}
\begin{bmatrix}
x_1\\ x_2\\ x_3\\ \vdots \\ x_n
\end{bmatrix}\\
\end{align*}
Matrix Addition:
\begin{align*}
\begin{bmatrix}
2 & 0\\
-1 & 1\\
5 & 4\\
\end{bmatrix}
+
\begin{bmatrix}
1 & 4\\
3 & 1\\
7 & 7\\
\end{bmatrix}
=
\begin{bmatrix}
3 & 4\\
2 & 2\\
12 & 11\\
\end{bmatrix}
\end{align*}
Scalar Multiplication:
\begin{align*}
5 *
\begin{bmatrix}
1 & 4\\
3 & 1\\
7 & 7\\
\end{bmatrix}
=
\begin{bmatrix}
5 & 20\\
15 & 5\\
35 & 35\\
\end{bmatrix}
\end{align*}
Adding vectors can be represented by a graph with vectors on it.\\
Vector addition can be expressed using the tip-to-tail rule.\\
Scalar multiplication can also be expressed by a graph with vectors on it.\\

\begin{defn}
Suppose $V_a=(a_1,a_2,a_3,...,a_n)$ and $V_b = (b_1,b_2,b_3,...,b_n)$ are n-tuples. The dot product of $V_a$ and $V_b$ is $V_a \dot V_b$
\end{defn}
\end{document}