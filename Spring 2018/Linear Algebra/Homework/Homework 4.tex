\documentclass[12pt]{article}
\usepackage[margin=1in]{geometry} 
\usepackage{amsmath,amsthm,amssymb,amsfonts}
\usepackage{tabto}

% Spacers:
% BEGIN BLOCK------------------------------------------
% END BLOCK============================================




\newcommand{\N}{\mathbb{N}}
\newcommand{\Z}{\mathbb{Z}}

% CUSTOM SETTINGS
% BEGIN BLOCK------------------------------------------
% For equation system alignment
\usepackage{systeme,mathtools}
% Usage:
%	\[
%	\sysdelim.\}\systeme{
%	3z +y = 10,
%	x + y +  z = 6,
%	3y - z = 13}


% For definitions
\newtheorem{defn}{Definition}[section]
\newtheorem{thrm}{Theorem}[section]

% For circled text
\usepackage{tikz}
\newcommand*\circled[1]{\tikz[baseline=(char.base)]{
            \node[shape=circle,draw,inner sep=0.8pt] (char) {#1};}}

\newenvironment{problem}[2][Problem]{\begin{trivlist}
\item[\hskip \labelsep {\bfseries #1}\hskip \labelsep {\bfseries #2.}]}{\end{trivlist}}
%If you want to title your bold things something different just make another thing exactly like this but replace "problem" with the name of the thing you want, like theorem or lemma or whatever
 
%used for matrix vertical line
\makeatletter
\renewcommand*\env@matrix[1][*\c@MaxMatrixCols c]{%
  \hskip -\arraycolsep
  \let\@ifnextchar\new@ifnextchar
  \array{#1}}
\makeatother 

% END BLOCK============================================

\newtheorem*{lemma}{Lemma} %added
\newtheorem*{result}{Result} %added
\newtheorem*{theorem}{Theorem} %added


% HEADER
% BEGIN BLOCK------------------------------------------
\usepackage{fancyhdr}
 
\pagestyle{fancy}
\fancyhf{}
\lhead{Homework \#4}
\rhead{Bryan Greener}
\cfoot{\thepage}
% END BLOCK============================================

% TITLE
% BEGIN BLOCK------------------------------------------
\title{Bryan Greener}
\author{MATH 2300 CRN:15163}
\date{2018-01-21}
\begin{document}
\maketitle
% END BLOCK============================================

\TabPositions{4cm}


\begin{enumerate}
\item [3.69.] Let $A= \begin{bmatrix}[rr] 2 & -5\\ 3 & 1\\ \end{bmatrix}$. Find:
	\begin{enumerate}
	% 3.69.a
	\item $A^2$ and $A^3$\\
	\begin{align*}
	 	A^2 = AA = 
		\begin{bmatrix}[rr]
		(2*2)+(-5*3) & (2*-5)+(-5*1)\\
		(3*2)+(1*3) & (3*-5)+(1*1)\\
		\end{bmatrix}
		=
		&\begin{bmatrix}[rr] -11 & -15\\ 9 & -14\\ \end{bmatrix}\\
		A^3 = A^2A =
		\begin{bmatrix}[rr]
		(-11*2)+(-15*3) & (-11*-5)+(-15*1)\\
		(9*2)+(-14*3) & (9*-5)+(-14*1)\\
		\end{bmatrix}
		=
		&\begin{bmatrix}[rr] -67 & 40\\ -24 & -59\\ \end{bmatrix}
	\end{align*}
	
	% 3.69.b	
	\item $f(A)$ where $f(x)=x^3-2x^2-5$
	\begin{align*}
		f(A) =
		\begin{bmatrix}[rr] -67 & 40\\ -24 & -59\\ \end{bmatrix}
		-2 \begin{bmatrix}[rr] -11 & -15\\ 9 & -14\\ \end{bmatrix}
		-5 \begin{bmatrix}[rr] 1 & 0\\ 0 & 1\\ \end{bmatrix}
		= \begin{bmatrix}[rr] -50 & 70\\ -42 & -36\\ \end{bmatrix}
	\end{align*}
	
	% 3.69.c	
	\item $g(A)$ where $g(x)=x^2-3x+17$
	\begin{align*}
		g(A) =
		\begin{bmatrix}[rr] -11 & -15\\ 9 & -14\\ \end{bmatrix}
		- 3 \begin{bmatrix}[rr] 2 & -5\\ 3 & 1\\ \end{bmatrix}
		+ 17 \begin{bmatrix}[rr] 1 & 0\\ 0 & 1\\ \end{bmatrix}
		= \begin{bmatrix}[rr] 0 & 0\\ 0 & 0\\ \end{bmatrix}
	\end{align*}
	\end{enumerate}
	
\item [3.71.] Let $A= \begin{bmatrix}[rr] 6 & -4\\ 3 & -2\\ \end{bmatrix}$. Find a nonzero vector $u= \begin{bmatrix}[r] x\\ y\\ \end{bmatrix}$ such that $Au=4u$.
	\begin{align*}
		\begin{bmatrix}[rr] 6 & -4\\ 3 & -2\\ \end{bmatrix}
		\begin{bmatrix}[r] x\\ y\\ \end{bmatrix}
		&=
		4 \begin{bmatrix}[r] x\\ y\\ \end{bmatrix}\\
		\begin{bmatrix}[r] 6x-4y\\ 3x-2y\\ \end{bmatrix}
		&=
		\begin{bmatrix}[r] 4x\\ 4y\\ \end{bmatrix}
	\end{align*}
	\[
		\sysdelim{.}{.}\systeme[xy]{6x-4y=4x,3x-2y=4y} \rightarrow
		\sysdelim{.}{.}\systeme[xy]{2x-4y=0,3x-6y=0} \rightarrow 
		\sysdelim{.}{.}\systeme[xy]{2x-4y=0,0=0} \rightarrow
		2x-4y=0 \]
	Let $y=a$, then $2x-4y=2x-4a=0$ and so $x=2a$. Thus $u=(2a,a)^T$.

\item [3.72.] Let $A= \begin{bmatrix}[rr] 1 & 2\\ 0 & 1\\ \end{bmatrix}$. Find $A^n$.
	First, we need to convert this problem into a formula. Observe that $A^2=AA$, $A^3=AAA=A^A$, $A^4=AAAA=A^3A$. Thus for each value $n$, $A^n=A^{n-1}A$. We set out to prove this formula below.
	\begin{proof}
	We proceed by induction.\\
	Let $n=1$, then $A^1 = A^0A = 1A = A$. Assume for an arbitrary integer $k \geq 1$, that $A^k=A^{k-1}A$. We show that $A^{k+1} = A^{(k+1)-1}A = A^kA$.\\
	So,
		\begin{align*}
		A^{k+1} &= A^{(k+1)-1}A^1\\
		&= A^kA
		\end{align*}
	Thus by the principle of mathematical induction, $A^n = A^{n-1}A$.
	\end{proof}

\item [3.77.] Find the inverse of each matrix (if it exists):
	\[ 	A= \begin{bmatrix}[rr] 7 & 4\\ 5 & 3\\ \end{bmatrix}, \quad
		B= \begin{bmatrix}[rr] 2 & 3\\ 4 & 5\\ \end{bmatrix}, \quad
		C= \begin{bmatrix}[rr] 4 & -6\\ -2 & 3\\ \end{bmatrix}, \quad
		D= \begin{bmatrix}[rr] 5 & -2\\ 6 & -3\\ \end{bmatrix} \]
	\begin{align*}
		A^{-1}= \frac{1}{(7)(3)-(4)(5)} \begin{bmatrix}[rr] 3 & -4\\ -5 & 7\\ \end{bmatrix}
		= &\begin{bmatrix}[rr] 3 & -4\\ -5 & 7\\ \end{bmatrix}\\
		B^{-1}= \frac{1}{-2} \begin{bmatrix}[rr] 5 & -3\\ -4 & 2\\ \end{bmatrix}
		= &\begin{bmatrix}[rr] -\frac{5}{2} & \frac{3}{2}\\ 2 & -1\\ \end{bmatrix}\\
		C^{-1}=& undefined\\
		D^{-1}= \frac{1}{-3} \begin{bmatrix}[rr] -3 & 2\\ -6 & 5\\ \end{bmatrix}
		= &\begin{bmatrix}[rr] 1 & -\frac{2}{3}\\ 2 & -\frac{5}{3}\\ \end{bmatrix}
	\end{align*}


\item [3.78.] Find the inverse of each matrix (if it exists):
	\[	A= \begin{bmatrix}[rrr] 1 & 2 & -4\\ -1 & -1 & 5\\ 2 & 7 & -4\\ \end{bmatrix}, \quad
		B= \begin{bmatrix}[rrr] 1 & -1 & 1\\ 0 & 2 & -2\\ 1 & 3 & -1\\ \end{bmatrix}, \quad
		C= \begin{bmatrix}[rrr] 1 & 2 & 3\\ 2 & 5 & -1\\ 5 & 12 & 1\\ \end{bmatrix} \]

	\begin{align*}
	A= 
	\begin{bmatrix}[rrr|rrr]
	1 & 2 & -4 & 1 & 0 & 0\\
	-1 & -1 & 5 & 0 & 1 & 0\\
	2 & 7 & -4 & 0 & 0 & 1\\	
	\end{bmatrix}&
	\begin{bmatrix}[r] R_1\\ R_2\\ R_3\\ \end{bmatrix}\\
	%
	\begin{bmatrix}[r]
	R_1\\
	R_2 + R_1\\
	R_3 - 2R_1\\
	\end{bmatrix}
	\begin{bmatrix}[rrr|rrr]
	1 & 2 & -4 & 1 & 0 & 0\\
	0 & 1 & 1 & 1 & 1 & 0\\
	0 & 3 & 4 & -2 & 0 & 1\\
	\end{bmatrix}&
	\begin{bmatrix}[r] R_4\\ R_5\\ R_6\\ \end{bmatrix}\\
	%
	\begin{bmatrix}[r]
	R_4 - 2R_5\\
	R_5\\
	R_6 - 3R_5\\
	\end{bmatrix}
	\begin{bmatrix}[rrr|rrr]
	1 & 0 & -6 & -1 & -2 & 0\\
	0 & 1 & 1 & 1 & 1 & 0\\
	0 & 0 & 1 & -5 & -3 & 1\\
	\end{bmatrix}&
	\begin{bmatrix}[r] R_7\\ R_8\\ R_9\\ \end{bmatrix} \\
	%
	\begin{bmatrix}[r]
	R_7 + 6R_9\\
	R_8 - R_9\\
	R_9\\
	\end{bmatrix}
	\begin{bmatrix}[rrr|rrr]
	1 & 0 & 0 & -31 & -20 & 6\\
	0 & 1 & 0 & 6 & 4 & -1\\
	0 & 0 & 0 & -5 & -3 & 1\\
	\end{bmatrix}&\\
	%	
	%%
	%
	B=
	\begin{bmatrix}[rrr|rrr]
	1 & -1 & 1 & 1 & 0 & 0\\
	0 & 2 & -2 & 0 & 1 & 0\\
	1 & 3 & -1 & 0 & 0 & 1\\
	\end{bmatrix}&
	\begin{bmatrix}[r] R_1\\ R_2\\ R_3\\ \end{bmatrix}\\
	%
	\begin{bmatrix}[r]
	R_1\\
	R_2\\
	R_3 - R_1\\
	\end{bmatrix}
	\begin{bmatrix}[rrr|rrr]
	1 & -1 & 1 & 1 & 0 & 0\\
	0 & 2 & -2 & 0 & 1 & 0\\
	0 & 4 & -2 & -1 & 0 & 1\\
	\end{bmatrix}&
	\begin{bmatrix}[r] R_4\\ R_5\\ R_6\\ \end{bmatrix}\\
	%
	\begin{bmatrix}[r]
	R_4\\
	(\frac{1}{2})R_5\\
	R_6\\
	\end{bmatrix}
	\begin{bmatrix}[rrr|rrr]
	1 & -1 & 1 & 1 & 0 & 0\\
	0 & 1 & -1 & 0 & \frac{1}{2} & 0\\
	0 & 4 & -2 & -1 & 0 & 1\\
	\end{bmatrix}&
	\begin{bmatrix}[r] R_7\\ R_8\\ R_9\\ \end{bmatrix} \\
	%
	\begin{bmatrix}[r]
	R_7 + R_8\\
	R_8\\
	R_9 - 4R_8\\
	\end{bmatrix}
	\begin{bmatrix}[rrr|rrr]
	1 & 0 & 0 & 1 & \frac{1}{2} & 0\\
	0 & 1 & -1 & 0 & \frac{1}{2} & 0\\
	0 & 0 & 2 & -1 & -2 & 1\\
	\end{bmatrix}&
	\begin{bmatrix}[r] R_{10}\\ R_{11}\\ R_{12}\\ \end{bmatrix}\\
	%
	\begin{bmatrix}[r]
	R_{10}\\
	R_{11} + (\frac{1}{2})R_{12}\\
	(\frac{1}{2})R_12\\
	\end{bmatrix}
	\begin{bmatrix}[rrr|rrr]
	1 & 0 & 0 & 1 & \frac{1}{2} & 0\\
	0 & 1 & 0 & -\frac{1}{2} & -\frac{1}{2} & \frac{1}{2}\\
	0 & 0 & 1 & -\frac{1}{2} & -1 & \frac{1}{2}\\
	\end{bmatrix}&\\
	%
	%%
	%
	C=
	\begin{bmatrix}[rrr|rrr]
	1 & 2 & 3 & 1 & 0 & 0\\
	2 & 5 & -1 & 0 & 1 & 0\\
	5 & 15 & 1 & 0 & 0 & 1\\
	\end{bmatrix}&
	\begin{bmatrix}[r] R_1\\ R_2\\ R_3\\ \end{bmatrix}\\
	%
	\begin{bmatrix}[r]
	R_1\\
	R_2 - 2R_1\\
	R_3 - 5R_1\\
	\end{bmatrix}
	\begin{bmatrix}[rrr|rrr]
	1 & 2 & 3 & 1 & 0 & 0\\
	0 & 1 & -7 & -2 & 1 & 0\\
	0 & 5 & -14 & -5 & 0 & 1\\
	\end{bmatrix}&
	\begin{bmatrix}[r] R_4\\ R_5\\ R_6\\ \end{bmatrix}\\
	%
	\begin{bmatrix}[r]
	R_4 - 2R_5\\
	R_5\\
	R_6 - 5R_5\\
	\end{bmatrix}
	\begin{bmatrix}[rrr|rrr]
	1 & 0 & 17 & 5 & -2 & 0\\
	0 & 1 & -7 & -2 & 1 & 0\\
	0 & 0 & 21 & 5 & -5 & 1\\
	\end{bmatrix}&
	\begin{bmatrix}[r] R_7\\ R_8\\ R_9\\ \end{bmatrix}\\
	%
	\begin{bmatrix}[r]
	R_7 - 17((\frac{1}{21})R_9)\\
	R_8 + 7((\frac{1}{21})R_9)\\
	(\frac{1}{21})R_9\\
	\end{bmatrix}
	\begin{bmatrix}[rrr|rrr]
	1 & 0 & 0 & \frac{20}{21} & \frac{43}{21} & -\frac{17}{21}\\
	0 & 1 & 0 & -\frac{1}{3} & -\frac{2}{3} & \frac{1}{3}\\
	0 & 0 & 1 & \frac{5}{21} & -\frac{5}{21} & \frac{1}{21}\\
	\end{bmatrix}&	
	\end{align*}

\item [3.80.] Find the inverse of each matrix (where patterns of zeros have been omitted):
	\[ 	A= \begin{bmatrix}[rrrrr] 	1  & 2  & {} & {} & {}\\
									{} & 1  & 2  & {} & {}\\
									{} & {} & 1  & 2  & {}\\
									{} & {} & {} & 1  & 2 \\
									{} & {} & {} & {} & 1\\ \end{bmatrix}, \quad
		B= \begin{bmatrix}[rrrrr]	1  & 1  & 1  & 1  & 1\\
									{} & 1  & 1  & 1  & 1\\
									{} & {} & 1  & 1  & 1\\
									{} & {} & {} & 1  & 1\\
									{} & {} & {} & {} & 1\\ \end{bmatrix} \]
	For matrix $A$, we subtract each row by two times the row below it. For matrix $B$, we just subtract each row by the row below it. Performing these operations gives us the following inverse matrices:
	\[ 	A= \begin{bmatrix}[rrrrr] 	1  & -2  & {} & {} & {}\\
									{} & 1  & -2  & {} & {}\\
									{} & {} & 1  & -2  & {}\\
									{} & {} & {} & 1  & -2 \\
									{} & {} & {} & {} & 1\\ \end{bmatrix}, \quad
		B= \begin{bmatrix}[rrrrr]	1  & -1  & {}  & {}  & {}\\
									{} & 1  & -1  & {}  & {}\\
									{} & {} & 1  & -1  & {}\\
									{} & {} & {} & 1  & -1\\
									{} & {} & {} & {} & 1\\ \end{bmatrix} \]


\item [3.81.] Express each matrix as a product of elementary matrices:
	\begin{enumerate}
	% 3.81.a
	\item $A= \begin{bmatrix}[rr] 1 & 2\\ 3 & 4\\ \end{bmatrix}$\\
	To get this matrix to reduced row echelon form, we perform the following operations: $R_2 = R_2 - 3R_1$ then $R_2 = (-\frac{1}{2})R_2$ and finally $R_1 = R_1 - 2R_2$. Taking the inverse of each of these operation gives us the following inverse operations: $R_2 = R_2 + 3R_1$ then $R_2 = -2R_2$ and finally $R_1 = R_1+ 2R_2$. Performing these inverse operations on the reduced row echelon form of the original matrix produces the following result:
	\begin{align*}
	A= \begin{bmatrix}[rr] 1 & 0\\ 3 & 1\\ \end{bmatrix}
	\begin{bmatrix}[rr] 1 & 0\\ 0 & -2\\ \end{bmatrix}
	\begin{bmatrix}[rr] 1 & 2\\ 0 & 1\\ \end{bmatrix}
	= \begin{bmatrix}[rr] 1 & 2\\ 3 & 4\\ \end{bmatrix}
	\end{align*}
	\end{enumerate}

\item [3.82.] Express $A= \begin{bmatrix}[rrr] 1 & 2 & 0\\ 0 & 1 & 3\\ 3 & 8 & 7\\ \end{bmatrix}$ as a product of elementary matrices.\\
	To get this matrix to reduced row echelon form, we perform the following operations:
	\begin{center}
	\begin{tabular}{ c | c }
	Operation & Inverse Operation\\
	\hline
	$R_3-3R_1$ & $R_3 + 3R_1$\\
	$R_3-2R_2$ & $R_3 + 2R_2$\\
	$R_2-3R_3$ & $R_2 + 3R_3$\\
	$R_1-2R_2$ & $R_1 + 2R_2$\\
	\end{tabular}
	\end{center}
	
	Performing these inverse operations on the reduced row echelon form of the original matrix produces the following result:
	\begin{align*}
	A= 
	\begin{bmatrix}[rrr] 1 & 0 & 0\\ 0 & 1 & 0\\ 3 & 0 & 1\\ \end{bmatrix}
	\begin{bmatrix}[rrr] 1 & 0 & 0\\ 0 & 1 & 0\\ 0 & 2 & 1\\ \end{bmatrix}
	\begin{bmatrix}[rrr] 1 & 0 & 0\\ 0 & 1 & 3\\ 0 & 0 & 1\\ \end{bmatrix}
	\begin{bmatrix}[rrr] 1 & 2 & 0\\ 0 & 1 & 0\\ 0 & 0 & 0\\ \end{bmatrix}
	=
	\begin{bmatrix}[rrr] 1 & 2 & 0\\ 0 & 1 & 3\\ 3 & 8 & 7\\ \end{bmatrix}
	\end{align*}
	
\item [3.86.] Let $A=diag(1,2,3)$ and $B=diag(2,-5,0)$. Find:
	\begin{enumerate}
	% 3.86.b
	\item [(b)] $f(A)$ where $f(x)=x^2+4x-3$\\
	\begin{align*}
	f(1) = 1^2+4(1)-3 &= 2\\
	f(2) = 2^2+4(2)-3 &= 9\\
	f(3) = 3^2+4(3)-3 &= 18\\
	f(A) = diag(2,9,18)	
	\end{align*}
	
	\end{enumerate}


\item [3.89.] Find all real triangular matrices $A$ such that $A^2=B$, where:
	\begin{enumerate}
	% 3.89.a
	\item $B= \begin{bmatrix}[rr] 4 & 21\\ 0 & 25\\ \end{bmatrix}$\\
	Let $A= \begin{bmatrix}[rr] x & y\\ 0 & z\\ \end{bmatrix}$. Then 
	\begin{align*}
	A^2 =
	\begin{bmatrix}[rr] x & y\\ 0 & z\\ \end{bmatrix}
	\begin{bmatrix}[rr] x & y\\ 0 & z\\ \end{bmatrix}
	=
	\begin{bmatrix}[rr] 
	x^2 & y(x+z)\\
	0 & z^2\\
	\end{bmatrix}
	\end{align*}
	This results in the following $(x,y,z)$ combinations:
	\begin{align*}
	(2,3,5),(-2,-3,-5),(2,-7,-5),(-2,7,5)\\
	A^2 =
	\begin{bmatrix}[rr] 2 & 3\\ 0 & 5\\ \end{bmatrix},
	\begin{bmatrix}[rr] -2 & -3\\ 0 & -5\\ \end{bmatrix},
	\begin{bmatrix}[rr] 2 & -7\\ 0 & -5\\ \end{bmatrix},
	\begin{bmatrix}[rr] -2 & 7\\ 0 & 5\\ \end{bmatrix}
	\end{align*}
	\end{enumerate}


\item [3.90.] Let $B= \begin{bmatrix}[rrr] 1 & 8 & 5\\ 0 & 9 & 5\\ 0 & 0 & 4\\ \end{bmatrix}$. Find a triangular matrix $A$ with positive diagonal entries such that $A^2=B$.\\
	\begin{align*}
	A^2 =
	\begin{bmatrix}[rrr] a & b & c\\ 0 & d & e\\ 0 & 0 & f\\ \end{bmatrix}
	\begin{bmatrix}[rrr] a & b & c\\ 0 & d & e\\ 0 & 0 & f\\ \end{bmatrix}
	=
	\begin{bmatrix}[rrr] 
	a^2 & b(a+d) & c(a+f)+be\\ 
	0 & d^2 & e(d+f)\\ 
	0 & 0 & f^2\\ \end{bmatrix} 
	\end{align*}
	Using the matrix above, we get $a^2=1 \rightarrow a=\pm 1$, $d^2=9 \rightarrow d=\pm 3$, and $f^2=4 \rightarrow f=\pm 2$. Without loss of generality,we use the positive values for each. Thus 
	\begin{align*}
	b(a+d)&=8 \rightarrow b=2\\
	e(d+f)&=5 \rightarrow e=1\\
	c(a+f)+be&=5 \rightarrow c=1\\
	A^2 &= \begin{bmatrix}[rrr] 1 & 2 & 1\\ 0 & 3 & 1\\ 0 & 0 & 2\\ \end{bmatrix}
	\end{align*}
	
	
\item [3.94.] Find $x$ and $C$ if $C= \begin{bmatrix}[rr] 7 & x+1\\ 3x-7 & x-2\\ \end{bmatrix}$ is symmetric.\\
	First, we set the symmetric elements $x+1$ and $3x-7$ equal to each other to obtain $x+1=3x-7$ which gives us $x=4$. Thus $C= \begin{bmatrix}[rr] 7 & 5\\ 5 & 2\\ \end{bmatrix}$.

\item [3.95.] Write $A$ as the sum of a symmetric matrix $B$ and a skew-symmetric matrix $C$, where:
	\begin{enumerate}
	% 3.95.a
	\item $A= \begin{bmatrix}[rr] 4 & 5\\ 1 & 3\\ \end{bmatrix}$\\
	\begin{align*}
	A^T = \begin{bmatrix}[rr] 4 & 1\\ 5 & 3\\ \end{bmatrix}, \quad
	A+A^T = \begin{bmatrix}[rr] 8 & 6\\ 6 & 6\\ \end{bmatrix}, \quad
	A-A^T &= \begin{bmatrix}[rr] 0 & -4\\ 4 & 0\\ \end{bmatrix}\\
	A=(\frac{1}{2}(A+A^T)) + (\frac{1}{2}(A-A^T)) 
	&= \begin{bmatrix}8 & 2\\ 10 & 6\\ \end{bmatrix} 
	\end{align*}
	\end{enumerate}
	
	
\item [3.96.] Suppose $A$ and $B$ are n x n symmetric matrices. Show that each of the following is symmetric:
	\begin{enumerate}
	% 3.96.a
	\item $A+B$
	\begin{proof}
	Assume that $(A+B)^T$ is symmetric. By a law of transposes, $(A+B)^T = A^T+B^T$. Since $A^T$ and $B^T$ are symmetric then $A^T=A$ and $B^T=B$. Therefore $(A+B)^T = (A+B)$ and so $A+B$ is symmetric.
	\end{proof}
	
	% 3.96.b
	\item $kA$ for any scalar $k$
	\begin{proof}
	Assume that $kA=(kA)^T$. By a law of transposes, $(kA)^T = k(A^T)$. Since $A$ is symmetric, then $A^T=A$ and so $k(A^T) = kA$. Therefore $kA$ is symmetric for a scalar $k$.
	\end{proof}		
	
	
	% 3.96.c	
	\item $A^2$
	\begin{proof}
	Assume that $A$ is symmetric, then $A=A^T$. We show that $A^2=(A^2)^T$.\\
	Note that $A^2 = A*A$. Since $A$ is symmetric, then $A*A=A^T*A^T$ and by the distributive law,$A^T*A^T=(A*A)^T=(A^2)^T$.
	\end{proof}	
	
	\end{enumerate}


\item [3.97.] Find a 2 x 2 orthogonal matrix $P$ whose first row is:
	\begin{enumerate}
	% 3.97.a
	\item $(\frac{5}{13}, \frac{12}{13})$\\
	\[ P= \begin{bmatrix}[rr] \frac{5}{13} & \frac{12}{13}\\ x & y\\ \end{bmatrix}\]
	\begin{align*}
	\frac{5}{13}x + \frac{12}{13}y = 0 \rightarrow 5x+12y &= 0\\
	x^2 + \frac{25}{169} = 1 \rightarrow x &= \pm \frac{12}{13}
	\end{align*}
	We must observe two cases: $x= \frac{12}{13}$ and $x= -\frac{12}{13}$.
	\begin{align*}
	x=\frac{12}{13} \rightarrow \frac{5}{13}(\frac{12}{13}) + \frac{12}{13}y &= 0\\
	y &= -\frac{5}{13}\\
	x=-\frac{12}{13} \rightarrow \frac{5}{13}(-\frac{12}{13}) + \frac{12}{13}y &= 0\\
	y &= \frac{5}{13}
	\end{align*}
	This gives us two possible resulting matrices:
	\begin{align*}
	P= \begin{bmatrix}[rr] \frac{5}{13} & \frac{12}{13}\\ \frac{12}{13} & -\frac{5}{13} \end{bmatrix}, \quad
	P= \begin{bmatrix}[rr] \frac{5}{13} & \frac{12}{13}\\ -\frac{12}{13} & \frac{5}{13} \end{bmatrix}
	\end{align*}
	\end{enumerate}


\item [3.99.] Find a 3 x 3 orthogonal matrix $P$ whose first two rows are multiples of:
	\begin{enumerate}
	% 3.99.a
	\item $(3, -4)$\\
	We need to solve the unknowns such that $M^TM$ is equal to the identity matrix.
	\begin{align*}
	M= \begin{bmatrix}[r] R_1\\ R_2\\ R_3\\ \end{bmatrix}\\
	MM^T = \begin{bmatrix}[r] R_1\\ R_2\\ R_3\\ \end{bmatrix}
	\begin{bmatrix}[rrr] R_1 & R_2 & R_3\\ \end{bmatrix}\\
	&=
	\begin{bmatrix}[rrr]
	R_1R_1^T & R_1R_2^T & R_1R_3^T\\
	R_2R_1^T & R_2R_2^T & R_2R_3^T\\
	R_3R_1^T & R_3R_2^T & R_3R_3^T\\
	\end{bmatrix}\\
	&=
	\begin{bmatrix}[rrr]
	R_1 \cdot R_1 & R_1 \cdot R_2 & R_1 \cdot R_3\\
	\vdots & \vdots & \vdots\\
	\vdots & \vdots & \vdots\\
	\end{bmatrix}\\
	&=
	\begin{bmatrix}[rrr]
	1 & 0 & 0\\
	0 & 1 & 0\\
	0 & 0 & 1\\
	\end{bmatrix}
	\end{align*}
	\end{enumerate}


\item [3.100.] Suppose $A$ and $B$ are orthogonal matrices. Show that $A^T$, $A^{-1}$, and $AB$ are also orthogonal (Pick one).\\
\begin{proof}
Assume that A is orthogonal. Then A is square and $A^TA=I$, implying that A can be inverted and $A^{-1}=A^T$. By multiplying by A on the left, $AA^{-1} = AA^T$ or $I=AA^T$.\\
Notice that $(A^T)^T = A$, implying that $I=(A^T)^TA^T$.
\end{proof}

\end{enumerate}




















\end{document}